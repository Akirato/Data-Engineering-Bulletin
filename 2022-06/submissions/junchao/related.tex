\section{Related Work}
The age-old problem of \BFT{} consensus found first practical implementation 
through the introduction of \pbft~\cite{pbftj}.
However, \pbft{} yields a three-phase consensus, of which two necessitate quadratic 
communication complexity among the replicas.
Since then there have been several new \BFT{} protocols that aim to improve the 
performance of \pbft{} by reducing number of messages or phases~\cite{sharper,fireledger,ahl,sbft,poe,rcc,geobft,flexitrust,zyzzyva,ringbft,mirbft,basil,hotstuff}.
These \BFT{} protocols lie at the core of permissioned blockchain applications~\cite{xoxfabric,blockchain-book,bc-processing,blockplane,scalable-ledger}.
Several recent works extensively analyze the distinct properties of these protocols~\cite{bedrock,blockbench,bft-review}.

The introduction of Bitcoin~\cite{bitcoin} also led to the surge in blockchain applications 
that employ permissionless~\cite{red-belly,bc-processing} and \PoW{} consensus~\cite{pow}. 
As discussed, \PoW-based applications are computationally expensive and yield low throughputs. 
However, they help in designing secure ledgers. 
Hence, in the paper, we present the vision of our \DualChain{} architecture, which first commits a 
transaction through a \BFT{} consensus and then notarizes it using our novel \PoC{} protocol.
