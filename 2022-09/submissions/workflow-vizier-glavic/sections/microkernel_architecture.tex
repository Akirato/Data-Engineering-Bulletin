%!TEX root=../2022_IEEE_DEB_Vizier.tex
\section{Microkernel Notebooks}
As we outline above, a typical computational notebook relies on a kernel, a long-lived interpreter for a scripting language for a scripting language like python that retains the notebook's intermediate state.
When a cell is executed by the user, its contents are evaluated by the long running interpreter; the interpreter's state changes, and any output produced by the cell (e.g., console logs, charts, or maps) is displayed alongside the cell.
This behavior is independent of the order in which the cell appears in the notebook: The user may return to an earlier cell and modify it, but this cell is simply run against the current state of the interpreter.

Although this design allows users to revise cells in the notebook without being forced to re-run all of the notebooks code from scratch, it does pose several problems.
Most notably that it forces users to reason about the internal state of the kernel, for example by manually adopting a single static assignment variable allocation pattern.
Second, it also requires users to manually keep track of how different notebook cells relate to one-another; When a cell is modified, other cells that depend on it may also need revision.
Finally, in this design, persistence (e.g., of the results of a slow-running computation) must be managed entirely by the user.
Moreover, it is up to the user to manually manage this state to ensure consistent versioning, and portability.

One class of systems including Vizier~\cite{BS20,BB19} and Nodebook address these challenges by checkpointing global notebook state in between cell executions and restoring it when a cell is re-executed.
Thus, the cell is always evaluated on the state version emitted by the preceding cell, and changes to the state can be identified so that subsequent cells that depend on modified outputs can be re-evaluated.
This model 


\begin{itemize}
	\item Standard API for interacting with notebook state facilitates multi-modality
	\begin{itemize}
		\item No $N^2$ problem like for jupyter kernels
		\item Not just notebooks
	\end{itemize}
	\item Typed API allows fine-grained provenance
	\begin{itemize}
		\item Reproducibility
		\item Automatic refresh
		\item No hidden (mystical) dependencies
	\end{itemize}
	\item Challenges:
	\begin{itemize}
		\item Communicating state between kernels
		\item 2-dimensional version model (cell-order vs historical-order <- better name for this exists)
		\item Input/output changes in history vs Operation changes in history
		\begin{itemize}
			\item Vizier: Module vs Cell vs Result
			\item The Vizier execution state-model
		\end{itemize}
		\item Scheduling / Deciding state
	\end{itemize}
\end{itemize}

%%% Local Variables:
%%% mode: latex
%%% TeX-master: "../2022_IEEE_DEB_Vizier"
%%% End:
