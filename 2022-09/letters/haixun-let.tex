\documentclass[11pt]{article} 

\usepackage{deauthor,times,graphicx}
%\usepackage{url}
%\usepackage{hyperref}

\begin{document}
Curated by Sudeepta Roy and Jun Yang, the latest issue of the Bulletin
focuses on an unusual topic: education and tools for data management
practitioners. This topic typically does not receive a lot of
attention, although its significance has increased dramatically as the
impact of data engineering has grown. This issue fills the gap. For
instance, Chandra and Sudarshan introduced a system that supports
students in learning SQL in the first article of the issue.  We were
aware for many years that SQL could be difficult to use. But
previously, SQL was mostly used by a small number of database
administrators or data management engineers. Today, with the rise of
data science, SQL is used by business analysts, data engineers, data
scientists, software developers, machine learning engineers, and
virtually anybody else whose work requires access to data. In
addition, SQL queries have become sophisticated and difficult to
interpret, and improper use of SQL can not only lead to inaccurate
results but also incur exorbitant data warehouse expenditures. It is
therefore essential that we comprehend the obstacles and prepare
ourselves and the field properly.

This issue also includes an opinion piece titled "A bird's eye view of
the fourth industrial revolution" by Kyu-Young Whang et al. Although
the concept of the fourth industrial revolution (4IR) is not new, the
post-Covid world and the new technologies that have emerged in
response to this new world have given the term a new meaning. The
article by Kyu-Young Whang et al. provides a timely assessment of the
prospects and obstacles in this field based on an informative
categorization of 4IR technologies.

I would like to congratulate Joy Arulraj from Georgia Tech. and
Leilani Battle from Univ. of Washington for winning the 2022 IEEE TCDE
Award. Arulraj receives the Rising Star Award for his contributions to
the design of query processing engines for non-volatile memory and
video database systems.  Battle receives the Rising Star Award for her
interactive data-intensive systems for exploratory data analysis. The
letters shared their experience and their perspectives.

\end{document}
