\documentclass[11pt]{article} 

\usepackage{deauthor,times,graphicx}
%\usepackage{url}
\usepackage{hyperref}

\begin{document}
A year ago, we devoted a special issue of the Data Engineering
Bulletin to the state-of-the-art work of understanding the different
types of bias that is created, reinforced, and propagated by AI
systems.

We would like to dive into a broader topic, Ethics of Artificial
Intelligence, which is concerned with how humans interact with
(design, use, and treat) AI systems in general. Shimei Pan and James
Foulds put together the current issue--Responsible AI and Human-AI
Interaction–to provide an overview of the recent efforts in this
domain. The issue consists of five papers from leading researchers in
the AI and the Human-Computer Interaction communities, and covers
topics such as how people perceive AI, how to improve AI’s
transparency and interpretability, and how to foster effective
collaborations between humans and machines.

Unlike our typical issues of the Data Engineering Bulletin that mostly
focus on data management or data-driven challenges and solutions, this
issue brings up and highlights human-centered approaches, which we
believe are of increasing importance in the new era of computing. The
human-centered initiative provides a new angle for us to consider
questions ranging from whether AI is a threat to human exisitence or a
promise for humans to achieve unprecedented levels of creativity and
productivity to what is the role of Ethics of AI in the broad field of
artificial intelligence and computing.



%% In a recent issue, we looked into the bias that AI systems create,
%% reinforce, and propagate. Ethics of AI is a much broader topic, and
%% in this issue, we explore multiple topics that include loo

%% Data is often the channel through which bias transmits from human
%% experience to machines. For example, a training dataset with inherent
%% societal bias, or simply a limited and unrepresentative training
%% dataset, may lead to machine learning models that are unfair and
%% biased towards a specific class. However, detecting and correcting
%% unfair bias in AI systems are much more than just a data management
%% task. In particular, since bias starts with humans and will eventually
%% affect humans, addressing unfair bias requires insights from
%% psychological, societal, and cultural perspectives.

%% Shimei Pan and James Foulds put together the current issue --
%% Responsible AI and Human-AI Interaction -- that consists of five
%% papers from leading researchers in multidisciplinary areas.
\end{document}

