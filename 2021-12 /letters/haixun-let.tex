\documentclass[11pt]{article} 

\usepackage{deauthor,times,graphicx}
%\usepackage{url}
\usepackage{hyperref}

\begin{document}

Over the past few years, we have come to realize that a great danger
of technology is the unfair bias that AI systems create, reinforce,
and propagate. While the source of such bias is usually human,
machines' homogeneity and efficiency could easily magnify its damage,
a concern further exacerbated by the fact that AI systems are being
trusted to make more and more critical decisions in almost every
aspect of human experience.

Data is often the channel through which bias transmits from human
experience to machines. For example, a training dataset with inherent
societal bias, or simply a limited and unrepresentative training
dataset, may lead to machine learning models that are unfair and
biased towards a specific class. However, detecting and correcting
unfair bias in AI systems are much more than just a data management
task. In particular, since bias starts with humans and will eventually
affect humans, addressing unfair bias requires insights from
psychological, societal, and cultural perspectives.

James Foulds and Shimei Pan put together the current issue --
Interdisciplinary Perspectives on Fairness and Artificial Intelligence
Systems -- that consists of six papers from leading researchers in
multidisciplinary areas. It is an excellent collection of ideas and
best practices on this important topic.

\end{document}

