Recently, researchers and developers have begun to turn their attention to the optimization, 
usability, and scalability of dataframes as the community begins to recognize its important role in data exploration and analysis. 
Some of these issues are brought on by the increasingly complex API and ever-growing data sizes. 
Pandas itself has best practices for performance optimization embedded within
its user guide~\cite{pandas-opt}. However, enabling these optimizations often requires a change to the user's behavior.

Many open source systems attempt to provide improved performance or scalability for dataframes. Often, this means only supporting
dataframe functionalities that are simple to parallelize (e.g., Dask~\cite{Dask}),
or supporting only those operations which can be represented in SQL 
(e.g. Koalas~\cite{koalas} and SparkSQL~\cite{sparksql-func}). Our project, Modin~\cite{Modin}, is the only open source
system with an architecture that can support all dataframe operators.



In the research community, there are multiple notable papers that have tackled dataframe optimization through vastly different approaches. Sinthong et al. propose AFrame, a dataframe system implemented on top of AsterixDB by translating dataframe APIs into SQL++ queries that are supported by AsterixDB~\cite{sinthong2019aframe}. 
Another work by Yan et al. aims to accelerate EDA with dataframes by ``auto-suggesting'' data exploration operations~\cite{yan2020auto}. 
Their approach has achieved considerable success in predicting the operations that were actually carried out by users given an observed sequence of operations. More recently, Hagedorn et. al. designed a system for translating pandas operations to SQL and executing on existing RDBMSs~\cite{hagedornputting}. In a similar vein, Jindal et. al. built a system called Magpie for determining the optimal RDBMS to execute a given query~\cite{jindalmagpie}. Finally, Sioulas et. al. describe techniques for combining the techniques from recommendation systems to speculatively execute dataframe queries~\cite{sioulasaccelerating}. 

Our proposed approach draws upon a number of well established techniques from the systems, PL, and DB communities. Specifically, determining and manipulating the DAG of operators blends control flow and data flow analysis techniques from the PL community~\cite{cooper2011engineering}. 
The optimization of dataframe operators draws inspiration from battle-tested database approaches such as predicate pushdown, operator reordering, multi-query optimization, and materialized views~\cite{hellerstein2007architecture}, as well as compiler optimizations such as program slicing and common subexpression elimination. Furthermore, we borrow from the systems literature on task scheduling to take enable asynchronous execution of dataframe operators during think time.
