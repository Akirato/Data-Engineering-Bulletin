\documentclass[11pt]{article} 

\usepackage{deauthor,times,graphicx}
%\usepackage{url}

\begin{document}

Software applications that learn from data using machine learning (ML) are being deployed in increasing numbers in the real world. Designing and operating such applications introduces novel challenges, which are very different from the challenges encountered in traditional data processing scenarios. ML applications in the real world exhibit a much higher complexity than ``text book'' ML scenarios (e.g., training a classifier on a pre-existing dataset). They do not only have to learn a model, but must define and execute a whole ML pipeline, which includes data preprocessing operations such as data cleaning, standardisation and feature extraction in addition to learning the model, as well as methods for hyperparameter selection and model evaluation. Such ML pipelines are typically deployed in systems for end-to-end machine learning, which require the integration and validation of raw input data from various input sources, as well as infrastructure for deploying and serving the trained models. The system must also manage the lifecycle of data and models in such scenarios, as new (and potentially changing) input data has to be continuously processed, and the corresponding ML models have to be retrained and managed accordingly. The majority of these challenges have only recently begun to attract the attention of the data management community. 
%
A major obstacle is that the behavior of ML-based systems heavily depends on the consumed input data, which can rapidly change, for example due to changed user behavior or due to errors in external sources that produce the inputs. This area represents a gap between the data management and ML communities: research in ML mostly focuses on learning algorithms, and research in data management is mostly concerned with data processing and integration. In this issue, we focus on this gap in data validation for machine learning, and provide perspectives from both the academic and industrial research communities to learn about the state of the art, open problems and to uncover interesting research directions for the future.

The first paper presents \textit{A Data Quality-Driven View of MLOps} and demonstrates how different aspects of data quality propagate through various stages of machine learning development. It connects data quality to the downstream machine learning process, an approach that is also taken by our second paper, which argues that we should move \textit{From Cleaning before ML to Cleaning for ML}. The authors propose an end-to-end approach to take the entire application's semantics and user goals into account when cleaning data, instead of performing the cleaning operations in an isolated manner beforehand.

The next two papers on \textit{Validating Data and Models in Continuous ML pipelines} and \textit{Automated Data Validation in Machine Learning Systems} from Google and Amazon provide us with an industry perspective on the area in the focus of this issue. The first paper describes tools developed at Google for the analysis and validation of two of the most important types of artifacts: Datasets and Models. These tools (which are part of the Tensorflow Extended Platform) are currently deployed in production at Google and other large organizations, and are heavily inspired by well-known principles of data-management systems. The second paper from Amazon reviews some of the solutions developed to validate data at the various stages of a data pipeline in modern ML applications, discusses to what extent these solutions are being used in practice, and outlines research directions for the automation of data validation.

The subsequent paper on \textit{Enhancing the Interactivity of Dataframe Queries by Leveraging Think Time} focuses on the highly exploratory and iterative nature of data validation in the early stages of an ML application, where data scientists start with a limited understanding of the data content and quality, and perform data validation through incremental trial-and-error. The final paper of this issue on \textit{Responsible AI Challenges in End-to-end Machine Learning} completes the view on data validation for machine learning by connecting it with pressing issues from the area of responsible data management.\\ 

Working on this issue has been a privilege for me, and I would like to thank the authors for their  contributions.


\end{document}
