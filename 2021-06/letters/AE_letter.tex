\documentclass[11pt]{article} 

\usepackage{deauthor,times,graphicx}
%\usepackage{url}

\begin{document}

The global e-Commerce market size is valued at USD 9.09 trillion with an annual growth rate of 14.7%. The 2020 pandemic dramatically changed people's lifestyles. E-Commerce will further accelerate its growth and penetration into people's daily lives. E-Commerce websites and apps are among the top visits of everyone's daily routine. Customers want E-Commerce websites and apps as their personal assistant that finds the exact products they are searching for, provides recommendations when they are not sure which products to buy, and answers questions about product details. Extracting structural knowledge about e-Commerce products from their text descriptions, images, reviews, customer interaction logs is the key for building delightful shopping experience for search, recommendation, advertising, and product QA. Many challenges in building a product knowledge base can benefit from the learnings of building a semantic web. On the other hand, the unique data in e-commerce can spike new research directions in the web conference community. 

%
This special issue presents some recent work from both industry companies and academy on the issues in E-Commerce. The first two papers how structured data and knowledge about product can help improve product understanding and ranking. The next two papers talk about how to model user preference to build tailored and personalized experience in E-Commerce. Deep Learning is the new trend. The last two papers highlights how classical problems are solved with new deep learning models now. 


Working on this issue has been a privilege for me, and I would like to thank the authors for their  contributions.


\end{document}