\documentclass[11pt]{article} 
\usepackage{deauthor,times,graphicx} 
\usepackage{cancel}
% %%%%%%%%%%%%%%%%%%%%%%%%%%%%%%%%%
 \def\textit #1{{\it #1}}
  \def\mt{\tt}
  \newcommand{\bldl}{\smallskip\[\begin{array}{ll}}
 \newcommand{\cldl}{\[\begin{array}{ll}}
 \newcommand{\eldl}{\end{array}\]\rm}
 \newcommand{\prule}[2]{ \mt #1 \leftarrow & \mt #2 \\}
 \newcommand{\pfact}[2]{ \mt  #1 &  \mt #2 \\}
 \def\pbody#1#2{ \mt #1 & \mt #2 \\}
 \def\inv{\vspace{-0.2cm}}
 \def\sinv{\vspace{-0.1cm}}
 \def\pin{\vspace{0.1cm}}
 \def\magg#1{ min \langle #1 \rangle}
 \def\inv{\vspace{-0.2cm}}
 \def\sinv{\vspace{-0.1cm}}
 \def\pinv{\vspace{0.1cm}}
  \def\rof#1{$\tt r_{#1}$}
 \def\prem{$\cal P$$\!reM$~}
 % \def\Prem{{\large $\cal  P$$\!reM$\xspace\xspace}}
\def\Fix{$T_\gamma^{\uparrow n}(\emptyset)= T_\gamma^{\uparrow n{+}1}\!(\emptyset)$ }
\def\f--{\tt{ \_\!\_\;}}
 \begin{document}
%\title{Developing Big-Data Application as Queries:\\  an Aggregate-Based approach}
%a user-friendly unifying approach}
%\author{Carlo Zaniolo,$\;$Ariyam Das,$\;$Jiaqi Gu,$\;$Youfu Li,$\;$ Mingda Li,$\;$Jin Wang\\
%{\normalsize  University of California at Los Angeles}}\vspace{-0.1ex}
 %\email{\{zaniolo,ariyam,youfuli,limingda,jinwang\}@cs.ucla.edu\vspace{-2ex}}}
 
%\maketitle
 %\inv\inv
 \begin{abstract}
Recent advances on query languages (QLs) and DBMS suggest that their traditional role
 in   application development can and should be extended   dramatically  in 
 many big-data application areas, including graph, machine learning and data mining applications.
This  is made possible   by the  superior expressive power that 
database aggregates bring to  recursive queries  and the realization of their 
powerful  non-monotonic semantics via  efficient and scalable fixpoint-base
operational semantics.  Thus, in this paper, we discuss how 
 classical  algorithms can be expressed concisely using queries with aggregates in recursion 
that have a rigorous declarative semantics. Then we discuss what modifications, if any, are 
 needed on such programs to have an efficient and scalable fixpoint-based operational semantics,
 whereby we can also identify  queries that are conducive to 
 bulk-synchronous and  stale-synchronous parallelism. 
\end{abstract}
\section{{\large  \bf Introduction}}
Relational DBMS and their logic-based  QLs made possible for programmers to develop
applications without having to navigate   database 
storage structures via statements written in a procedural language. 
Many initial skeptics notwithstanding, relational DBMS  proved quite effective  in terms
of usability, performance and  scalability.  In fact their success led to and was reinforced by
significant extensions, including the introduction 
of very powerful aggregate functions, such as OLAP functions that enable direct support
for descriptive analytics by SQL queries. Another important extension was the  SQL support for  recursive queries
which allows  simple algorithms, such as transitive closure, to be expressed directly as  queries.
However, the quantum leap in expressive power achievable  by  combining recursive queries 
with aggregates was never realized  because of  SQL stratification requirement,
which specifies that non-monotonic constructs can be applied to the results of 
recursive definitions but cannot be used in the recursive definitions. This
requirement was  then enforced to avoid the major semantic problems faced by recursive reasoning
via non-monotonic constructs. However, significant progress was made since then
 by researchers  focusing on the use of aggregates in  AI,  logic programming and Datalog: 
 for instance, the concept of Stable Models has gained wide acceptance as the formal basis
 for declarative semantics  in the logic programming arena~\cite{gl:stable} \cite{DBLP:journals/tplp/GelfondZ14}.
So far, however, these advances  did not have much  impact upon the database field because
of two main issues. 
The first issue is that the  non-constructive definition of  Stable Model Semantics (SMS)  for programs with negation is making 
difficult for programmers to show that their queries with aggregates 
satisfy  SMS, and the second issue is that  establishing the SMS for a program does not guarantee its efficient
constructive realization, and significant re-writing of the original program is often needed to 
implement it via fixpoint computations and  the recursive query implementation techniques of SQL DBMS, 
as well as Datalog systems.

In this paper, we describe an approach that addresses these two issues and proved successful in a number of advanced 
applications~\cite{bigdatalog,bigdatalog-mc,rasql,afrati2011map,rasqldemoSigmod20,datalogml,kddlog}. 
We will start with an intuitive  treatment of the declarative SMS  of
 recursive queries with  extrema,  and show that queries with count, sum and average can
 be reduced to queries with max. Then, we provide simple criteria to detect when
the SMS of such queries can be turned directly into an efficient and scalable fixpoint computation
and when these instead require   significant rewriting  by the techniques described in the paper.
While in our examples we use Datalog programs, we will show how these can be expressed using
SQL queries for which the same conclusions apply. Throughout the paper, we will refer to 
queries and programs as synonyms.

\section{{\large \bf Stable Model Semantics and Fixpoint Computation for Programs with Extrema}} \label{ex:stratified}

A  simple application of extrema in recursive queries consists in finding the min or max distance
from a given initial node $\tt a$ of all  nodes in the graph where the edges have positive length.
The following program, computing max distances,  exemplifies  key semantic issues.

\sinv \begin{example} [{\it A stratified program to compute the max distance from $\tt a$}  ]~\\[-0.4cm]
 \bldl
 \pfact{r_1: dist(a, 0).}{ }
   \prule{r_2: dist(Y,  Dy)}{dist(X, Dx), arc(X, Y,  Dxy),  Dy{=}Dx{+}Dxy.}
    \prule{r_3: mxdist(Y,   max \langle Dy \rangle)}{dist(Y, Dy).}
   \eldl
 \end{example}
 
\sinv Assume for instance that we have the following fact base that describes an acyclic directed graph: 
 \inv $$\tt ~~~ arc(a, b, 10) ~~~arc(a, c,  20)  ~~~arc(b, c, 18)  ~~~arc(c, d, 12)$$

\sinv Then, the {\it semi-naive fixpoint} computation on the first two rules derives the following new atoms at each step
(whereas the {\it naive fixpoint} includes the atoms produced at previous steps along with those produced at this step):\\

\sinv\inv\begin{center}

\begin{tabular}{|c  |c | c | c |}
\hline
  Step~1 & Step~2  &  Step3 &  Step 4 \\
$\tt ~~dist(a, 0)~~$ & $\tt dist(b, 10), ~~dist(c, 20) $ &$\tt  dist(c, 28), ~~dist(d, 32)$ & $\tt  dist(d, 40)$\\
\hline
\end{tabular}
\end{center}

 \noindent With the computation on the first two rules having reached fixpoint, rule $\tt r_3$ is applied next, whereby  Step~5
 produces $\tt ~mxdist(a, 0)$,  $\tt ~mxdist(b, 10)$,   $\tt ~mxdist(c, 28) $,  and $\tt  ~mxdist(d, 40)$,  while 
$\tt mxdist(c, 20)$ and $\tt mdxist(d, 32)$ are not derived since they are dominated  by 
the previous atoms\footnote{We say that $\tt atom(X1, Y1)$ dominates $\tt atom(X2, Y2)$ when
 $\tt X1=X2$~ and  $\tt Y1 >Y2$.} and thus  they are not maximal.
 
 
 Our stratified derivation can be optimized by pushing  the max constraint into recursion, while keeping the
 original stratification whereby the fixpoint computation of $\tt dist$ by rules $\tt r_1$ and $\tt r_2$ must be completed
 before  $\tt r_3$ can be used to derive $\tt mxdist$. 


\inv\begin{example} [{\it {\it Using max in recursion to compute the max distance from $\tt a$}  }]~\\[-0.4cm]
\label{ex:premopt}
    \bldl
 \sinv  \pfact{r_1: dist(a, 0).}{ }
   \prule{r_2: dist(Y,  max \langle Dy \rangle)}{dist(X, Dx), arc(X, Y,  Dxy),  Dy{=}Dx{+}Dxy.~~~~~~~~~}
    \prule{r_3: mxdist(Y,   Dy )}{dist(Y, Dy).}
   \eldl
 \end{example}
 
\sinv  The semi-naive fixpoint computation  of our rules \rof{1} and \rof{2} so revised
produces atoms that are dominated by other atoms produced at later steps.
These non-maximal atoms are called  {\em provisional max} atoms, and they are depicted  as cancelled
in the picture below, since they will be lost at later steps.

\sinv\begin{center}

\begin{tabular}{|c  |c | c | c |}
\hline
  Step~1 & Step~2  &  Step3 &  Step 4 \\
$\tt ~~dist(a, 0)~~$ & $\tt dist(b, 10), ~~\cancel{dist(c, 20)} $ &$\tt  ~~dist(c, 28), ~~\cancel{dist(d, 32)}$ & $\tt  dist(d, 40)$\\
\hline
\end{tabular}
\end{center}
%The $\tt dist(\f--, \f--)$ 
%Atoms that are not provisional   called {\em final max} atoms. 
Max (min) atoms that are not provisional are called {\em final} max (min) atoms.
Then, with the fixpoint computation for $\tt dist$  completed Step~4,  \rof{3} at Step 5  simply renames  the final max atoms so obtained 
 producing
 $\tt ~mxdist(a, 0)$,  $\tt ~mxdist(b, 10)$,   $\tt ~mxdist(c, 28) $ and $\tt  ~mxdist(d, 40)$.

The minimal model of  a monotonic program can be derived by an  {\em eager} computation
that, at each step, derives all the possible consequences of the current interpretation. A derivation that at each step only
derives a non-empty subset of those consequences will be called {\it judicious}.  Then,  we have that
 {\it  a program with extrema has $M$ as its stable model  iff
$M$  can be derived via a judicious derivation that contains no provisional atom}~\cite{submitted}.  


For instance, the stratified derivation shown for Example~1 is a judicious one since the fixpoint computation of $\tt dist$
by rules \rof{1} and \rof{2}  produces no provisional atom, and  \rof{3}  then  selects from the atoms so generated
the final max $\tt mxdist$ atoms. Indeed, the iterated fixpoint computation of every max-stratified program defines a
judicious computation that produces its stable model.

On the other hand, although the eager  computation in Example~2 contains  provisional atoms,
it still computes the stable model of our program. Indeed, after all those provisional atoms 
are cancelled during the fixpoint computation, we  have a valid judicious derivation for the 
stable model of our program.  Programs that have this property will be said to be {\it resilient}.
The program in Example~2 is resilient.

The requirement that programs must have SMS is now widely accepted  in the field because otherwise
programs might not have sound logic-based semantics, as illustrated by the following example.
Say, for
 instance, that  users are only interested in nodes whose
distance from $\tt a$  along the longest path is  $\tt <40$, and they add  the condition  $\tt Dy {<}40$  to  
$\tt r_3$ in Example~\ref{ex:premopt}. The modified program is still stratified  and returns $\tt ~dist(a, 0)$,  $\tt ~dist(b, 10)$,   $\tt ~dist(c, 28) $
$\tt  dist(d, 40)$ and  $\tt ~mxdist(a, 0)$,  $\tt ~mxdist(b, 10)$,   $\tt ~mxdist(c, 28) $, which defines its stable model. \\[-0.3cm]

\noindent
However,   if $\tt Dy{<}40$  is added to $\tt r_2$, instead of  \rof{3}, then the  eager
computation  generates the following derivation:\\[-1.6cm]

\begin{quote}~\begin{example} [{\it Eager computation for Ecample~2 after $\tt dist(c, 20)$ is added to \rof{2}}]
~\\[-0.2cm]
%\begin{center}

\begin{tabular}{|c  |c  | c |}
\hline
  Step~1 & Step~2  &  Step~3 \\
$\tt ~~dist(a, 0)~~$ & $\tt dist(b, 10), ~~\cancel{dist(c, 20)} $ &$\tt  ~~dist(c, 28), ~~dist(d, 32)$\\
\hline
\end{tabular}
\end{example}
\end{quote}
%
\sinv Then in Step~4,~ \rof{3} derives from these: $\tt mxdist(a, 0)$, $\tt mxdist(b, 10),$ $\tt  mxdist(c, 28), mxdist(d, 32)$.\\
The logical problem with  this outcome is that 
the  atom ~$\tt dist(c, 20) $ is no  longer in the result;   thus, there is no justification for 
$\tt dist(d, 32)$  and $\tt mxdist(d, 32)$  to be  in the answer.

Therefore, for  Datalog programs with aggregates to produce  logically sound  results, the programs
must  have SMS. We  will  first discuss how  to achieve this objective for programs
with max and min,  and then for programs with count, sum and average whose formal semantics
is actually defined using max.

\paragraph{Pre-Mappable Extrema and Stable Model Semantics:}
In our previous examples, we have seen that an eager fixpoint computation produces 
a stable model for some queries but not for others.   Thus programmers need simple criteria to 
guarantee that their programs belong to the  first group (i.e., they are resilient) and  the notion of  pre-mappable   (\prem)  constraints 
\cite{DBLP:journals/tplp/ZanioloYDSCI17} is introduced next to satisfy this requirement. 

The mapping defined by a set of rules in a program is called their Immediate Consequence Operator (ICO) 
and it is  denoted by $T$.
Then, the constraint $\gamma$ is said to be pre-mappable ($\!$\prem{\bf)} to
 $T$  when, for every interpretation $I$ of
 $P$, we have that: $\gamma(T(I)) = \gamma (T(\gamma (I)))$.
 
 For instance,  in rule  \rof{2}  of Example~1,  to find  the
 maximal $\tt Dy$ for a given $\tt Y$, we only need to consider the max $\tt Dx$ value for the $\tt X$ value that,
 via  $\tt arc(X, Y, Dxy)$, produces that $\tt Y$.
  Thus, in Example~2, the constraint $\tt is\_max(Y, Dy)$ applied to
 the head of \rof{2} can pre-applied to $\tt  dist(X, Dx)$ in the body of the rule, and from there to 
 the rules generating  $\tt  dist(X, Dx)$ in the previous step of the fixpoint computation without changing 
 the results of the fixpoint computation. This \prem property  allows us to transform  Example~1 into Example~2
with assurance that the same $\tt mxdist$ results will be obtained.
More recently, it was proved~\cite{submitted} that \prem also guarantees that  recursive programs with extrema
satisfying \prem have a SMS that can be computed using the fixpoint techniques now used in the efficient implementation of
recursive queries by SQL and Datalog systems. 
General conditions for  \prem to hold is a given program were discussed in \cite{DBLP:conf/amw/ZanioloYIDSC18}.

The \prem property provides a very useful sufficient condition for
 recursive queries with aggregates to have a declarative SMS that can
 be computed quite efficiently using the techniques currently used 
 for monotonic queries.  Nevertheless, many queries  have SMS 
 although they do not satisfy \prem$\!\!.$
For instance,  consider Example~2 with  $\tt Dy \!<\!40$ added to \rof{2}. 
Then, instead of using an eager derivation we can omit  deriving  $\tt dist(c, 20)$
at Step~2, and derive $\tt dist(c, 28)$ at Step~3: this completes a judicious derivation
free of provisional atoms that thus delivers its stable model.  In this revised program, the \prem property is lost and and
 eager fixpoint derivation does not produce its stable model, whereby other techniques,
 such as those discussed in the next sections are needed to compute it.

\sinv\paragraph{Programs with MIN.}
The formal properties of programs with min can be derived by duality from those of programs with max
and imply that  many min programs of practical interest  have SMS. For instance, if we revise our examples
to use min instead of max,  we find that \rof{2} still satisfies \prem after the addition of  $\tt Dx \!< \! 40$, and therefore
an eager fixpoint computation can be used to compute its stable model.  Moreover, the presence
of cycles of positive length  in $\tt graph(\f--, \f--)$ does not compromise the SMS of our min programs, since the second time
a node $\tt Y$ in the cycle is visited during  derivation, its larger $\tt Dy$ value is discarded, 
whereas  in max  queries larger values turn previous values into provisional ones.  
 A wide  range of classical graph algorithms can be expressed concisely 
 using aggregates in recursive  rules that  have the \prem property  and are thus
 conducive to efficient and scalable implementations~\cite{bigdatalog,bigdatalog-mc,rasql,rasqldemoSigmod20}.
In fact, along with other optimization techniques, the  semi-naive fixpoint can be extended to programs with extrema,
with the simple provision  that the new atoms generated  at each step will
replace any provisional atom they dominate.
%\paragraph{Seminairve fixpoint}

\inv\section{{\large Semantics of Programs Using the Continuous Count and Final Count Aggregates}}

\sinv 
 The OLAP functions of SQL  introduced the 
 continuous count aggregate 
which returns all positive integers up to the cardinality of the set. 
 Continuous count will be denoted by $\tt mcnt$, since it is monotonic in the lattice
of set containment, i.e. if $\tt S1 \subseteq S2$ then $\tt mcnt(S1) \subseteq mcnt(S2)$.
The  final count $\tt f\!cnt$ that computes the cardinality of  a set
can be expressed as the max of the monotonic count on the set.
For instance, with  $\tt bintbl(\f--, \f--)$ denoting an arbitrary binary table the following rule 
computes the cardinality of the set of
distinct  $\tt Y$-values associated with a group-by value $\tt X$:

\inv\inv\inv\begin{equation}
\tt r: mcagr(X, mcnt\langle Y \rangle)  \leftarrow ~ bintbl(X, Y).
\end {equation}
%In our rule, we have used the symbol $\tt f\!cnt $ do denote the final count, whereas the
% the continuous monotonic count  is denote by $\tt mcnt$. 
 The semantics of this rule is as follows:
\sinv\begin{definition} [{\it Defining  the semantics of continuous monotonic count $\tt mcnt$}   \inv]
\label{def:gbfcount}
~\\[-0.4cm]
\cldl
\prule{r_{a}: mcnt(X, 0)} {bintbl(X, \f--).}
 \prule{r_{b}: mcnt(X,  C1)}{ mcnt(X,  C), bintbl(X,Y),  onenext(Y, (X), C), C1=C+1.~~~~~}
\prule{r_{c}: mcagr(X, max\langle C \rangle)}  {mcnt(X, C).}
\eldl
 \end{definition}
 
 The predicate  $\tt onenext(Y, (X), C)$ guarantees that, for each group-by value $\tt X$ , each 
value $\tt Y$ is counted exactly once. In Datalog  this can be expressed by replacing 
 $\tt onenext(Y, (X), C)$  in \rof{b} with the pair of goals $ \tt choice((X,C), Y), choice((X, Y), C)$,
since programs that use the choice construct have SMS~\cite{DBLP:conf/pods/GrecoZG92}.
 In the actual implementations,  the use of the  $\tt get\_next$ construct, that visits the tuples in $\tt bintb$
exactly once,  guarantees that this constraint is never violated. Also observe that the zero count assigned
by \rof{a} is  always eliminated by \rof{c}. Thus the  semantics of the final count that compute the 
cardinality of the set can now be defined using $\tt mcnt$ and $\tt max$. For example, the semantics of the following rule:
\inv \cldl
\prule{f\!cagr(X, f\!cnt\langle Y \rangle)} {bintbl(X, Y).\inv}
\eldl
\inv is defined by the following two rules:
\cldl
\prule{(a): magr(X, mcnt\langle Y \rangle)} {bintbl(X, Y).}
\prule{(b): f\!cagr(X, max\langle C \rangle)} { magr(X, C).\sinv}
\eldl
% \noindent With max the only non-monotonic
%construct used in the definition of $\tt fcnt$ we observe similar  SMS patterns.
%Observe how this 
%definition uses  the predicate  $\tt new$ to prevent  duplicate counting of  $\tt Y$-values
%associated with  the same value $\tt X$---thus, duplicate $\tt [X,Y]$  pairs are ignored.  Definition~2 creates,
%for each group-by value $\tt X$, lists  containing  the permutations of the values of $\tt Y$ associated with this $\tt X$.
%This definition is logically correct since all permutations produce the same count result, a property that
%is then exploited in the optimized computation performed by the system, which wil
%use only the permutation  that is most efficiently retrieved, e.g. the one derived
%according to the order in which the data is stored~$\!\!$\footnote{
%E.g., by  the $\tt get$-$\tt next$ primitive supported in vintage Database Systems.}.  
%Furthermore, the \prem property  allow us to  apply  $\tt max$ in \rof{b}, whereby 
%only  $\tt X$ value is kept at each step for each $\tt Y$ value, as it is done in the actual implementation.
%Thus, we 
%have here a fully declarative logic-based semantics that is then  optimized into the very efficient
%operational semantics  used at  the system level.\\[-0.1cm]
% This  synergistic combination of the strengths of
%declarative and operational semantics is what we seeks to achieve
%for all programs that use aggregates, including the  situations where aggregates
%are used in  recursive  rules.
%$ \tt choice((X,C), Y), choice((X, Y), C)    onenext(Y, X, C).$


\paragraph{\bf A  Bill of Materials  (BoM)  Example.} $\!\!$Here,  $\tt ~basic(Part, Cost)~$
is  the price charged by the supplier of a  part. 
%\footnote{Example  \ref{ex:sssprime}   can also
%be re-expressed using only two rules without affecting its SMS properties.}:
 \begin{example}[$\tt basic$ {\it describes the cost of basic parts, and $\tt arc$ defines  the part-subpart graph}]

~~

$\tt  ~~~~~~~~ basic(a,6.2).~ basic(b, 9.4).~ basic(c, 13.2).~ basic(d, 4.8).~ basic(e, 4.8).$

~~~~~~~~$\tt arc(a,f).$ $\tt ~arc(b, f).~ arc(c,f).~ arc(c,g).$ $\tt arc(d,g).~ arc(e,g).~arc(f, g).~$
\end{example}

~\\[-0.5cm]
%\begin{example}[Days,  in excess of 10, needed for  all components of an assembly to be delivered]
%\label{ex:wait1}
%\inv\inv\cldl
% \prule {\rho_1: wait(Part,Days)}{basic(Part,  Days).}
% \prule{\rho_2:  wait(To, D1 )}{ wait(Frm, Days),  D1{=} Days+1, arc(Frm, To), is\_max(To, D1).}
% \prule{\rho_3:  mxwait(Part, Days)}{wait(Part, Days), Days \geq 10.}
% \eldl
% \end{example}
Assume now that we introduce the (somewhat artificial) notion of complex assemblies  as those which are
assembled from three or more components which are either basic parts  or complex
subassemblies. Then the following query returns the complex assemblies in our BoM.

\sinv \begin{example}[{\it Find the  assemblies and the total count of basic parts they contain when this is $\geq 3$}]

 \label{ex:count}
\cldl
 \prule {\rho_1: cassb(To,  3)}{basic(To, \f--), C=3.}
 \prule{\rho_2:   cardc(To, fcnt\langle  Frm \rangle)}  {cassb(Frm, \f--),  arc(Frm, To).}
 \prule{\rho_3:  cassb(To, Totcnt)} { cardc(To, Totcnt),  Totcnt \geq 3.}
 \eldl
\end{example}
The semantics of this program is defined by the following program obtained by expanding  $\rho_2$ into  rules
$\rho_{2a}$ and $\rho_{2b}$ that express
the computation of  $\tt f\!cnt$ aggregate via $\tt mcnt$ and $\tt max$:\\[-0.1cm]

 \begin{example}[{\it  Defining the  semantics of $\tt f\!cnt$ in Example~5  as the $\tt max$ of $\tt mcnt$.} ] 
 \label{ex:countbis}
\inv\sinv\cldl
 \prule {\rho_1: cassb(To,  2000)}{basic(To, \f--).}
 \prule{\rho_{2a}:   cardc(To, mcnt\langle  Frm \rangle)}  {cassb(Frm, \f--),  arc(Frm, To).}
\prule{\rho_{2b} :   mxcard(To, max\langle C \rangle)} {  cardc(To, C ).}
 \prule{\rho_3:  cassb(To, Totcnt)} { mxcard(To, Totcnt ),  Totcnt \geq 3.}
 \eldl
\end{example}

%Thus $\tt onenext(To, C, Frm)$  is predicates that assure that, for a certain group-by To, and for each only one
%value From is selected and this is a new value never selected before.  In Datalog this can be expressed
%using the choice construct, that is available under SMS,  whereby $\tt nextone( C, To, Frm)$ can be re-written as 
%$\tt choice(To, From), choice({To, C), Frm$.  Alternative ways to express the same constraint using lists have been proposed in \cite{xxx}.
%Most important, though,  the getnext construct used in the systems performs the same function whereby every node
%from is only considered once and as it is considered the count of its successors To in the BoM graph is incremented.

The max aggregate is the only non-monotonic construct in this program.  Therefore, to achieve SMS, we must find
a judicious derivation that does not produce any provisional max.  One such derivation could e.g. use
$\rho_1$ and $\rho_{2a}$ to produce
$\tt cardc(f, 3)$  and  $\tt cardc(g, 3)$.    At this point, our judicious derivation produces $\tt mxcard(f, 3)$ but not $\tt mxcard(g, 3)$.   
Then, from  
$\tt mxcard(f, 3)$  we derive   $\tt cassb(f, 3)$.  From this, we derive  $\tt cardc(g, 4)$, which yields $\tt mxcard(g, 4)$ 
and $\tt cassb(g, 4)$. Since this derivation has produced no provisional max, SMS is guaranteed. On the other 
hand, an eager computation woud have used  $\tt cardc(g, 3)$ to produce $\tt mxcard(g, 3)$ 
before deriving $\tt mxcard(g, 4)$ and  $\tt cassb(g, 4)$, which  remove the provisional  $\tt mxcard(g, 3)$
but not   $\tt cassb(g, 3)$.   Thus we cannot use  the eager fixpoint technology of our current systems
to compute the stable model for this program. As we look for a solution, we see that
there is a  simple one that consists in  revising  $\rho_3$ into the
following rule:
\sinv \inv\cldl
\prule{~~~~~~~~\rho'_3:  cassb(To, max\langle Totcnt \rangle)} { mxcard(To,  Totcnt ),  Totcnt  \geq 3.\sinv}
\eldl

The {\it max-enhanced} version of the program so obtained  has the same SMS as the original program
and its   eager fixpoint computation produces its stable model, thus we will say that our program is {\it quasi-resilient}.
While many programs of interest are quasi resilient, others are not, and no max enhancement will deliver their SMS.
For instance, say that we add the additional 
goal  $\tt Totcnt {<}4$;  then,  the stable model for this program contains   $\tt cassb(f, 3)$,  
but it does neither contains $\tt cassb(g, 3)$ nor $\tt cassb(g, 4)$. However the
fixpoint of its max-enhanced version now contains $\tt cassb(g, 3)$. Thus  the max-enhancement here
neither preserves the original SMS  nor it  makes it resilient.  To address this situation,  users
must be provided with (i)  criteria to recognize programs that can  be max-enhanced into equivalent 
programs that are resilient, and (ii)  more general implementation methods for programs that have a SMS
but do not belong to group (i). The notion of {\em Reverse Premappability} (R\prem)  provides a simple answer
to (i).  With  $T$  denoting the  the ICO of one or more rule,  we will say that R \prem  holds for $T$ if
$\gamma(T(I)) = T(\gamma(I))$ for every $I$.  Since  \prem holds for  max in rule $\tt \rho_3$,
max can be introduced into this rule  without changing the result; indeed, 
 it  only enforces the max constraint on atoms generated from rule $\tt \rho_{2b}$, i.e. atoms that already
satisfy the max constraint. However if we instead use rule  $\tt \rho'_3$, then   a max atom,   such as  $\tt mxcard(g, 4)$
obtained from    $\tt \rho_{2b}$, will be filtered out by the $<4$ condition,
 whereby  $\tt \rho'_3$ will now return $\tt mxcard(g, 3)$ as max: a clear violation of SMS.
Therefore, to realize  the SMS for the program with the ${<}4$ condition, we need
different solutions, such as the one that relies on the   {\it pre-counting}  technique discussed in the next section.
 
 \paragraph{\bf  Dealing with Duplicates.}
 Duplicates are immaterial for extrema because of their
idempotence property, but not for the other aggregates a special notation 
is needed to specify that duplicates are not excluded from the computation.
Say for instance that we have
a  ternary table $\tt terntbl(A1, A2, A3)$ and for each value of $\tt A1$ we would like  to count all occurrences 
of $\tt A2$, where every occurrences of  $\tt A2$ associated with different 
$\tt A3$ value contributes to the count. Then, instead of the following rule,
\inv\begin{equation}
\tt  dbagr(X, f\!cnt\langle [Y, Z] \rangle)  \leftarrow ~ terntbl(X, Y, Z). \sinv
\end {equation}
we can use the rule (3) below with the special duplicate notation that will also be used for sum and average:
\inv\begin{equation}
\tt dbagr(X, f\!cnt\langle Y, Z \rangle)  \leftarrow ~ terntbl(X, Y, Z).
\end {equation}

~\\[-1.8cm]

\section{{\large Queries Using Sum and Average.}}
\sinv  
The semantics of the sum of the elements in a set can be specified by 
adding up its elements while counting them so that  the sum value
associated with the final count can be returned.\footnote{Using the max of the continuous count
on the set elements does not identify their sum when negative numbers are present.}


Consider for instance the following example that for each assembly derives the total of the costs of the basic
parts it uses, by adding up those of its subparts. The notation $\tt sum\langle  Cost, Frm \rangle)$ denote that
duplicate  $\tt Cost$ from different subparts $\tt Frm$ all contribute to the sum.\\[-0.4cm]
\begin{example}[{\it For each assembly, find the total  cost of  the basic parts it uses}] 
 \label{ex:cost}
 
 ~\\[-0.2cm]
\cldl
 \prule {\rho_1: pcst(To, Cost)}{basic(To, Cost).}
 \prule{\rho_2:  ragr(To, sum\langle  Cost, Frm \rangle)}
{pcst(Frm, Cost),  arc(Frm, To).}
 \prule{\rho_3:  pcst(To, Cost)} {ragr(To, Cost).}
 \eldl
\end{example}
Now,  the computation of $\tt sum\langle  Cost, Frm \rangle$  requires (a) an initial step where
count and sum are initialized to zero, (b) an iterative step that adds one to the current count and adds
the new $\tt Cost$ to the current sum, and (c) a final max step that selects the final count, which 
is used to return the final sum value associated with it.

%Thus this query can be re-expressed in the following way:
% \begin{example}[{\it Expressing the previous example using the continuous sum value associated with the final count}]
%\inv\sinv\cldl
% \prule {\rho_1: pcst(To, Cost)}{basic(To, Cost).}
% \prule{\rho_2:  ragr(To, fcnt\langle  Cost, Frm \rangle, msum\langle  Cost, Frm \rangle)}
%{pcst(Frm, Cost),  arc(Frm, To).}
% \prule{\rho_3:  pcst(To, Cost)} {ragr(To, Cost).}
% \eldl
%\end{example}

%The special
%notation $\tt sum\langle  Cost, Frm \rangle$ in $\tt \rho_2$ is used to 
%$\tt Cost$ values are included in the computation if they are  associated with different $\tt Frm$ values.
%Thus,  both the  $ \$ 4.8$  cost of basic part  $\bf d$ and the
%$\$ 4.8$ cost of basic part $\bf e$  are  included in the computation of  the  sum   for $\bf g$.  
%In $\tt \rho_3$ we have also added the condition $\tt CST <$ to assure termination just in case
%that dirty data have created cycles in our graph.\\[-0.2cm]


%\paragraph{\bf Sum in Recursive Programs} Let us study now  the declarative and operational semantics 
%of programs that use sum in recursive computations. For instance, to add up the coast of
%the basic parts contained directly or indirectly in a super-part, we can use the following program:
%\vspace*{-0.2cm}~~


%The declarative semantics of  this program can be specified by
%its FOE which, for each  assembly $\tt To$, adds up the costs of its suparts $\tt Frm$,
%while keeping count on  how many subparts were added so far, since 
%the correct value of the sum is the one associated with the final max count.

%in Example \ref{ex:thesum}  can be specified as follows:
%The  sum of the elements in set  can be obtained aggregate is defined by assuming that, along with 
%$\tt sum$, the $\tt fcnt$ aggregate is also computed, since the  value of the sum 
%s returned when the count has reached its final value:
 \begin{example}[{\it The max-based formal semantics for  Example \ref{ex:cost}}]
 \label{ex:sumbis}
\cldl
 \prule {\rho_1: ~\; pcst(To, Cst)}{basic(To, Cst).}
\prule{ \rho_{2a}: rsum(To, C, S)}{pcst(Frm, \f--), arc(Frm, To),
 C=0,  S=0.}
 \prule{ \rho_{2b}: rsum(To,C1, S1)\!} {pcst(Frm, Cst),  arc(Frm,To),  rsum(To, C, S),}
\pbody{}{onenext(Cst, Frm, (To), C), C1{=}C{+}1 ,S1{=}S+Cst.}
\prule{\rho_{2c}: ragr(To, max\langle C \rangle)}{rsum(To, C, \f--).}
\prule{\rho_3:~\; pcst(To, Cst)} { ragr(To, Fcnt), rsum(To, Fcnt, Cst).}
 \eldl
\end{example}
%
To define the semantics of  average,  rule   $\rho_3$ above will be modified to return $\tt Cst/Fcnt$ instead
of $\tt Cst$.

As we now investigate the declarative and operational semantics of this example, we see that 
we  can assume that no cycle exists in the BoM part-subpart graph,
no provisional max value is ever generated by $\rho_{2c}$  and SMS is thus guaranteed.
%Then for each node,  its correct  final count and sum can be computed as soon as this
%information is available for all predecessors of the node. Then, with no provisional max value  generated
%SMS  is guaranteed.   
However, when it comes to operational semantics, the situation of sum is quite different from 
that of count;  this can be easily seen by comparing 
rule $\tt \rho_{2a}$ in Example \ref{ex:countbis}, where successive  count values are ignored by $\tt cassb(From, \f--)$, against
rule $\tt \rho_{2a}$ in Example \ref{ex:sumbis} in which successive values are cumulatively added  incorrectly to the sum.
Thus, except for
special cases, such as that of perfectly balanced trees, an eager computation
will not deliver the correct  sum value.
%Observe, that the only non-monotonic construct  in this program
%is max that is used to defined the count, whereby the existence of SMS
%for this program follows directly from the absence of cycles in the 
%underlying graph.  As we have previously seen, however,
%SMS  does not guarantee their resiliency for more complex count programs,
%and eager FPC cannot be used  for our program. This can be
%his can be easily seen
%by contrasting  the rules in  \ref{ex:maxcnt} against those in 
%\ref{ex:thesum}.  In fact, as per the goal  $\tt inclub(From,  \f--)$ in $\tt r_{2b}$
%of Example \ref{ex:maxcnt} the actual value of count is not used 
% the actual non-zero value of the predecessor node is immaterial in the 
% computation.  In our Example \ref{ex:thesum} instead, the 
% second argument of $\tt pcst(Frm, Cst)$ in $\tt \rho_{2b}$
% directly contributes to the sum. 
To  realize SMS, we must therefore revise the original program to 
 make sure that, for each node,  the $\tt Cst$ contributions of 
its predecessors are added all at once. 
 One technique to achieve that is  the  {\it pre-counting} approach used in~\cite{datalogml}
 which will compute the sum at a node only after the sum is computed at each of
 its immediate predecessors.
% Toward this goal we have two solutions. The first is the layering 
% solution used for count, and the other is cardinality pre-counting 
% discussed next.
To implement this technique, the program in Example \ref{ex:cost} is  revised
into that in Example \ref{ex:below}, below,  by  the addition of
rule $\bar{\rho_0}$ 
that precomputes the in-degree of each node. Then, 
$\tt \rho_2$  is modified into $\bar{\rho}_2$
to keep count of the number of a node's immediate predecessors that have so far  contributed to its sum. 
Thus, the correct final  sum value used in $\bar{\rho_3}$ is the one obtained when  count is  equal to the in-degree of the node.
In passing, observe that  $\bar{\rho}_2$  also illustrates  how multiple aggregates sharing the
same group-by value can be specified in the head of a rule.\\[-0.5cm]


 \begin{example} [{\it Cost of nodes by pre-computing the number of their incoming arcs}]
\label{ex:below}

~\\[-0.7cm]

\cldl
 \prule {\bar{\rho}_0: indgr(To, f\!cnt\langle Frm \rangle )}{arc(Frm, To).}
 \prule {\bar{\rho}_1: pcst(To, Cost)}{basic(To, Cost).}
 \prule{\bar{\rho}_2:  ragr(To, sum\langle  Cost, Frm \rangle, f\!cnt \langle  Frm \rangle)}
{pcst(Frm, Cost),  arc(Frm, To).}
 \prule{\bar{\rho_3}:  pcst(To, Cost)} {ragr(To, Cost, NNods), indgr(To, NNods).}
 \eldl
\end{example}

Thus we have now a program where  the eager FPC realizes its SMS,
returning the same $\tt pcst(To, Cost)$ atoms as the original program.  
Pre-counting  is also applicable to programs with count and average and
 represents a simple technique to derive an efficient and scalable fixpoint computation for
programs that have a declarative SMS. However, in the simple form we have discussed, pre-counting
relies on the assumption that all the nodes are reachable form basic parts. When
that is not the case, more complex programs could be used to compute the actual
count of immediate predecessors of a node reachable form the basic parts. Alternatively, 
the Group-by Layering  technique described  in \cite{submitted}  can be used   to avoid 
the generation of provisional values, by delaying the derivation of each group-by node 
 until a derivation step where the last of its predecessors is computed. In acyclic graphs, this step
could be determined using topological sorting, but this requires a complex Datalog
program and  implies the serialization of  nodes that could be computed in parallel.
The technique described in ~\cite{submitted} instead computes for each node 
its maximum distance  from $\tt basic$ nodes and is amenable to parallelism.

\inv \section{\large Parallelism and Layered Computation.}
\sinv  By applying pre-counting  on Example \ref{ex:cost},   we obtained Example~9  
where the computation a new node is started only after it is completed at its
predecessors. In many
applications of interest, this kind  of revision  is not  needed since  the
completion conditions that enable  the fixpoint computation  to realize SMS follow directly
from the structure of the program and  the parallelization  strategy used by the system. 
 Among such applications, we find those using algorithms such as Markov-Chains,  Lloyd's Clustering and Batch Gradient Descent.

\inv\paragraph{\bf Markov Chains.} This algorithm is interesting because of its
similarity to the Page Rank algorithm that led to Map-Reduce.
 % and because  it provides an interesting situation where the sum 
%is applied to a set of known cardinality,
 Thus,  in Example \ref{ex:markov}, we assume a database of facts
 $~\tt mov(Frm, To,  Perc)$,  where $\tt Perc$ denotes
 the  fraction of population that every year relocates from city $\tt Frm$  to city $\tt  To$.  
For each city, there is also a non-zero arc from the city back to itself showing the fraction of people who remain in the same city.
Therefore, the sum of $\tt Perc$  for the  arcs leaving a city (i.e., a node) is equal to one. 
 
Now, assuming that initially every city has a population of $100,\!000$ people we would like to
 determine  how the  population evolves over the years.    Also  to assure termination, we stop the computation after
 999 steps (i.e., a number of steps that is normally sufficient  for the computation to either
 converge to a final state or reveal that no convergence should be expected).
 Then, we can use the following program, where  $\tt  sum\langle  In, Frm \rangle$
  adds up the
 $\tt In$  contributions from all $\tt Frm$ cities,  without discarding  duplicate  $\tt In$ values  coming from different $\tt Frm$ cities:\\[-0.1cm]

 \begin{example}[The  Markov Chains algorithm]
\label{ex:markov}
\inv\inv \cldl
\prule{\! \rho_1:markv(1, Cit, Pop)} {\!mov(Cit,  To, \f-- ), Pop= 100000.}
\prule{\!\rho_2: next(J, To, sum\langle  In, Frm\rangle)}{\!markv(J, Frm, Pop),  mov(Frm, To, Perc),In = Pop {\times} Perc .}
\prule{\! \rho_3: markv(J1,  Cit, Pop)}{\!next(J, Cit, Pop),  J\leq 999, J1=J+1.}
\eldl
\end{example}
Observe that we have here a program that is locally stratified by the 
value of the first argument in $\tt markv$, whereby SMS hold. The same conclusion
follows by representing as a directed arc the dependency from 
 the group-by arguments $\tt markv(J, Frm, \f--)$ in  $\rho_2$ to 
$\tt markv(J1,  Cit, \f--)$ in the head of $\rho_3$. Since $\tt J1=J{+}1$ the
graph so established is acyclic, and this provides yet another 
proof that  declarative SMS holds.

Now, for an operational semantics that correctly realizes the SMS of this program, we can observe that 
the number of incoming arcs representing the population that
migrate into a city each year remains the same, and thus we can use
the pre-counting approach  previously described. Then, the  head
$ \rho_2$ will be extended with an additional argument that counts the number of  contributions added so far.   
Them,  $ \rho_3$  ignores the sum values  produced by $ \rho_2$, until the count argument 
equals the pre-counted in-degree of the node. 
The stable model for the program generated by this pre-computing transformation is  then computed by 
the eager derivation  that is conducive to scalability via stale-synchronous parallelism \cite{ssp-iclp19,submitted}.

However, other solutions are available and actually preferable 
in Datalog systems that support more advanced operational semantics.
 Indeed some Datalog compilers \cite{ldl++} will
recognize the explicit local stratification that holds for the
group-by argument in our rules, and arrange for an evaluation where  all computations at level
$\tt J1{=}J{+}1$ are performed after
the $\tt J^{th}$-level ones are completed. In these Datalog systems, the stable model for Example \ref{ex:markov} 
can computed with no revision required in the program.
The same conclusion holds for systems
 designed to support bulk-synchronous 
parallelism (BSP)  in which a new distribution-computation
cycle at level $\tt J{+}1$  is not started  until the completion of the previous 
cycle,  in which aggregates at group-by level $\tt J$ were evaluated. This approach achieves high parallel performance \cite{kddlog}.

\inv\inv\paragraph{\bf Applications  Requiring a Combination of Different Aggregates.}
%In the previous sections, we explored the semantic properties of  different  aggregates,
%and applied them in different programs  to provide a concise formulations
% for several  useful algorithms having SMS and efficient operational semantics.
 Queries that combine multiple aggregates 
can express concisely a  broad spectrum of advanced algorithms that support
graph, data mining and ML applications with superior performance and scalability
~\cite{seo2013socialite,wang2015asynchronous,bigdatalog,rasql,appl-iclp19,bigdatalog-mc,rasqldemoSigmod20,datalogml,kddlog}. 
%Many of these applications rely on a combination of different aggregates
%and the fact that they all share in their cardinality-based definition simplifies
%the study of their formal semantics and the proof that their efficient and SSC-scalable
%fixpoint produces a stable model.   
For instance,  we next discuss  Lloyd's clustering algorithm  that combines sum, average and min aggregates.

\paragraph*{\bf K-means Clustering:}
 \label{sec:lloyd}
 We are given a large set of D-dimensional points.  
Each point is described by a unique $\tt Pno$ and its  coordinate values in each of the $D$
dimensions, i.e., by $D$ facts conforming to the following template: $\tt point(Pno, Dim, Value)$. 
We also have a small set of centroids,  for which we
generate an initial assignment $\tt center(0, Cno, Dim, Val)$ 
by the predicate $\tt init(Cno, Dim, Val)$  defined using  any of the
simple techniques described in the literature.  Then,  Lloyd's clustering 
algorithm can be expressed 
concisely as shown in  Example~\ref{ex:lloyd}. 

%\footnote{ We use  $\tt encd$  to concatenate 
%two integers into a  double-length,  whereas and $\tt decd$ reconstruct from this  the original pair.
%A recent generalization of extrema, called chained min and max \cite{kddlog} 
%supports  this functionality as part of the aggregate.}.

\sinv\begin{example} [Clustering a l\'a Lloyd.]
 \label{ex:lloyd}
 
 ~\\[-0.5cm]
 
 \cldl
  \prule{\!r_0:center(J, Cno, Dim, Val)}{\! init(Cno, Dim, Val), J=0. \hspace{2cm}}
  \prule{r_1:cdist(J1, Pno, Cno, sum\langle SqD\rangle)}{\!point(Pno, Dim, Val) ,  center(J0, Cno, Dim, CVal).~~~~~~\sinv}
  \pbody{} {SqD{=} (Val-Cval)*(Val {-} Cval), J1=J0+1.}
  %\prule{r_2:isdist(J1, Pno, Cno,  DSm)}{cdist(J, Pno, Cno,  DSm), J1=J+1.\pin}
  \prule{\!r_2:mdist(J2, Pno, min \langle  [DSm, Cno] \rangle)}{ cdist(J1, Pno, Cno,  DSm),J2=J1+1.} 
  %encd(DSm, Cno, DCno),\sinv}
 %\pbody{} {J1=J+1.}
% \prule{r_4:mdist(J1, Pno, Dst)} {adst(J, Pno, Dist), J1=J+1.\pin}
\prule{\!r_3:center(J0, Cno,  Dim, avg \langle Val, Pno\rangle )}{mdist(J2, Pno, [\f--, Cno]), points( Pno, Dim, Val),\sinv}
 %[DmCno =  [\f--, Cno].} %\sinvdecd(DmCno, \f--, Cno), \sinv}
\pbody{}{J2 \leq 999, J0=J2+1.\pinv}
%\prule{r_6:center(J1, Cno,  Dim, Val)}{ncntr(J, Cno,  Dim, Val), J \leq 999, J1=J+1.}
\eldl
\end{example}
At each step $\tt J$  Example \ref{ex:lloyd} computes the
  quadratic distance of each point from each  centroid,  so that \rof{2}
 can select for each point the centroid closest to it. Then,  for each centroid, \rof{3}
recomputes its  position by averaging the coordinates of the points
that have this as their nearest centroid\footnote{Thus min assume a total ordering where  $\tt  [X1, Y1] \leq [X2, Y2]$ holds if  
$\tt X1< X2$ or if   $\tt X1=X2$ and $\tt Y1 \leq Y2$. This ordering is
easily extended to lists of arbitrary length.}:

This program is structurally similar to  the  Markov Chains  program  of Example \ref{ex:markov},
and the two programs  have similar properties in terms of declarative  and operational semantics.
In fact, the first argument in the heads of their rules  define an explicit stratification that implies SMS.
Moreover, since the cardinality of   $\tt Pno$ and $\tt Cno$ sets remain constant throughout the computation,
 we can use the pre-counting approach to construct the stable model by an eager fixpoint computation that
 is conducive to SSP scalability. However, with bulk-synchronous parallelish (BSP) the SMS of this program can be realized
 quite naturally and efficiently by simply  synchronizing the BSP steps with the 
first arguments in the head of the rule, i.e., with $\tt J0$, $\tt J1$,  $\tt J2$.
% of these two acyclic programs
%can be derived in a similar way.    In fact,  explicit GBL program
% whereby   from $\tt center$ atoms at level $\tt J$  we can only 
%derive $\tt cdist$ atoms at level $\tt J+1$ and from those we can only derive $\tt mindist$ atoms at level
%$\tt J+2$, and  from those  $\tt center$ atoms  at level $\tt J+3$.
%Thus if we view aggregates as standard macros then the $t\omega$ computation 
%provides a simple abstraction with the iterated fixpoint computation which
%can e.g.  be used in testing and debugging actual programs. The aggregate computation
%that takes place at this stratum is a recursive computation that realizes the SMS using
%the specific fixpoint computations that we have defined for the various aggregates.
% 
This guarantees that the declarative and operational  semantics of our programs are
 completely aligned and  conducive to  efficient implementations that are scalable via BSP.
In fact, Lloyd's algorithm is among the several data mining algorithms implemented
on Apache Spark using  Datalog with recursive aggregates. The 
performance and scalability of these algorithms have undergone extensive
experimentation, showing that they perform as well or better than 
the same algorithms expressed by lengthy\footnote{For instance,
the corresponding  procedural version requires  twenty-fold the
lines of codes used in Example \ref{ex:lloyd}.} procedural-language programs~\cite{bigdatalog,bigdatalog-mc,rasql,afrati2011map,rasqldemoSigmod20,datalogml,kddlog}.
%Furthermore, this declarative
%approach to Big Data analytics and allows the adaptations and customizations
%that are so frequently required in real-life applications.  
%Indeed  the coupling of formal SMS with operational fixpoint semantics 
%explored in this paper  shows that  Datalog with recursive aggregates represents a logic programming 
%paradigm of superior expressive power and performance.

\paragraph{\bf Expressing ML Applications}\label{subsec-express}
The problem of supporting efficient and scalable  ML applications concisely expressed
as Datalog queries with aggregates was studied in \cite{datalogml}. For instance, to support gradient descend, we
can  use a verticalized representation   $\tt vtrain({Id},{C},{V}, {Y})$ for the
 training set, where  $\tt Id$ denotes the id of a training instance, $\tt Y$ denotes its label, $\tt{C}$ and  $ \tt{V}$
 denote the dimension and the value along that dimension, respectively. Then a Batch Gradient Descent application
can be expressed by the program in Example \ref{ex:bgd} where
\begin{description}
	\inv\item [~~--~~$\tt model(J,C,P)$]   is the training model in verticalized form,  where
	$\tt J$  is the iteration counter $\tt C$  is a dimension in the model, and $\tt P$ is the parameter value for that dimension.
	\sinv\inv\item [~~--~~$\tt gradient(J,C,D)$]  contains the gradient result  $\tt G$  at  iteration  $\tt J$ for the ${\tt C}^{th}$ dimension.
	\sinv\inv\item [~~--~~$\tt predict(J, Id, YP)$] represents the intermediate prediction results, where $\tt Id$
	  denotes the id of the training instance, and $\tt YP$ its the predicted $y$ value at iteration  $\tt J$.
\end{description}

%Among these steps, the gradient computation and prediction with the current model can be easily represented with aggregates in recursion.
%Therefore, the iterative training process can be expressed with a recursive Datalog program Query~2.
Firstly, the model is initialized according to some predefined mechanisms in \rof{0} (Here we use  0.01). 
Then the function $f$ is used to make predictions on all training instances according to the model obtained in 
the previous iteration in  \rof{1}.
Next the gradient is computed by the function $g$ (derived according to the loss function $L$) using the predicted results in \rof{2}.
Finally, in \rof{3} the model is updated w.r.t the gradients (and optional regularization $\Omega$).
Here, $lr$ denotes the learning rate and $n$ is the number of training instances.
Then, the training process moves on to the next iteration.
\begin{example} [ Batch Gradient Descent (BGD)]
\label{ex:bgd}
~\\[-0.2cm]
\cldl
\prule{r_{0}:model(J, C, P)}{ vtrain(\_, C, \_, \_), J=0, P=0.01.}
\prule{r_{1}:predict(J,Id, sum \langle Y0 \rangle) }{vtrain(Id, C, V, \_), model(J, C, P),Y0 = f(V, P), J1 {=} J {+}1.}
\prule{r_{2}: gradient(J2,C,sum \langle G0 \rangle) }{vtrain(Id, C, V, Y), predict(J1, Id, YP),G0 {=} g(YP, Y, V), J2 {=} J1 {+}1.}
\prule{r_{3}:model(J, C, NP) }{ model(J, C, P), gradient(J, C, G), }
\pbody{}{NP = P - \emph{lr} * (G / \emph{n} + \Omega(P)), J {=} J3 {+}1.\sinv}
\eldl
\end{example}
As in previous two examples, the $\tt J$ argument in the group-by establishes a structure that guarantees SMS 
and superior scalability via BSP. The same is true for a wide spectrum of ML algorithms 
that can be expressed with different functions $f$, $g$, $\Omega$ in above program.
In particular, the Mini-batch Gradient Descent (MGD)  require only minor changes to above queries~\cite{datalogml}.

~\\[-0.4cm]

\section{Conclusion}

\sinv The usage of aggregates in recursive Datalog programs  entails a concise expression for very powerful algorithms
that combine a formal declarative SMS with a powerful and scalable fixpoint-based operational semantics.
This paper has  unified the semantics of programs with different aggregates by  turning them all
into equivalent programs with extrema,  and thus significantly simplified the verification of their SMS by
reducing it to the exclusion of provisional extrema from the derivation.  Using the \prem property
we  identified a large class of resilient or quasi-resilient  programs for which the current  implementation techniques
used in Datalog provide an efficient and SSP-scalable implementation  of their SMS.  For programs
that do not belong to this class, the paper proposed simple rewriting techniques, such as pre-counting
to realign their operational semantics with declarative one. Finally in  many programs
of practical interest, the synchronized execution produced by SSP also guarantees the correct derivation of their SMS.
These findings suggest that many big-data algorithms that are now developed using procedural languages
can instead be developed directly using  query languages, including SQL, 
since recursive Datalog queries with aggregates, can be translated into equivalent  SQL queries,
as discussed in the appendix. 

%{\small
%\bibliographystyle{plain}
\bibliographystyle{ACM-Reference-Format}
%\bibliography{mybib}
%}


\begin{thebibliography}{10}
{\small
\bibitem{afrati2011map}
Foto~N Afrati, Vinayak Borkar, Michael Carey, et~al.
\newblock Map-reduce extensions and recursive queries.
\newblock In {\em EDBT}, pages 1--8. ACM, 2011.

\bibitem{ldl++}
Faiz Arni, KayLiang Ong, Shalom Tsur, Haixun Wang, and Carlo Zaniolo.
\newblock The deductive database system {LDL++}.
\newblock {\em TPLP}, 3(1):61--94, 2003.

\bibitem{appl-iclp19}
Ariyam Das, Youfu Li, Jin Wang, Mingda Li, and Carlo Zaniolo.
\newblock Bigdata applications from graph analytics to machine learning by
  aggregates in recursion.
\newblock In {\em ICLP'19}, 2019.

\bibitem{ssp-iclp19}
Ariyam Das and Carlo Zaniolo.
\newblock A case for stale synchronous distributed model for declarative
  recursive computation.
\newblock In {\em 35th International Conference on Logic Programming}, ICLP'19, 2019.

\bibitem{gl:stable}
M.~Gelfond and V.~Lifschitz.
\newblock The stable model semantics for logic programming.
\newblock In {\em Proceedings of the Fifth Joint International Conference and
  Symposium}, pages 1070--1080. MIT Press, 1988.

\bibitem{DBLP:journals/tplp/GelfondZ14}
Michael Gelfond and Yuanlin Zhang.
\newblock Vicious circle principle and logic programs with aggregates.
\newblock {\em Theory Pract. Log. Program.}, 14(4-5):587--601, 2014.

\bibitem{DBLP:conf/pods/GrecoZG92}
Sergio Greco, Carlo Zaniolo, and Sumit Ganguly.
\newblock Greedy by choice.
\newblock In Moshe~Y. Vardi and Paris~C. Kanellakis, editors, {\em Proceedings
  of the Eleventh {ACM} {SIGACT-SIGMOD-SIGART} Symposium on Principles of
  Database Systems, June 2-4, 1992, San Diego, California, {USA}}, pages
  105--113. {ACM} Press, 1992.

\bibitem{rasql}
Jiaqi Gu, Yugo Watanabe, William Mazza, Alexander Shkapsky, Mohan Yang, Ling
  Ding, and Carlo Zaniolo.
\newblock Rasql: Greater power and performance for big data analytics with
  recursive-aggregate-sql on spark.
\newblock In {\em {ACM} {SIGMOD} Int. Conference on Management of Data, SIGMOD
  2019}, 2019.

\bibitem{kddlog}
Youfu Li, Jin Wang, Mingda Li, Ariyam~Das an~Jiaqi~Gu, and Carlo Zaniolo.
\newblock Kddlog: Performance and scalability in knowledge discovery by
  declarative queries with aggregates.
\newblock {\em International Conference on Data Enginering, {ICDE}}, 2021.

\bibitem{seo2013socialite}
Jiwon Seo, Stephen Guo, and Monica~S Lam.
\newblock {SociaLite}: {Datalog} extensions for efficient social network
  analysis.
\newblock In {\em ICDE}, pages 278--289. IEEE, 2013.

\bibitem{bigdatalog}
Alexander Shkapsky, Mohan Yang, Matteo Interlandi, Hsuan Chiu, Tyson Condie,
  and Carlo Zaniolo.
\newblock Big data analytics with {Datalog} queries on {Spark}.
\newblock In {\em SIGMOD}, pages 1135--1149. ACM, 2016.

\bibitem{datalogml}
Jin Wang, Jiacheng Wu, Mingda Li, Jiaqi Gu, Ariyam Das, and Carlo Zaniolo.
\newblock Formal semantics and high performance in declarative machine learning
  using datalog.
\newblock {\em VLDB J.}, 2021.

\bibitem{rasqldemoSigmod20}
Jin Wang, Guorui Xiao, Jiaqi Gu, Jiacheng Wu, and Carlo Zaniolo.
\newblock {RASQL:} {A} powerful language and its system for big data
  applications.
\newblock In {\em {SIGMOD} 2020 Int. Conference on Management of Data}, pages
  2673--2676. {ACM}, 2020.

\bibitem{wang2015asynchronous}
Jingjing Wang, Magdalena Balazinska, and Daniel Halperin.
\newblock Asynchronous and fault-tolerant recursive datalog evaluation in
  shared-nothing engines.
\newblock {\em PVLDB}, 8(12):1542--1553, 2015.

\bibitem{bigdatalog-mc}
Mohan Yang, Alexander Shkapsky, and Carlo Zaniolo.
\newblock Scaling up the performance of more powerful datalog systems on
  multicore machines.
\newblock {\em {VLDB} J.}, 26(2):229--248, 2017.

\bibitem{submitted}
Carlo Zaniolo, Ariyam Das, Youfu Li, Mingda Li, and Jin Wang.
\newblock Declarative and operational semantics for datalog programs with
  aggregates.
\newblock In {\em Submitted for Publication}, pages 1--15, 2021.

\bibitem{DBLP:journals/tplp/ZanioloYDSCI17}
Carlo Zaniolo, Mohan Yang, Matteo Interlandi, Ariyam Das, Alexander Shkapsky,
  and Tyson Condie.
\newblock Fixpoint semantics and optimization of recursive {Datalog} programs
  with aggregates.
\newblock {\em {TPLP}}, 17(5-6):1048--1065, 2017.

\bibitem{DBLP:conf/amw/ZanioloYIDSC18}
Carlo Zaniolo, Mohan Yang, Matteo Interlandi, Ariyam Das, Alexander Shkapsky,
  and Tyson Condie.
\newblock Declarative bigdata algorithms via aggregates and relational database
  dependencies.
\newblock In {\em Proceedings of the 12th Alberto Mendelzon International
  Workshop on Foundations of Data Management, Cali, Colombia, May 21-25,
  2018.}, 2018. }

\end{thebibliography}

\appendix
\section{Aggregates in Recursive  SQL Queries}

The  single source shortest path  Datalog query of Example~1 can be expressed in SQL.

\begin{example}[SSSP on base table:  $\tt edge(Src\!:int, Dst\!: int, Cost\!:double)$ by stratified SQL]
\label{ex:apsp}

~

\begin{verbatim}
WITH recursive sssp (Dst, Cost) AS
  (SELECT "a", 0)
   UNION
  (SELECT edge.Dst, sssp.Cost + edge.Cost
   FROM sssp, edge WHERE sssp.Dst = edge.Src)
SELECT Dst, MIN(Cost) AS minCost FROM sssp GROUP BY Dst
\end{verbatim}
\end{example}
\noindent
To express the query of Example~2 by a compact representation, 
we will assume that the aggregate columns, such as MIN(Cost) are implicitly
grouped by the other columns  in  $\tt SELECT $: i.e., by $\tt Dst$ for the example at hand.

\begin{example}[SSSP on base table:  $\tt edge(Src\!:int, Dst\!: int, Cost\!:double)$ by unstratified SQL]

~
\begin{verbatim}
WITH recursive sssp (Dst, min(Cost) AS minCost GROUP BY Dst) AS
  (SELECT "a", 0)
   UNION
  (SELECT edge.Dst, sssp.minCost + edge.Cost
   FROM sssp, edge WHERE sssp.Dst = edge.Src)
   SELECT Dst, minCost FROM sssp 
\end{verbatim}
\end{example}

Similar translations apply to the other  examples and aggregates discussed in the paper, with the provision 
that the keywords {\tt ALL} and {\tt DISTINCT} will
be added to specify their behavior in the presence of duplicates.
\end{document}


