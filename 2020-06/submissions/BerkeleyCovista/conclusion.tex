\section{Conclusion}
With these two simple innovations, we can move past the conflicts and controversy and on to utilizing privacy-sensitive mobile contact tracing to improve the efficacy of public health measures - thereby saving lives and unnecessary suffering - while respecting civil liberty.  The companies should provide the interoperable building blocks without themselves getting into the business of providing the Apps or holding the data.  They can maintain their privacy-first, decentralization posture, but should advocate for an interoperable COVID key Commons, rather than dictating policy on the app ecosystem.  Government actors can regain policy determination and influence the app ecosystem so as to best tailor offerings to their constituents. The tailoring could include questions around how best to provide a Commons of appropriate scale, and utilize physical measures to relate anonymous key reports to the places within their jurisdiction. Government actors already have extensive experience with civic infrastructure such as security cameras, traffic signals and health inspections. They can apply the similar principles towards recognizing the importance of public awareness, potential for creating stigma, and bringing benefit to non-participating members of the communities.  The technology can assist the contact tracing process, but should not be dictating it or replacing the human relationship of the patient and the health professionals performing interviews and care.  The trust that is built there is what makes opt-in approaches viable, and only with appropriate individual protections.