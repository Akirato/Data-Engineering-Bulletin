%!TEX root = ../main.tex

\section{Concluding Remarks}
\label{conclusion}

There exist many systems for monitoring and analyzing spatio-temporal data, such as a dashboard for visually tracking the outbreak of COVID-19~\cite{dong2020interactive} and a tweet stream sentiment analysis system for US election 2016~\cite{DBLP:conf/kdd/PaulLTYF17}. 
One lesson from the existing systems is  that they are usually designed on a case-by-case basis and built from scratch, which cannot fully leverage the recent techniques for data integration and automatic visualization.

On the one side, \sys-COVID-19 shares many common visualizations as the other popular websites for tracking COVID-19 cases.
On the other side, it differs from the others in 
(1) \sys-COVID-19 is based on a general end-to-end framework \sys, and leverages recent techniques for data preparation (\eg~{\sc VisClean}~\cite{visclean-icde}) and for visualization recommendation (\eg~{\sc DeepEye}~\cite{deepeyeicde});
%
(2) it supports linked visualization for the users to easily zoom in/out multiple visualizations by a single click; and
%
(3) it also obtains some private data that is not publicly available, so it can demonstrate some unique features.
%
%Hopefully we can survive the war of fighting COVID-19 with the minimum cost, and by the time of VLDB 2020, we will have much more to demonstrate.

%\add{Open Challenges: (1) Smart and Effective Data Preparation. (2) Data Sharing Platform. (3) Intelligent Data Analysis (4) Abstract general modules, the system supports reusability}