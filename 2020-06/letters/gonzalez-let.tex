% !TEX root=../deissue.tex 

\documentclass[11pt]{article} 

\usepackage{deauthor,times,graphicx}
%\usepackage{url}
\usepackage{hyperref}



\begin{document}

 
These are not normal times and this is not a normal edition of the Data Engineering Bulletin.  
When we started to prepare this edition, we were planning to focus on emerging trends in serverless computing and their impact on the data engineering community.  
In normal times this would have been a compelling edition full of recent advances and open challenges. 
However, everything changed with outbreak of the COVID-19 pandemic. 
The COVID-19 pandemic halted the world economies, transformed our daily lives, and has already claimed over four hundred thousands lives and it continues to surge around the world.
% While serverless (consumption based) computing will likely transform the cloud and shape future data technologies, COVID-19 presents a more pressing topic
 % remain a topic for a future data engineering bulletin, as COVID-19 is the fundamental 


In midst of this defining moment in human history, many in the computing world have asked an important question ---
\emph{How can we help?}
What skills and technologies can the computing world offer that could help in the fight against COVID-19?
In this edition of the Data Engineering Bulletin we attempt to provide an answer to this question by examining current efforts to build digital contact tracing systems. 
We believe that the data engineering community has a unique opportunity to help in this effort and hope that this edition will both provide context and inspire people to join in the effort.


Contact tracing, the process of identifying and notifying individuals that may have been exposed to the virus, is a critical step in 
curbing the outbreak of COVID-19 and reopening the world economies.  
% Nations that have safely begun to reopen have relied on contact tracing in conjunction with extensive testing. 
Unfortunately, traditional contact tracing is labor intensive and relies on interviews and manual field work to identify where infected individuals have been, who they may have exposed, and then finding, isolating, and testing all potential contacts.
Scaling this process to hundreds of thousands of infected individuals in a monumental task. 
While there are ample opportunities to deploy data management and even CRM technologies to help in the manual effort, there is substantial academic and industrial interest in automating the process by leveraging mobile devices.
It is this automated effort that we focus on in this edition.


Mobile contact tracing relies on location services and (or) short range radio hardware on mobile phones to help in the contact tracing efforts.  
There are a broad range of approaches and this edition of the Data Engineering Bulletin tries to capture work in several of the major approaches and provide some of the tradeoffs. 
One of the key tradeoffs discussed in this edition is between privacy and our ability to identify contacts, augment existing manual tracing efforts, and inform public health efforts.  
The need to preserve privacy is critical both to defend citizens from unjust government and civil actions but also to ensure the adoption needed for these technologies to be effective.


In selecting contributors for this edition, we attempted to construct a snapshot of major efforts around the world as well as some of the smaller research projects.
 % that emerged to support the bigger efforts. 
The first two papers in the edition outline the problem of mobile contact tracing and provide two views that form the basis of the major Apple and Google joint Exposure Notification protocol.
The first paper \emph{``{PACT}:   Privacy-Sensitive Protocols And Mechanisms for Mobile Contact Tracing''} represents a collaboration between US academic and industrial research groups to characterize the various approaches to mobile contract a propose a simple privacy preserving solution. 
The second paper \emph{``Decentralized Privacy-Preserving Proximity Tracing''} captures the contemporaneous European effort to develop a decentralized mobile contact tracing protocol and provides greater details around the design and analysis of the security protocols.
Both of these papers ultimately make a case for a cryptographic Bluetooth based approach that does not collect or sahre location information.  


The second pair of papers address two major mobile app development efforts that also explored both Location Services and Bluetooth based contact tracing.  
The first paper, \emph{``Contact Tracing: Holistic Solution Beyond Bluetooth''}, describes an effort based out of the MIT Media Labs using location based solutions.  
% This work addresses some of the key limitations of the Bluetooth based approaches.
The second paper, \emph{``Slowing the Spread of Infectious Diseases Using Crowdsourced Data''}, describes the Covid-Watch effort that initially explored a dual approach combining Bluetooth and GPS technologies. 
% Both of these apps are already available.
 % and were developed prior to the release of the Apple and Google Exposure Notification protocol.



We then have a series of papers that address particular issues with the above approaches. 
The first paper \emph{``CoVista: A Unified View on Privacy Sensitive Mobile Contact Tracing''} describes extensions of the Apple and Google Exposure Notification Protocol to capture exposure information associated with public spaces and shared data across public health authorities.  
The paper \emph{``Epione: Lightweight Contact Tracing with Strong Privacy''} describes a way to improve privacy in the Bluetooth protocol by leveraging private set intersection.  
The paper \emph{``BeeTrace: A Unified Platform for Secure Contact Tracing that Breaks Data Silos''} explores how to aggregate and query different data sources (e.g., location, bluetooth, calendar events) while preserving user privacy.


Finally, we conclude with two papers addressing the broader data science challenges and implications of COVID-19.
% with some work on tooling to help in making sense of COVID-19 data as well as an opinion piece on the longer term implications of the data centric thinking in COVID-19 and other disaster scenarios.
The paper \emph{``DeepEye: A Data Science Ssytem for Monitoring and Exploring COVID-19 Data''} describes a widely used system to support data science efforts around COVID-19.
The paper \emph{``The Road for Recovery: Aligning COVID-19 efforts and building a more resilient future''} examines the greater implications of the data technologies in COVID-19 and
longer term implications of the data centric thinking in other disaster scenarios.


We hope this edition will both serve as a snapshot of the efforts to develop data technologies to address COVID-19 and also inspire the data community to get involved. 
All of the publication have active development and research efforts and are looking for contributors.
What are you doing to help in the fight against COVID-19?





\end{document}


