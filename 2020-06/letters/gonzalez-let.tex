% !TEX root=../deissue.tex 

\documentclass[11pt]{article} 

\usepackage{deauthor,times,graphicx}
%\usepackage{url}
\usepackage{hyperref}



\begin{document}


These are not normal times and this is not a normal edition of the Data Engineering Bulletin.  
When we started to prepare this edition, we were planning to focus on emerging trends in serverless computing and their impact on the data engineering community.  
In normal times this would have been a compelling edition full of recent advances and open challenges. 
However, everything changed in the beginning of 2020, with the COVID-19 pandemic. 
The COVID-19 pandemic halted the world economies, transformed our daily lives, and has already claimed over four hundred thousands lives and it continues to surge around the world.
% While serverless (consumption based) computing will likely transform the cloud and shape future data technologies, COVID-19 presents a more pressing topic
 % remain a topic for a future data engineering bulletin, as COVID-19 is the fundamental 

In midst of this defining moment in human history, many in the computing world have asked an important question ---
\emph{How can we help?}.  
What skills and technologies can the computing world offer that could help in the fight against COVID-19?
In this edition of the Data Engineering Bulletin we attempt to provide an answer to this question by examining current efforts to build digital contact tracing systems. 


Contact tracing, the process of identifying and notifying individuals that may have been exposed to the virus, is a critical step in reopening the world and curbing the outbreak of COVID-19.  
% Nations that have safely begun to reopen have relied on contact tracing in conjunction with extensive testing. 
Unfortunately, traditional  contact tracing is labor intensive and relies on interviews and manual field work to identify where infected individuals have been, who they may have exposed, and then finding, isolating, and testing those individuals.
Scaling this process to hundreds of thousands of infected individuals in a monumental effort. 
While there are ample opportunities to deploy data management and even CRM technologies to help in the manual effort, there is substantial academic and industrial interest in automating the process by leveraging mobile devices.
It is this automated effort that we address in this edition.


Mobile contact tracing relies on location services and (or) short range radio hardware on mobile phones to help in the contact tracing efforts.  
There are a broad range of approaches and this edition of the Data Engineering Bulletin tries to capture each of these approaches and provide some of the tradeoffs. 
One of the key tradeoffs discussed in this edition is between privacy, accuracy, and our ability to augment existing manual tracing efforts and inform public health authorities.  
The need to preserve privacy is critical both to defend citizens from unjust government and civil actions but also to ensure the adoption needed for these solutions to work.


In selecting contributors for this edition, we attempted to construct a snapshot of major efforts around the world as well as some of the smaller research projects that emerged to support the bigger efforts. 
The first two papers in the edition outline the problem of mobile contact tracing and provide two views that form the basis of the major Apple and Google joint Exposure Notification protocol built on top of Bluetooth. 
The first paper \emph{``{PACT}:   Privacy-Sensitive Protocols And Mechanisms for Mobile Contact Tracing''} represents a collaboration between US academic and industrial research groups to characterize the various approaches to mobile contract a propose a simple privacy preserving solution. 
The second paper \emph{``Decentralized Privacy-Preserving Proximity Tracing''} captures the contemporaneous European effort to develop a decentralized mobile contact tracing protocol and provides greater details around the design and analysis of the security protocols.


The second pair of papers address two major US based mobile app development efforts.  The first paper, \emph{``Contact Tracing: Holistic Solution Beyond Bluetooth''}, describes an effort based out of the MIT Media Labs using location based solutions.  
This work addresses some of the key limitations of the Bluetooth based approaches and in many ways reflects the state of tracing today.  
The second paper, \emph{``Slowing the Spread of Infectious Diseases Using Crowdsourced Data''}, describes the Covid-Watch effort that initially explored a dual approach combining Bluetooth and GPS technologies. 





% We then have a set of three papers from multiple groups at UC Berkeley who have been exploring how to bridge the Bluetooth protocols





% Fortunately, contact tracing is fundamentally a sensing and data analysis problem and one in which the computing world is well positioned to help. 
% There


% Identify, who has interacted with an infected individual can be reduced to established location or radio proximity sensing problems for which we have technologies.
% Aggregating this information to inform individuals and public health authorities. 


% their economies have combined contact tracing with extensive testing and other social measures to return to 


% The successful management of diseases spread depends on contact tracing 


% had a profound impact on the world, transforming economies and with a death toll surpassing four hundred thousand 

% and as the death toll approaches half a million people, many have sought ways to get involved in th



% The data engineering, and more generally the computing world, sought ways to help in the fight against COVID-19 and many landed on the fundamentally data centric task of digital contact tracing.
% Rather than proceed with our plans to capture the trends in Serverless computing, we decided instead to focus on the rapidly evolving world of digital contact tracing.  




% The world population was forced to adapt and many sought ways to leverage their skills to rise to the challenge presented by the SARS-CoV-2 virus

% This is a special edition of the Data Engineering Bulletin, dedicated to capturing the state of research and system building efforts targeted at digital contact tracing.  



\end{document}


