\documentclass[11pt]{article} 

\usepackage{deauthor,times,graphicx}
%\usepackage{url}
\usepackage{hyperref}

\begin{document}


%\section*{Farewell}

It was way back in 1992 that Rakesh Agrawal, then the TCDE Chair, appointed me as Editor-in-Chief of the Data Engineering Bulletin. At the time, I saw it as a great opportunity. But it did not occur to me that it would become such an enormous part of my career. Now, 26 years later, it is time, perhaps past time, for me to pass this position on to younger hands, in this case to the capable hands of Haixun Wang. It should not come as a surprise that I am stepping down. Rather, the surprise should be ``why did I stay so long?'' This message is a combination of answer to that question and historical sketch of my time as EIC. These are not unrelated.

When I first became EIC, the Bulletin had already established a reputation as an industry and engineering focused publication, each issue of which was on a special topic.  Won Kim, my predecessor, had very capably established that publication model. Papers are solicited by each issue editor, with the editor selecting which authors to invite. The papers are a mix of work in progress, position statements, surveys, etc. But all focused on the special topic. I was determined not to screw this up. Indeed, I accepted the EIC appointment because I believed that the role that the Bulletin played is unique in our database community.  I stayed so long because I still believe that.  

Over the years, the Bulletin went through several major changes. As early as 1993, the Bulletin could be accessed online as well as via print subscription. This was a major transition. Mark Tuttle, then a colleague of mine in Digital (DEC) Cambridge Research Lab designed the latex style files that enabled this. Shortly thereafter, to economize on costs, the Bulletin became one of the earliest all electronic publications in our field. 

In 1995, hosting the Bulletin web site was provided by Microsoft- continuing until three years ago. Around 2010, the IEEE Computer Society became the primary host for the Bulletin. Around 2000, at the suggestion (prodding) of Toby Lehman,individual articles in addition to complete issues were served from the Bulletin web sites. Over this time, the style files and my procedures for generating the Bulletin evolved as well. Mark Tuttle again, and S. Sudarshan, who had been a Bulletin editor, provided help in evolving procedures used to generate the Bulletin and its individual articles. 

The Computer Society, and specifically staff members John Daniel, Carrie Clark Walsh, and Brookes Little, provided a TCDE email membership list used to distribute issue announcements, as well as helping in myriad other ways. The existence of dbworld (one of Raghu Ramakrishnan enduring contributions) enabled wider announcement distribution to our database community. The cooperation of Michael Ley with the prompt indexing of the Bulletin at dblp both ensured wider readership and provided an incentive for authors to contribute. Over the years, I was given great support by TCDE Chairs, starting with Rakesh Agrawal, then Betty Salzberg, Erich Neuhold, Paul Larson, Kyu-Young Whang, and Xiaofang Zhou. 

The most important part of being Bulletin EIC was the chance to work with truly distinguished members of the database community. It was enormously gratifying to have stars of our field (including eight Codd Award winners- so far) serving as editors. I take pride in appointing several of them as editors prior to their wider recognition.  It is the editors that deserve the credit for producing, over the years, a treasure trove of special issues on technologies that are central to our data engineering field. Superlative editors, and their success in recruiting outstanding authors, is the most important part of the Bulletin's success. Successfully convincing them to serve as editors is my greatest source of pride in the role I played as Bulletin EIC.  

Now I am happy to welcome Haixun to this wonderful opportunity.  Haixun's background includes outstanding successes in both research and industry.  He recently served ably as a Bulletin associate editor for issues on ``Text, Knowledge and Database'' and ``Graph Data Processing''.  His background and prior editorial experience will serve our data engineering community well and ensure the ongoing success of the Bulletin.  I wish him and the Bulletin all the best.

And so ``farewell''.  I will always treasure having served as Bulletin EIC for so many years. It was a rare privilege that few are given. Knowing that we were reaching you with articles that you found valuable is what has made the effort so rewarding to me personally. Thank you all for being loyal readers of the Bulletin. 

\end{document}

