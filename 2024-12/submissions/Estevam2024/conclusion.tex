\section{Conclusion}

In this work, we explored the limitations of current Retrieval-Augmented Generation (RAG) models and proposed that a System 2 perspective should be adopted to address the challenges faced by LLMs in complex, domain-specific enterprise applications. Despite the advancements in integrating external information for grounding LLM outputs, we highlighted the shortcomings of existing RAG approaches, which often lack rigorous reasoning and deliberative analytics characteristic of System 2 thinking. Our analysis is based on the literature review and results obtained in previous work on different aspects of LLMs limitations and current RAG approaches, and it reinforces the necessity of transitioning from monolithic LLM architectures to compound AI systems, which employ specialized agents to enhance retrieval, ensure factual correctness, and mitigate issues like hallucination.

Based on the results already obtained by the previously described approaches that incorporate initial steps towards the System 2 type of thinking, we outlined a vision for the future, emphasizing the design of compound systems that better align with System 2 principles, featuring coordinated, logic-driven workflows capable of holistic reasoning and cross-document synthesis. While our work provides a foundational perspective for these advancements, there are still open questions about optimizing retrieval strategies, seamlessly integrating multiple data types, and fine-tuning decision-making modules. Addressing these challenges will be crucial for deploying robust, trustworthy AI systems that meet the high standards of reliability and precision required in enterprise contexts.