\section{Formal Aspects of Computational Division of Labor}

\subsection{Computational Model and Complexity}
 
As with conventional computation in computer science, computation in
computational division of labor also takes some input, does some
computation along with some algorithm, then outputs some
result. However, now AI or human workers participate in computation,
which affects the way of computation itself. Therefore, (a)
establishing a formal computation model covering such factors and (b)
investigating computational complexity under such a model are of great
interest. As a preceding study, a computation model for crowd data
mining is proposed and computational complexity of several problems on
the model is investigated \cite{AM14}.

Here, let us consider a more general case in which we have a workflow
$W$ with tasks $t_1,t_2,\cdots,t_n$. In this case, task $t_i$ is
``computed'' in {\em one step\/} by an AI/human worker. Here, suppose
that the complexity of task $t_i$ is in class $C_i$ (PTIME, NP,
etc.). Then the computation of $W$ would be modeled as a Turing
machine with a set of oracle $C = \{C_1,C_2,\cdots,C_n\}$. But this is
still too simplified, off course, and we have a number of things to be
considered.
\begin{itemize}
\item How the computational power of workers affects workflow is an
  essential problem of computational division of labor. What makes
  workers, especially human workers, different from machines is that
  workers sometimes report incorrect answers but can respond to
  ``hard'' problems instantly with intuition. If we can estimate
  worker's computational power in any way (this is also an interesting
  problem), workflow using workers can be modeled as oracle machine,
  and we may be able to discuss computational power of such
  machine. For example, suppose that each worker has enough
  computational power to solve problems in PP, i.e., solve such
  problems in polynomial time with an error probability of less than
  $1/2$, and that workflow $W$ (without oracle) covers class P. Then
  the computational power of $W$ with oracle covers class
  ${\rm P}^{\rm PP}$. Note that
  ${\rm PH} \subseteq {\rm P}^{\rm PP}$~\cite{Tod91}, which would
  imply that $W$ can solve PH problems in polynomial time! Although
  this may be an extreme case, investigating how the computational
  power of workers affects workflow would be of great interest.
   
\item In conventional computation, algorithm is usually ``static'';
  different input values are given at each time but the algorithm
  itself is {\em fixed}. However, in the context of computational
  division of labor, workers and tasks affect each other, and both
  workers and tasks changes dynamically. In other words, workflow and
  tasks ``evolve'' over time along with changes of workers. Such
  dynamic aspects of computation is the essence of computational
  division of labor, and the concept of ``evolving algorithm'' would
  become more important.
\end{itemize}

\subsection{Optimization}

In general, optimization problems are hard to solve exactly, and an
exact optimum solution can be obtained efficiently under only limited
conditions, e.g., (a) the objective function is linear and the
variables are continuous (linear programming), and (b) the objective
function is submodular and minimized \cite{Orl09}.

In addition to dealing with such hardness, dynamic aspects of
computational division of labor needs to be taken into account.
Worker set is not constant because workers' join and quit occur
constantly. The abilities of workers change as well. Such changes
affects tasks, and the change of tasks affect workers. In such a
dynamic situation, it would be inefficient to calculate the optimal
solution from scratch each time a change occurs. Therefore,
optimization mechanism that adapts to changes of workers and tasks,
e.g., incremental algorithm, is becoming more important than before.
% The design of algorithm that is ``robust'' to changes of
% workers may also be worth considering.  
As a preceding study, an incremental algorithm for finding an optimum
worker assignment when a worker set changes is proposed
\cite{RLT+15}.

Finally, optimization itself can be regarded as ``task'' in the
context of computational division of labor. Therefore, ``optimization
algorithm'' in which workers contribute their computational power to
some part of optimization would be an algorithm design with great
potential.


% \cite{RTA+19} proposes an approximation algorithm that
% constructs an optimal group when a task is performed by a group of
% multiple people.


