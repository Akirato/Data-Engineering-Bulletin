%!TEX root = ../main.tex

\begin{abstract}
Human-in-the-loop techniques are playing more and more significant roles in the machine learning pipeline, which consists of data preprocessing, data labeling,  model training and inference. Humans can not only provide training data for machine learning applications, but also directly accomplish some tasks that are hard for the computer in the pipeline, with the help of machine-based approaches. 
%Humans are playing more and more significant roles in the machine learning pipeline, which consists of data preprocessing, data labeling,  model training and inference. Humans  not only can provide training data for machine learning applications, but also directly accomplish some tasks that are hard for the computer in the pipeline, with the help of machine-based approaches.
In this paper, we first summarize the human-in-the-loop techniques in machine learning, including: (1) Data Extraction: Non-structured data always needs to be transformed to structured data for  feature engineering, where humans can provide training data or generate rules for extraction.    (2) Data Integration: In order to enrich data or features, data integration is proposed to join other tables. Humans can help to address some machine-hard join operations.  (3) Data Cleaning: In real world, data is always dirty. We can leverage humans' intelligence to clean the data and further induce rules to clean more. (4) Data Annotation and Iterative labeling. Machine learning always requires a large volume of high-quality training data, and humans can provide high quality data for training. When the budget is limited, iterative labeling is proposed to label the informative examples.  (5) Model training and inference. For different  applications(e.g. classification, clustering), given human labels, we have different ML techniques  to train and infer the model. Then we summarize several commonly used techniques in human-in-the-loop machine learning applied in the above modules, including quality improvement, cost reduction, latency reduction, active learning and weak supervision. Finally, we provide some open challenges and opportunities.

%In this survey, we summarize  overall techniques in human-in-the-loop machine learning, including: (1) Quality Improvement: Humans may return noisy results, so effective techniques need to be applied to achieve high quality; (2) Cost Reduction: Since the humans are not free, we should reduce the monetary cost; (3) Latency Reduction: Compared with the computers, humans can be slow, so latency reduction techniques are required. (4) Active Learning: In most cases, budget for labeling the training data is limited, so active learning selects the most interesting examples to label iteratively. 
%5)Weak Supervision; Enough perfect labels are expensive to acquire, so weak supervision is proposed to generate high quality results from weak labels.
%Then we survey how to apply the above techniques to different modules in the machine learning pipeline,  introduce some future works and finally conclude.
\end{abstract}
