\documentclass[11pt]{article} 

\usepackage{deauthor,times,graphicx}
%\usepackage{url}
\usepackage{hyperref}


\begin{document}
The fast development of AI technology has changed our dail life significantly,  from personal assistant such as Siri to self-driving cars, online recommendations
sites such as Netflix and Amazon and many other real-world AI applications have emerged and brought great benefits to humanity. All these AI-powered applications have shown great intelligence if their models are well trained.  For example, Alpha go, a well-trained go player defeated top human players in the world.  However, the current AI technology relies strongly on the quality of the training datasets and can be restricted by the cognitive nature of a task. Tasks such as recognizing objects from blurred images and understanding sentiment
from complicated and poorly-structured sentences still need human assistance.  Therefore, in this issue, we study an interesting topic, human-powered AI.  Compared to current AI technology which is more advanced in solving closed domain problems, human intelligence is more advanced in addressing open domain problems, such as arts and design. How to seamlessly incorporate human intelligence into the whole process of AI is the theme of human-powered AI.

In this issue, we present works on human-powered AI from different aspects, 

The first paper discusses a challenge and essential problem in Human-powered AI, how to divide computation between human and AI effectively to achieve a specific target. Based on the discussion, the authors have introduced a set of dimensions and terms to classify existing solutions on this topic. 

The second paper addresses the challenges on how to untilize human intelligence (crowdsourcing) on federate learning,  which is one of the popular solutions to overcome the problem of data isolation and data privacy in a distributed learning environment.


The third  paper addresses the challenges on providing an end to end solution in human-in-the-loop of machine learning, from data extract, data integration, data cleaning, data labelling to machine learning and inference.


The fourth  paper  presents human-powered AI from the angle of AI technology, specifically, how to use AI technology to model, discover and explore human behaviour for human intelligence data management and mining. 


The last paper presents a real application of human-powered AI  techniques for online misinformation detection, challenges on various aspects in implementing such a system are outlined.


We  would like to thank all the authors for their insightful contributions. 
\end{document}


