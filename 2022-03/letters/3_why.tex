%!TEX root = paper.tex
\section{Reasoning about the Why: The Causes of Errors}
\label{sec:why}

It is time to ask (and answer) the \emph{why} question!
Existing work in the area of data quality, error detection and data cleaning almost exclusively focuses on alleviating the symptoms, rather than removing the cause of the error.
None of the methods asks \emph{why} a particular value is missing, why duplicates exist in the data, why violations occur. Answers as to ``why'' include: faulty (human) data entry, such as missing entries, misplaced values, typos, and vandalism; faulty reading from sensors; missed or not-propagated updates; faulty computations; and misconfigured data pipelines. 

While researchers and practitioners (and medical doctors) will acknowledge the truism that problems are best addressed at their source rather than treating their symptoms, the research community has not adequately addressed this opportunity possibly for several reasons:
\begin{itemize}
    \item In some scenarios, once errors are detected, it is too late -- fixing their cause is futile because the data was intended for a one-time use.
    \item Often, the creation of data is out of the control of the data engineers or data consumers: the data stems from an external source and the data creation can be influenced only through human intervention, such as communicating with the data owners or creators.
    \item Modifying or improving the data creation process is difficult or impossible, for instance due to technical or human limitations: sensors have an inherent error margin; humans are not infallible, etc.
    \item Data processing pipelines have become so complex, that treating the symptoms is the easier short-term goal with quick rewards.
\end{itemize}

Knowledge of the cause of an error and not only its symptom can improve cleaning methods and can help avoid such errors in the first place. To seize this opportunity, multiple challenges must be overcome:
\begin{itemize}
    \item \emph{Modelling} the processes and data generators (including humans) in the system, instead of modelling only the data and errors.

    \item \emph{Detection} of data errors without context and detection of erroneous data processes.
    
    \item Extending the notion of \emph{provenance} to include (possibly faulty) processes, computation (internal provenance), and data generation steps (external provenance). More on this in Section\ref{sec:provenance}.
    
    \item Designing of algorithms and systems to efficiently and effectively trace such extended provenance.
    
    \item Designing \emph{repair} operations for such errors and processes, which need to reach beyond the mere deletion or replacement of data instances that is the currently common approach.

%    \item Algorithms that can point to these new things
\end{itemize}

These challenges can be summarized as creating a more holistic view of data creation and consumption than is currently practiced. Especially the extended notion of provenance deserves a closer look in the next section.

%existing work: Modelling the process or Cleaning actions at the source

%Definition of cause of error: SOME TEXT. Understanding the cause of an error asks ``Why do we observe this error?''

%Raw-data generative process (link to ICDT paper by Ihab, and others)


%``So I'll remove the cause. But not. The symptom!'' Dr.\ Frank N.\ Furter, Rocky Horror Picture Show