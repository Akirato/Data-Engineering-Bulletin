\section{Conclusion}
\label{sec:conc}

To conclude, we suggest opening a new chapter of data quality and data cleaning that understands the entire data processing pipeline, in particular tracing it to the very beginning -- the genesis of the raw data. We have pointed out the challenges, with a focus on a new view of data provenance. 

Having discussed the \emph{how} (symptom), the \emph{why} (cause), and the \emph{where} (via provenance), other questions about errors remain. We have only glossed over the question \emph{what} is erroneous: an individual value, a row, a column, a table, or a process? Our general discussion allows these questions for data model beyond the relational, including tree or graph data, or even images, sound and video. When regarding data as it is created over time, we can ask \emph{when} the data error was introduced, and use data versions to understand the nature of the error~\cite{bleifuss2018exploringchange}. The final question of \emph{who} to blame, we leave to the management sciences.

%Definition of location of error (optional): “Where…?”
%{Reasoning about Where: The }
%WHERE (where in the process/pipeline)

%Ingestion, transformations, predictions, etc 

%Provenance is more relevant (tracking): DBRx, Data Xray, find errors at derivatives 

%relating prediction errors to faulty training data
