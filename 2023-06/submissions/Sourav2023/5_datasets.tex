
\section{Datasets}
% dataset will have datasets
% references some papers
% nature - synthetic or real
%Task - graph or node classification
% explanation counterfactual or factual
\label{sec::datasets}


A set of synthetic as well as real-world datasets have been used for evaluating the proposed explainers in several tasks such as node classification and graph classification. Table \ref{tab::dataset} lists down the set of datasets and the corresponding explanation types and tasks used in the literature.
\subsection{Synthetic datasets}
Annotating ground truth explanations in graph data is laborious and requires domain expertise. To overcome this challenge, several explainers have been evaluated using synthetic datasets that are created using certain motifs as ground truth values. We highlight \textit{six} popular synthetic datasets:% node classification explanation and 2 are used for graph classification explanation

\noindent\textbf{BA-Shapes} \cite{ying2019gnnexplainer}: This graph is formed by randomly connecting a base graph to a set of motifs. The base graph is a Barabasi-Albert (BA) graph with $300$ nodes. It includes $80$ house-structured motifs with five nodes each, formed by a top, a middle, and a bottom node type. 
% Nodes are divided into $4$ classes with There are four classes of nodes with  three classes for nodes in the motif (top, middle and bottom of the house) and the last one is for nodes in the base graph.
    
\noindent\textbf{BA-Community} \cite{ying2019gnnexplainer}: The BA-community graph is a combination of two BA-Shapes graphs. The features of each node are assigned based on two Gaussian distributions. Also, nodes are assigned a class out of eight classes based on the community they belong to.

\noindent\textbf{Tree Cycle} \cite{ying2019gnnexplainer}: This consists of a 8-level balanced binary tree as a base graph. To this base graph, 80 cycle motifs with six nodes each are randomly connected. It just has two classes; one for the nodes in the base graph and another for nodes in the defined motif.
    
\noindent\textbf{Tree Grids} \cite{ying2019gnnexplainer}: This graph uses same base graph but a different motif set compared to the tree cycle graph. It uses 3 by 3 grid motifs instead of the cycle motifs.
    
\noindent\textbf{BA-2Motifs} \cite{pgexplainer}: This is used for graph classification and has two classes. The base graph is BA graph for both the classes. However, one class has a house-structure motif and another has a 5-node cycle motif.
    
\noindent\textbf{Spurious Motifs} \cite{D_invariant_rationale}: With 18000 graphs in the dataset, each graph is a combination of one base $S$ (Tree, Ladder or Wheel) and one motif $C$ (Cycle, House, Crane). Ground-truth $Y$ is determined by the motif. A spurious relation between $S$ and $Y$ is manually induced. This spurious correlation can be varied based on a parameter that ranges from $0$ to $1$.

\begin{table}[tb]
\vspace{-2mm}
  \centering
  \scriptsize
  \caption{It shows the datasets for different categories, explanation types and tasks. }
    \begin{tabular}{ccccc}
    \toprule
          \textbf{Dataset} & \textbf{References} & \textbf{Nature} & \textbf{Explanation Type} & \textbf{Task}  \\  \midrule
        BA-Shapes & \cite{ying2019gnnexplainer,pgm-ex,RELex,pgexplainer,Gen-causal,cfgnnex,robust-counter}& Synthetic  & Compared to Motif  & Node classification  \\  
        BA-Community & \cite{RL-enhanced,ying2019gnnexplainer, pgexplainer, robust-counter, clear-counter} & Synthetic  & Compared to Motif  & Node classification \\  
       Tree Cycle & \cite{Gen-causal, pgexplainer, RELex, robust-counter,cfgnnex} & Synthetic & Compared to Motif  & Node classification  \\  
        Tree Grids & \cite{ying2019gnnexplainer,pgexplainer, RELex, Gen-causal, robust-counter, cfgnnex} & Synthetic & Compared to Motif  & Node classification \\  
        BA-2Motif & \cite{pgexplainer, subgraphX,GSAT,robust-counter} & Synthetic & Compared to Motif  & Graph classification \\  
        Spurious Motifs & \cite{GSAT,D_invariant_rationale}& Synthetic & Compared to Motif  & Graph classification \\  
        Mutagenicity & \cite{ying2019gnnexplainer, pgexplainer, Gen-causal, protgnn, subgraphX,robust-counter,xgnn} & Real-World & Compared to Chemical property  & Graph classification \\  
        NCI1 & \cite{Gen-causal, robust-counter}& Real-World & Compared to Chemical property  & Graph classification \\  
        BBBP & \cite{protgnn,cf^2-counter, moleculenet} & Real-World & Compared to Chemical property  & Graph classification \\  
        Tox21 & \cite{meg-counter, moleculenet} & Real-World & Compared to Chemical property  & Graph classification \\  
        MNIST-75sp & \cite{pgm-ex,GSAT,mnist_75}& Real-World & Visual  & Graph classification \\  
        Sentiment Graphs & \cite{subgraphX,GSAT,protgnn,sst-datasets} & Real-World & Visual  & Graph classification \\  \bottomrule

        
    \end{tabular}%
    \vspace{-2mm}  \label{tab::dataset}%
\end{table}%

\subsection{Real-world datasets}
Due to the known chemical properties of the molecules, molecular graph datasets become a good choice for evaluating the generated explanation structure. 
% These chemical properties act as ground truths. 
We highlight some widely used molecular datasets for evaluating explainers in the \textit{graph classification task}. 

\noindent\textbf{Mutag} \cite{mutag}: This consists of $4337$ molecules (graphs) with two classes based on the mutagenic effect. Using domain knowledge, specific chemical groups are assigned as ground truth explanations.    
    
\noindent\textbf{NCI1} \cite{NCI1_data}: It is a graph classification dataset with 4110 instances. Each graph is a chemical compound where a node represents an atom and an edge represents a bond between atoms. Each molecule is screened for activity against non-small cell lung cancer or ovarian cancer cell lines.

\noindent\textbf{BBBP} \cite{moleculenet}: Similar to Mutag, Blood-brain barrier penetration (BBBP) is also a molecule classification dataset with two classes with 2039 compounds. Classification is based on their permeability properties.

\noindent\textbf{Tox21} \cite{moleculenet}: This dataset consists of 7831 molecules with 12 different categories of chemical compounds. The categorization is based on the chemical structures and properties of those compounds. 
    
Visual explanation can be an important component of comparing explainers. Hence, researchers also use datasets that do not have ground truth explanations but can be visually evaluated through generated examples. Below are some of the datasets used for visual analysis:

\noindent\textbf{MNIST-75sp}~\cite{mnist_75}: An MNIST image is converted to a super-pixel graph with at most 75 nodes, where each node denotes a ``super pixel''. Pixel intensity and coordinates of their centers of masses are used as the node attributes. Edges are formed based on the spatial distance between the super-pixel centers. Each graph is assigned one of the 10 MNIST classes, i.e., numerical digits. 

\noindent\textbf{Sentiment Graphs}~\cite{sst-datasets}: Graph SST2, Graph SST5, and Graph Twitter are based on text sentiment analysis data of SST2, SST5, and Twitter datasets. A graph is constructed by considering tokens as nodes, relations as edges, and sentence sentiment as its label. The BERT architecture is used to obtain 768-dimensional word embeddings for the dataset. The generated explanation graph can be evaluated for its textual meaning. 




