\section{Introduction}
Graph Neural Networks (GNNs)~\cite{dlgsurvey_tkde20,wu2020comprehensive} have gained prominence in recent years due to their remarkable theoretical and empirical potential for learning powerful representations of graph-structured data. Many real-world graphs or networks exhibit homophily, where nodes predominantly connect with others belonging to the same class~\cite{mcpherson2001birds,zhu2020beyond}. While early GNNs demonstrated promise on graphs with this property, they faced challenges on graphs exhibiting heterophily, where the majority of nodes connect to those of different classes~\cite{MixHop,Pei2020Geom-GCN,zhu2020beyond}. This prompted investigations into GNN design choices conducive to learning on graphs with heterophily and sparked interest in developing new GNN models tailored for this property~\cite{zhu2020beyond,yan2022two,chien2021adaptive,zhang2021beyond,luan2022revisiting,yan2022two,zheng2022graph,song2023ordered}.

Beyond improving the effectiveness of GNNs on heterophilous datasets, recent research has shown that the challenges posed by graphs with heterophily are closely connected to other GNN challenges, including oversmoothing~\cite{li2018deeper,chen2020simple}, algorithmic bias~\cite{li2021dyadic,wang2022improving}, and sensitivity to adversarial attacks~\cite{zugner2018adversarial,dai2018adversarial,xu2019topology,wu2019adversarial,li2020adversarial,ma2020towards}. Designs addressing heterophily often improve the ability of GNNs to handle these challenges as well, leading to significant advances in overall GNN capabilities~\cite{chien2021adaptive,yan2022two,loveland2022graph,zhu2022heterophily,bodnar2022neural}.

Another line of work, however, has revisited whether early GNN designs were as ill-suited for learning from heterophilous graphs as initially thought. On some heterophilous networks, basic Graph Convolutional Networks (GCNs)~\cite{kipf2016semi} have proven competitive with, or even outperformed, models specifically designed for heterophily~\cite{ma2021homophily,luan2022revisiting}. This has led to the proposition that the challenges posed by some graph datasets are not best captured by the traditional homophily ratio. Consequently, other works have focused on analyzing the properties of heterophilous graphs that challenge early GNNs and designing generalized homophily metrics that offer more insight into the difficulties a graph dataset may present~\cite{ma2021homophily,luan2022revisiting}. Thus, a valid debate exists over whether ``heterophily'' is a real problem that GNNs face.

Our work revisits this debate with additional analysis. First, we provide a concise review of recent designs for graphs with heterophily, their connections to other GNN research objectives, as well as generalized heterophily metrics. We then examine the heterophilous conditions under which conventional GNNs have been shown to be competitive with those tailored for heterophily. Our analysis reveals that while conventional GNNs can sometimes succeed in learning on heterophilous graphs without specialized designs,
such condition is often broken 
when the underlying data has low-degree nodes and complex heterophilous patterns (``compatibility matrices'').
Thus, we believe that continuing to develop GNNs that can learn across the spectrum of low-to-high homophily remains an important theoretical and empirical problem. We summarize our contributions as follows:
\begin{itemize*}
\item We review and summarize recent designs proposed for graphs with heterophily (\S\ref{sec:progress-designs}), providing a unifying intuition. Moreover, we discuss their use in subsequent GNN works and their implications for other objectives of GNN research (\S\ref{sec:progress-connections}), such as fairness, robustness, and reducing oversmoothing.
\item We conduct an empirical analysis on the conditions under which conventional GNNs can succeed on heterophilous datasets (\S\ref{sec:complexity}). Our analysis demonstrates the unique challenges in achieving high separability of Neighborhood Label Distribution (NLD) when low-degree nodes (\S\ref{sec:complexity-factor-degree}) or complex heterophilous patterns (\S\ref{sec:complexity-factor-compatibility}) are present. These challenges hinder the effectiveness of conventional GNNs and are best addressed by GNN designs specifically tailored for heterophily (\S\ref{sec:complexity-experiments}).
\item We discuss future research directions aimed at enhancing our understanding of how heterophily impacts GNNs across a broader range of contexts (\S\ref{sec:conclusion}). These include moving beyond node classification and global homophily, introducing more diverse graph datasets and applications, and exploring the connections between heterophily and heterogeneity. 
\end{itemize*}



