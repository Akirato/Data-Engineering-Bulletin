\section{Conclusion \& Future Directions}
\label{sec:conclusion}

In this work, we revisited the debate of whether heterophily is a challenge for GNNs. 
We first reviewed representative architectural designs that have been proposed in the literature for improving the performance of GNNs on heterophilous data, and then discussed the connections with other objectives of GNN research, such as robustness, fairness, and reducing oversmoothing. 
To address the debate and reconcile seemingly contradictory statements in the literature, we conducted an extensive empirical analysis that aimed to provide a better understanding of when heterophily is challenging and when it does not pose significant additional challenges compared to handling graphs with homophily. 
We also considered recently proposed measures for quantifying the complexity of heterophily and evaluated their effectiveness across synthetic datasets based on different generation processes. 
Our analysis revealed two key factors that increase the complexity of heterophily: (F1) the presence of low-degree nodes, and (F2) the complexity of the class compatibility matrices of the underlying graphs. 
These factors present unique challenges for GNNs under heterophilous settings, and necessitate architectural designs that can improve the performance of GNNs. 
We hope that our review and empirical analysis will inspire future research 
on better understanding the unique challenges of heterophily in GNNs and  developing more effective GNN models that can handle well both graphs with homophily and heterophily (of variable complexity).



\paragraph{Future Directions.} There are many promising research directions towards understanding the unique challenges that heterophily poses to GNN models.
Next we discuss some representative open problems: 
\setlist{leftmargin=*}
\begin{itemize}
    \item \textbf{Beyond node classification and global homophily.} 
    Most existing works on GNNs and heterophily (including the ones we review in this work) focus on node classification, where heterophily can be defined and measured with respect to the agreement of class labels for connected nodes. 
    However, many important applications on graphs, such as recommendation systems, query matching, and the prediction of molecular properties, are based on other learning tasks such as link prediction and graph classification. 
    It is thus important to understand the effects of heterophily on these tasks and inform the design of tailored GNN models that can handle heterophily.
    While few works have discussed heterophily in the settings of link prediction~\cite{zhou2022link,zhu2023simplifying} and graph classification~\cite{ye2022incorporating}, their definition of heterophily is still based on node class labels, which are often not available for these tasks.
    Measuring homophily in the absence of node is an interesting problem for these graph learning tasks. 
    Moreover, going beyond a global perspective and exploring the effect of different mixing patterns across different neighborhoods is an important research direction that has started to gain reaction~\cite{loveland2022graph}. 
    
    \item \textbf{More datasets \& applications.} Despite recent efforts in collecting and introducing new datasets that address the drawbacks of existing heterophilous ones~\cite{lim2021large,platonov2023critical}, we believe that the call for more heterophilous graph datasets and applications is still important and timely. 
    Many existing works on GNN and heterophily rely on the six heterophilous graph datasets which were first adopted by \citet{Pei2020Geom-GCN}. 
    While these datasets were useful during the early stages of research on GNNs and heterophily, multiple works~\cite{zhu2020beyond,lim2021large,platonov2023critical} have pointed out the drawbacks of these commonly adopted benchmark datasets, namely their small sizes, artificial class labels, imbalanced class sizes, unusual network structure, and even leakage of test nodes in the training set. 
    In light of these, \citet{lim2021large} and \citet{platonov2023critical} proposed a set of mid- to large-scale social, citation and web networks with more diverse node features and realistic class labels, 
    but these datasets have yet to gain widespread adoption,
    and the relationship between the (heterophilous) links and the class labels is often ambiguous   
    (e.g., predicting product ratings on Amazon based on edges connecting frequently bought items). 
    Thus, we believe that there is still a need for datasets that have naturally-occurring heterophilous connections that align better with defined node class labels. 
    In terms of application domains, it would be useful to go beyond social, citation, and webpage networks and introduce benchmarks that capture 
    molecular or protein structures, which could also aid the investigation of more graph learning tasks that we discuss above. 
    
    \item \textbf{Connections between heterophily \& heterogeneity.} 
    Although we highlighted in \S\ref{sec:preliminaries} that heterophily and heterogeneity are two distinct concepts that should not be confused, heterogeneity may introduce unique forms and challenges of heterophily that are worth investigating: connected nodes of different \emph{types} could imply dissimilarity in their embeddings, resembling the concept of heterophily, while the level of homophily may also vary across different local mixing patterns. As a result, GNN models operating on heterogeneous graphs have already adopted designs similar to those tailored for heterophily, such as the separation of ego- and neighbor-embeddings and the use of type-specific kernels in message passing~\cite{schlichtkrull2018modeling}, in order to address the challenges of heterogeneity. Moreover, recently, \citet{guo2023homophily} also discussed how enhancing the homophily level in the meta-paths of heterogeneous graphs can improve GNN performance. Therefore, we believe that further research on the connections between heterophily and heterogeneity can help better understand the connections between the methodologies and findings of these two settings, which in turn may lead to the development of more effective GNNs for both scenarios.
\end{itemize}




