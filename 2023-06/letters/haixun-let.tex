\documentclass[11pt]{article} 

\usepackage{deauthor,times,graphicx}
%\usepackage{url}
\usepackage{hyperref}

\begin{document}
Graph is an essential representation for many real-world data exhibiting intricate relationships. We first surveyed work in this field in one of our 2017 issues, where the focus was on developing efficient algorithms for very large graphs. Then, in one of our 2022 issues, we highlighted applications on graph data, including knowledge graphs, causality, and reasoning over knowledge graphs.

This issue is devoted to Graph Neural Networks (GNNs), a topic that has garnered a great deal of attention and is the driving force behind many important applications such as personalization and recommendations systems. GNNs are effective at relational reasoning, which expands the frontiers of traditional machine learning methodologies. Significant effort was also devoted to the scalability challenge, which is crucial in today's data-driven landscape characterized by vast and complex data structures. This issue, curated by Karthik Subbian, contains eight papers from leading researchers in this field, which include comprehensive surveys as well as deep dives into key technical challenges.

Wang-Chiew Tan has penned an opinion piece on the topic of data management and LLMs. It is a call to action for the database community in light of the rise of LLMs and generative AI. While LLMs provide an adept natural language interface to unstructured data,  the absence of structure or a definitive data schema frequently results in less reliable outcomes. DBMSs, on the other hand, cannot handle unstructured data,  a category that greatly eclipses structured data in terms of sheer volume. Tan suggests that a more robust, next-generation data management paradigm might emerge from the fusion of LLMs' strengths with those of DBMSs.

We would also like to congratulate Professor Kyu-Young Whang on receiving the prestigious TCDE Service   Award, as well as celebrate his retirement and enormous contributions to the field of data engineering. We are privileged to publish a letter from Whang in this issue, in which he reminisced about the many research projects he had spearheaded, the changes he had witnessed in the field over the decades, and the community he had mentored.
\end{document}

