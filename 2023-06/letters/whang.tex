\documentclass[11pt]{article} 

\usepackage{deauthor,times,graphicx}
%\usepackage{url}
\usepackage{hyperref}

\begin{document}

\section*{A Life-long Saga with Data Engineering}

It is my great honor to receive this prestigious TCDE Service Award in recognition of my life-long
contribution to the data engineering community over 3 to 4 decades. I am retired now but, looking
back, I have really been privileged to serve our community for the advancement of the data
engineering discipline through various opportunities.
\subsection*{ICDE, TCDE, and VLDB Endowment}
I had opportunities to serve VLDB and ICDE in various capacities including the general chair
(VLDB2006), honorary general chair (ICDE 2015), a PC co-chair (VLDB2000, ICDE 2006), ICDE
steering committee member (2007-2015), VLDB Endowment trustee twice (1998-2003, 2010-2015),
and TCDE executive including chair and advisor (2011-2022).
An early contributor of ICDE from the 2$^{nd}$ conference in 1986 as a PC member, I also served as a
program co-chair or vice chair many times in initial years from 1989 helping to settle the newly
established conference. I am glad that ICDE has continuously been a top conference in the data
engineering field.
During my tenure as the TCDE Chair, we significantly broadened TCDE activities raising the level of
vitality and prestige of TCDE. We initiated TCDE Archives restoring many years of institutional
memory, newly instituted the IEEE TCDE Awards, and initiated membership promotion tripling the
membership.
In the VLDB Endowment, we have done a lot to promote global database research, but I would like to
note on two efforts in particular.  The first one is 
"broadening"; the scope of database research, in
which I participated as an endowment trustee and PC co-chair (VLDB2000). This effort started from
VLDB2000, resulting in the creation of the “Infrastructure for Information System (IIS)” track in 2002.
The IIS track had lasted until 2013 when it was merged back with the Core DB track to a single one
as it fulfilled its original mission. This broadening initiative significantly enlarged the scope of
database research as it is today. The second one is that the endowment eagerly supported the Asia-
Pacific region, then lagging in database research, to help bring it up to a level equivalent to those of
the Americas and European regions—by various programs including the “VLDB database school.”
Nowadays, the Asia-Pacific region stands very strong and competitive with others.
\subsection*{The VLDB Journal}
I also had the honor to serve the VLDB Journal for 19 years continuously as a founding editorial board
member, an Editor-in-Chief (EIC), and the coordinating EIC (1990-2009). We emphasized on the
strong editorial board, identifying timely impactful topics for thematic special issues, guaranteeing
timely reviews, and increasing availability. During my tenure as the coordinating EIC, the VLDB
Journal ranked the top in the Information Systems field with the highest impact factor (7.067 in 2008)
according to Thomson’s Science Citation Index. I am glad that nowadays the VLDB Journal stands
itself as a top journal in the data engineering field.
\section*{Awards Committees}
It was an honor to serve many prestigious awards committees including the SIGMOD Jim Gray
dissertation award committee (2007-2012), the VLDB 10-year best paper award committee
(’03,’05,’06,’10,‘12), ICDE Influential paper award committee (2004-2008), TCDE awards committee
(2014-2017 as advisor and member), DASFAA awards committee (2011-2019 as chair and member),
and many best paper award committees including ICDE2006 (as chair) helping to ensure high
academic standards.
\subsection*{Asia-Pacific}
I also was privileged to serve the Asia-Pacific community through steering committee activities of
DASFAA including chair, advisor, and awards chair for 15 years (1999-2014). An early contributor
from the 2 nd conference in 1991 as a PC co-chair, I helped establish the current stature of DASFAA
and globalize the DASFAA conferences. I am glad that DASFAA stands itself now as a prestigious
data engineering conference serving the world-wide community as well as the Asia-Pacific one. I also
had an opportunity to contribute to PAKDD as a life member of the steering committee and to the
Korea-Japan Database (KJDB) Working Group as a co-founder, chair, and advisor promoting active
academic exchanges through annual KJDB workshops.
\subsection*{Korea}
A no less important goal of my effort was to help bring up the level of data engineering research in
Korea to a global one, which was barely sprouting when I first came back to Korea in 1990. I served
as the chair of the Special Interest Group on Databases of the Korea Information Science Society
(SIGDB of KISS—later renamed to be the Database Society of KIISE) in early 90’s and the president
of the Korean Institute of information Scientists and Engineers (KIISE) in ‘20’s, through which I
promoted globalization of computer science and data engineering research in Korea—including
hosting VLDB2006, PAKDD2003, and DASFAA2004 in Seoul and creating the KIISE JCSE journal
and IEEE BigComp conference with KIISE scholars. Today I am glad to see that the Korean data
engineering research community stands strong by global standards.

\subsection*{Leadership and Goals}
In all my effort in the leadership positions, my primary goals have been to ensure the highest
standards for publications and to vitalize the research activities, which I hope made whatever little
contribution to the advancement of our field.
I wish to share this honor with so many colleagues who selflessly took initiatives, helped, and
cooperated in various roles and responsibilities in the course of this decades-long saga. They are
true heroes who are behind this flourishing field of data engineering that we are enjoying today.
Thank you very much.
\end{document}

