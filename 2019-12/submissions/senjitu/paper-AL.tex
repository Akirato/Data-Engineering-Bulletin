\documentclass[11pt]{article} 

\usepackage{amsmath}
\usepackage{verbatim}
\usepackage{comment} 
\usepackage{algorithm}
\usepackage{algorithmic}
\usepackage{amsfonts}
\usepackage{amssymb}
%\usepackage{amsmath}
\usepackage{subfigure}
\usepackage{multirow}
\usepackage{flushend}
\usepackage{caption}
%\usepackage{xspace}
%\usepackage{xcolor}
%\usepackage{enumitem,url}
\usepackage{deauthor,times,graphicx}

\begin{document}
\title{Capturing Human Factors to Optimize Crowdsourced Label Acquisition through Active Learning }

\author{
Senjuti Basu Roy\\
New Jersey Institute of Technology\\
senjutib@njit.edu}


\date{}

\maketitle
\vspace{-0.2in}
\begin{abstract}
\vspace{-0.2in}
The goal of this article is to propose an {\em optimization framework by acknowledging human factors to  enable label acquisition through active learning }. In particular, we are interested to investigate tasks, such as, providing (collecting or acquiring) and validating labels, or comparing data using active learning techniques. Our basic approach is to take a set of existing {\em active learning techniques for a few well known supervised and unsupervised algorithms, but study them in the context of crowdsourcing, especially considering worker-centric optimization (i,e., human factors)}. Our innovation lies in  designing optimization functions that appropriately capture these two fundamental yet complementary facets, performing systematic investigation to understand the complexity of such optimization problems, and designing efficient solutions with theoretical guarantees.
\end{abstract}


\newcommand{\hide}[1]{}


\section{Introduction}  
\label{s:intro}
The past decade has observed a surge in the design and deployment of 
decentralized systems. A key reason for this surge is the growing desire in 
the society to have self-governing democratic financial systems that are not 
under the control of a privileged set of entities. A central control often 
translates to a forced trust model with limited provision to support 
transparency and accountability. The adoption of Blockchain, for example, 
is a by-product of the ability to break away from the forced-central control 
in a trust-worthy fashion~\cite{blockchain-book}. The emerging blockchain 
platforms facilitate a reliable execution of any digital contracts 
(i.e., transactions) in a decentralized manner despite the existence of malicious 
actors. At the core of any blockchain platform is a Byzantine 
fault-tolerant (\BFT{}) consensus protocol and a tamper-proof replicated 
ledger~\cite{bedrock,blockchain-book,scalable-ledger}. The \BFT{} 
protocol helps to achieve {\em consensus} on the order of incoming client 
requests among all the replicas, while the ledger logs this agreement. 

Traditional \BFT{} protocols expect a {\em permissioned} system where the 
identities of all the replicas (i.e., participants) are known prior to any 
consensus as they rely on having a verifiable voting right in a democratic 
setting. These protocols rely on a {\em communication-oriented} consensus 
model, where all the participants exchange endorsements across multiple 
rounds before they can reach a decision~\cite{sharper,pbftj,ahl,poe,rcc,geobft,flexitrust,ringbft,mirbft,basil,hotstuff}.
In these protocols, a system of $\n{}$ replicas can reach a common decision 
if at most $\f{}$ of them are malicious, such that $\n{} \ge 3\f{}+1$. 
The $\n{}$ parties are said to reach a decision when at least a majority 
of honest parties agrees to that decision. This decision is logged by 
requiring all the agreeing parties to {\em sign} the decision. Hence, the 
reached decision is considered {\em tamper-proof} because it has support 
of a majority of honest participants.

Despite being around for more than two decades, traditional \BFT{} protocols 
did not see any major practical applications until the introduction of 
blockchain technology. We attribute {\em two} key factors for this lack of 
adoption. (i) To ensure that the malicious participants do not spawn multiple 
identities, these \BFT{} protocols need an authority (i.e., {\em a forced 
trust gateway}) to verify and register every participant to verify every 
vote~\cite{sybil-attack}; some participants may find this intrusive if they 
do not want to reveal their personal information. (ii) To overwrite the ledger, 
malicious participants just require access to the private keys of honest 
participants. In a sense, the proof of the validity of the ledger is not 
self-contained, and it operates on the assumption that the private-keys 
are kept safe externally indefinitely.

To resolve these challenges, initial blockchain platforms such as 
Bitcoin~\cite{bitcoin} and Ethereum~\cite{ether} offer a {\em permissionless} 
model of consensus. These systems employ the {\em Proof-of-Work} (\PoW{}) 
protocol~\cite{bitcoin,ether}, which follows a {\em computation-oriented} 
consensus model and requires all the participants to compete with each other 
and try to solve a complex puzzle. Whichever participant solves the puzzle 
first gets to add a new entry ({\em block}) to the ledger. As a result, 
\PoW{} protocol eliminates the three challenges seen by traditional \BFT{} 
protocols. (i) Malicious participants can spawn multiple identities, but what 
actually matters is the available compute power. (ii) Each block includes 
the hash of the previous block; overwriting the ledger requires recomputing 
all the blocks making it computationally infeasible. (iii) Since reaching the 
consensus is based on presenting the proof of work that is embedded on the 
ledger (i.e., self-contained), there is no longer any need for external 
safe-keeping of private keys to sign endorsements.

These properties offered by \PoW{} protocol help blockchain platforms to 
design a {\em decentralized economy}, where any person can participate in the 
consensus process, and the economy has a self-generating currency to monetize 
its participants. Monetizing the participants is necessary as the \PoW{} 
protocol expects the participants to spend their resources to solve a complex 
puzzle. Clients of the Bitcoin platform, create transactions that exchange 
Bitcoins and send them to the participants (miners) in the \PoW{} protocol. 
These miners check if the transaction is valid; the client has sufficient 
Bitcoins to transfer. If the transaction is valid, they run \PoW{} protocol to include 
this transaction in the ledger. The winning miner of \PoW{} gets a portion of 
the client's Bitcoin as {\em fees}, while the mining process (\PoW{}) mints  
new tokens to fund the economy. This new token is transferred to the winning 
miner's account and is recorded as a transaction in the block.

The key challenge with platforms like Bitcoin is their {\em practicality}. 
These platforms have abysmally low throughput in the order of $10$ transactions 
per second in part due to inadequate choice of small block sizes. Furthermore, as 
more miners join the network, the complexity of the puzzle has to be increased. 
For example, the complexity of the current Bitcoin puzzle is so high that the 
miners work in large groups to have any positive probability of creating the 
next winning block~\cite{blockchain-book}. Moreover, as miners are competing 
with each other, it leads to massive wastage of computational resources (energy) 
as only the winning miner's efforts are recorded and rewarded. This results in an unsustainable ecosystem~\cite{badcoin,badbadcoin}.

We observe these challenges in the designs of existing \BFT{} protocols and 
blockchain platforms and envision a \DualChain{} system that learns from these 
models and eliminates their key challenges. Essentially, we aim to establish a 
new research agenda; a new field of hybrid consensus protocols that depart from 
competitive consensus to a collaborative consensus that is both resilient and 
sustainable. Our \DualChain{} architecture takes a step in this direction by 
running two consensuses on each client transaction while ensuring there is no 
increase in the latency observed by the client. Each client request is first 
ordered through a state-of-the-art \BFT{} consensus protocol ({\em commitment}), 
subsequently, this ordered request is engraved into the ledger through the 
\PoW-style consensus ({\em settlement}). Specifically, \DualChain{} causes no 
increase in commitment latency while improving the settlement latency observed 
by existing protocols. Ordering the client transaction through a \BFT{} consensus 
protocol first allows our \DualChain{} system to guarantee the following benefits: 
(i) clients receive low-latency responses, and (ii) \PoW{} participants no longer 
need to compete, resulting in a high-throughput sustainable chain. As a result, 
instead of employing the \PoW{} for consensus, we design a novel protocol that 
allows miners to collaborate. We refer to this paradigm as 
{\em Power-of-Collaboration} (\PoC{}). 

Our \PoC{} protocol splits the complex puzzle into disjoint slices and requires 
each miner to work on a distinct slice. This slice distribution significantly 
reduces the resource wastage and provides each honest miner with a reward 
for each new transaction added to the ledger. As each ledger entry is added 
collaboratively, any malicious entity that wishes to overwrite the ledger 
needs to match the computational power of all the existing miners making it 
practically impossible. These features of our \DualChain{} system make it 
lucrative; its design is the bedrock for a secure and efficient decentralized 
economy.

\vspace{-0.2in}
\section{Data Model}\label{dm}
\vspace{-0.1in}
We introduce different variables and characterize human factors~\cite{hf1,motiv00,motiv0,motiv1,motiv2,motiv3,motiv4}. A crowdsourcing platform typically comprises of workers and tasks that serve as the foundation of the framework we propose. We also note that not all the variables are pertinent to every application domain (for example, citizen science applications are usually voluntary contributions). Our effort is to propose a generalization nevertheless. 

{\bf Domains/types:} A set $D=\{d_1,d_2,\ldots,d_{l}\}$  of given domains is used to describe the different types of tasks in an application. Using Example~\ref{ex1}, a particular species may construe a domain. 

{\bf Workers:} A set of $m$ human workers $\mathcal{U}=\{u_1,u_2,\ldots,u_m\}$ are available in a crowdsourcing platform.
% Each worker $u$ is described by a set of  {\em human factors}~\cite{hf1,motiv00,motiv0,motiv1,motiv2,motiv3,motiv4}, i.e., variables that are important to understand their behavior in a crowdsourcing platform.

{\bf Tasks and sub-tasks:} A task $\mathcal{T}$ is a hybrid human-machine computational task (classification for example), with a quality condition $Q^\mathcal{T}$  and an overall monetary budget $B^\mathcal{T}$ that decide its termination. Using Example~\ref{ex1}, $\mathcal{T}$ is a classification task which terminates, when $Q^\mathcal{T} = 80\%$ accuracy is achieved, or $B^\mathcal{T} =\$100$ is exhausted.

Without loss of generality, $\mathcal{T}$  comprises of a set of $n$ subtasks, i.e., $\mathcal{T} =\{t_1,t_2,\ldots,t_n\}$. These sub-tasks are of interests to us,  as workers will be involved to undertake these sub-tasks. Each sub-task can either be performed by human workers or computed (inferred) by machine algorithms. We consider {\em pool based active learning}, where a finite pool of sub-tasks exists and given.

{\em Sub-tasks:}
For single label, a sub-task is an unlabeled instance of the data that requires labeling. Considering Example~\ref{ex1}, this is analogous to confirming the presence or absence of a species in a particular geographic location. 
%To generalize, we consider similar settings as that of {\em pool-based active learning}, where a pool of unlabeled instances are available for further labeling.
For multi-label scenario, a sub-task requires multiple labels to be obtained. Using Example~\ref{ex1}, this is analogous to obtaining Kingdom, Phylum, Class, Order, etc of the insect.

{\bf Worker Response:} We assume that a worker $u$'s {\em response to a particular sub-task $t$ may be erroneous, which is used by the machine algorithm in one or more rounds of interactions}. Our framework may ask multiple workers to undertake the same task to reduce the error probability, and may decide which questions to ask in the next round to whom based on the answers obtained in the previous round.

{\bf Human Factors:} These are the variables that characterize the behavior of the workers  in a crowdsourcing platform~\cite{hf1,motiv00,motiv0,motiv1,motiv2,motiv3,motiv4}.

{\em Skill (Expertise/Accuracy):}  Worker's skill in a domain is her expertise/accuracy. Skill of a worker in a domain $d$ is quantified in a continuous $[0,1]$ scale (to allow a probabilistic  interpretation). A worker $u$ may have skills in one or more domains (e.g., different species observation accuracy). 
%Given $l$ domains, $u$'s skill is described by a $l$-dimensional vector, where $s^{u_i}$ is her skill in the $i$-th domain. 

{\em Wage:} A worker $u$ may have a fixed wage $w_u$, or may have to accept the wage a particular task offers. $u$'s may have different wage for different types of tasks.

{\em Motivation:} Motivation aims at capturing the worker's willingness to perform a task. A related work~\cite{motiv1} proposes a theoretical foundation in motivation theory in crowdsourcing platform and characterizes them in two different ways: 

{\em (a) Intrinsic motivation:} Intrinsic motivation exists if an individual works for fulfillment generated by the activity (e.g. working just for fun). Furthermore, related works~\cite{motiv1,motiv2,motiv3} have identified that intrinsic motivation emerges in the following ways: (1) skill variety (refers to the extent to which a worker can utilize multiple skills), (2) task identity (the degree to which an individual produces a whole, identifiable unit of work, versus completion of a small unit which is not an identifiable final product), (3) task significance (the degree to which the task has an influence over others), (4) autonomy (the degree to which an individual holding a job is able to schedule his or her activities), (5) feedback (the extent to which precise information about the effectiveness of performance is conveyed). 

%Hackman and Oldham~\cite{hackman1976motivation} have combined these factors mathematically and defined {\em motivating potential score (MPS)} to capture intrinsic motivation:

%\begin{equation}\label{eqn}
%\begin{aligned}
%MPS= \frac{\text{skill-variety} + \text{task-identity} + \text{task-significance}}{3}  \\
%* \text{ autonomy } * \text{ feedback}
%\end{aligned}
%\end{equation}


{\em (b) Extrinsic motivation:} Extrinsic motivation is  an instrument for achieving a certain desired outcome (e.g. making money).

%{\em Acceptance ratio:} Acceptance ratio describes the {\em probability} at which a worker actually participates in a task (notice that a worker can always decline a task).Acceptance ratio may be correlated to worker motivation.
 %Given $l$ domains, $u$'s acceptance ratio is described by a $l$-dimensional vector $A^u$. 

The challenge however is, either the values of these factors have to be explicitly given or they have to be estimated. Related works, including our own, have proposed solutions to estimate skill~\cite{skill,DBLP:conf/kdd/JoglekarGP13} by analyzing historical data. %Acceptance ratio or wage of the workers are estimated by designing surveys and asking explicit questions~\cite{roy2015task}.
Nevertheless, we are not aware of any effort that models motivational factors or design optimization involving them.

{\em Worker specific constraints:} Additionally, a worker may specify certain constraints (e.g., can not work more than $6$ hours, or travel farther than $10$ miles from her current location).

{\bf Characterizing sub-tasks considering human factors:}  It is easy to notice that the motivational factors described above are actually related to tasks (i.e, sub-tasks). 

Formally, we describe that a set $A$ of attributes or meta-data is available to characterize each sub-task $t$. They are its required skill-domain\footnote{\small for simplicity, we assume that each sub-task requires one skill, whereas, in reality, multiple skills may be needed for a sub-task. The latter assumption is trivially extensible by our framework.} $s^t$ , cost/wage $w^t$, duration $time^t$, location $location^t$, significance $sig^t$, identity $iden^t$, autonomy $auto^t$, task feedback $fb^t$. Each $t$, if performed correctly, contributes by a quantity  $q^t$ to $Q^\mathcal{T}$. These contributions are purely dictated by the active learning principles, such as how much it reduces the uncertainty.



%{\em Unlike other human factors, there does not exist any mathematical model that captures human motivation}. We therefore make an effort to mathematically model human motivation. 

 %and has a cost $b^t$. The cost of a sub-task is the money that has to be paid when it is performed by a human worker.

 







\vspace{-0.2in}
\section{Worker-Centric Optimization through Human Factors Modeling}\label{hf}
\vspace{-0.1in}
Recall Section~\ref{intro} and note that worker-centric optimization is a common theme across single and multi-labels tasks, which we first examine here.

{\bf Objectives:}
Our objective here is to explore mathematical models for worker-centric optimization in crowdsourcing platforms. Specifically, given an available pool of tasks and workers where workers perform repetitive tasks, we first obtain human factors of the workers by analyzing their past tasks and then study the problem of task assignment to enable worker-centric optimization. A recent work performs an ethnomethodological study at {\em Turker Nation}\footnote{\small http://www.turkernation.com/} and argues~\cite{martin2014being} that it is critical to enable worker-centric optimization.
Our effort here is to make a formal step towards that goal, independent of any specific system-centric optimization (i.e., the active learning principles). Therefore, such a study has a much broader applicability that goes beyond active learning. Of course, our framework will ultimately combine both system and worker-centric criteria. 

 {\bf Challenges:} While the significance of human factors is well-acknowledged in crowdsourcing, the challenge is to be able to estimate them effectively and propose appropriate models that could capture them during task assignment. Added to the fact is the dependence of the underlying crowdsourcing domain, which makes some of these factors more important than the rest (e.g., unlike AMT, there is no monetary pay-offs in citizen science activity, but skill variety is acknowledged to be critical to avoid monotony). 

%{\bf Our Prior Work:} In our recent investigation, we have studied some human-factors, such as, worker skill (accuracy), wage (pay-off), acceptance ratio of tasks, social affinity~\cite{hf1,roy2015task,DBLP:journals/corr/RahmanRTAD15} and proposed mathematical functions to include them explicitly in task assignment. We have designed medium-scale user studies~\cite{roy2015task,DBLP:journals/corr/RahmanRTAD15} in Amazon Mechanical Turk (AMT), involving hundreds of workers to learn the relationship between these factors and have observed that skill, wage, acceptance ratio - all follow normal distributions. We also have observed that a strong positive correlation exists between worker accuracy and expected wage. These studies have demonstrated empirically that task assignment considering human factors yields superior outcomes compared to the one without. In another recent work, we present how to combine task relevance (Boolean match of worker's expertise with the skill requirements of the tasks) and motivation in the task assignment process~\cite{edbt172}, albeit in a non-active learning context. Moreover, this work only considers {\em skill-variety} in formalizing motivation. In that sense, this latter work is still preliminary and does not solve the problem we propose. A recent survey paper summarizes different aspects of worker centricity in crowdsourcing platforms~\cite{amer2016toward}. 


\vspace{-0.1in}
\subsection{Proposed Directions}
\vspace{-0.1in}
First, we propose how to model and estimate human factors~\cite{hf1,motiv00,motiv0,motiv1,motiv2,motiv3,motiv4} that are pertinent to capture motivation using the variables that are described in Section~\ref{dm}. Then, we describe mathematical models that leverages these estimated human factors to explicitly assign tasks to workers. 

{\bf Estimating human factors:} We leverage the past task completion history of the workers as well as the new tasks to compute a Boolean task completion matrix $T$, where the rows are the workers and the columns are the (sub)-tasks. If a worker $u$ has completed a (sub)-task $t$ successfully in the past, the corresponding entry gets a $1$, it gets a $0$ otherwise. We assume that the factors that capture intrinsic motivation, i.e., skill variety, task identity, task significance, autonomy, feedback are independent yet latent variables. The second matrix we consider is the task factor matrix $\mathcal{F}$, where the rows are the tasks and the columns are the motivation related latent variables.  The final matrix is the user factor matrix $\mathcal{U}$ where rows are the factors and columns are the users. This matrix could be fully latent or observed.  In case it is latent, we minimize the error function, as defined below:

\begin{equation}
\sum_{i,j}(t_{ij}-\mathcal{U}_iF_j)^2+\lambda(\|\mathcal{U}\|^2+ \|\mathcal{F}\|^2)
\vspace{-0.1in}
\end{equation}

Here, $\lambda$ is the regularization parameter. The goal is to find ${\mathcal U}$ and ${\mathcal F}$ such that it minimizes the error. For any new worker and new task, the predicted task completion score is calculated by multiplying $U_i$ with $F_j$. Here, the important thing is to notice that the optimization function only minimizes the error for which ratings are present.  We apply the alternating least square approach~\cite{stigler1981gauss} to solve this problem. This is an iterative approach, where at each iteration, we fix the tasks' latent factor matrix $\mathcal{F}$ in order to solve for $\mathcal{U}$ and vice versa. We have designed a similar solution for predicting tasks to workers considering implicit workers' feedback\cite{DBLP:journals/corr/RahmanJR16}.

 
{\bf Worker-Centric Task Assignment:} The solution above only estimates the intrinsic motivational factors, but does not describe how to aggregate them together or combine with extrinsic motivation to perform worker-centric task assignment. 

Psychologists Hackman and Oldham~\cite{hackman1976motivation} have combined factors associated to intrinsic motivations defined {\em motivating potential score (MPS)} :

\begin{equation}\label{eqn}
\begin{aligned}
MPS= \frac{\text{skill-variety} + \text{task-identity} + \text{task-significance}}{3}  
 *\text{ autonomy } * \text{ feedback}
\end{aligned}
\end{equation}

Considering this aforementioned formulation, we study the worker-centric task assignment as a global optimization problem to maximize the {\em aggregated intrinsic and extrinsic motivation}. For a given set of tasks $S^{t_u}$,  $V(S^{t_u})$ represents the overall motivation for worker $u$, by combining  her extrinsic motivation  (EXTM) (recall Section~\ref{dm} that EXTM could be modeled using wage $w^t$) and intrinsic motivation, i.e, {\em motivating potential score (MPS)}(refer to Equation~\ref{eqn})~\cite{hackman1976motivation}. In our initial effort, we combine them linearly, as that allows us to design efficient algorithms. Assigning a set of tasks per worker is reasonable as well as desirable from worker's perspective, because workers in a typical crowdsourcing platform intend to undertake multiple tasks as opposed to a single task. Workers may also have constraints, such as,  not spend more than $X^u$ hours, 
or the aggregated wage must at least be $b^u$ dollars. 
 
Technically, we want to assign tasks to the workers {\em to maximize the aggregated motivation, such that the assignment satisfies each worker-specific constraints}. One such optimization function is described in Equation~\ref{eqn:eq3} (Recall Section~\ref{dm} where $time^t$ and $w^t$ are the duration and wage of sub-task $t$, respectively).

\begin{equation}\label{eqn:eq3}
 \text{ Maximize }  \sum_{u \in \mathcal{U}} [V(S^{t_u}) =  EXTM(S^{t_u}) + MPS(S^{t_u})]
\end{equation}
\vspace{-0.2in}
\begin{align*}
 V(S^{t_u}) =
\begin{cases} 
 \\ \text{ if } \sum_{t \in S^{t_u}} time^t \leq X^u \text{ and } \sum_{t \in S^{t_u}} w^t \geq b^u \\
0 \text{\qquad otherwise} 
\end{cases}
\vspace{-0.1in}
\end{align*} 


%Recall that a task $t$ is associated with a set $A$ of attributes, whose values, we assume are provided by the domain experts and are available at our disposal. Formally, for each task, we know the required skill domain $s^t$, duration $time^t$, significance $sig^t$, identity $iden^t$, autonomy $auto^t$, feedback $fb^t$, and wage $w^t$.

As a simple example, given two tasks $i$ and $j$, we can add the individual significance $sig^i+sig^j$, identity $iden^i+iden^j$, autonomy $auto^i+auto^j$, or feedback $fb^i+fb^j$. Similarly, the  wage of two tasks could also be added and normalized to compute EXTM. 
Alternative problem formulation is explored below.
%However, in order to compute MPS,  we still need to model and quantify the {\em skill variety} of a given set of tasks. 

%{\em Modeling Skill-variety:} We intend to study two different ways to capture skill variety. 

%(1) Set based modeling: Simple set based measures, such as, {\em Jaccard Distance}~\cite{baeza1999modern} can  be used to quantify {\em skill variety} between a given set of tasks. As an example, given $2$ tasks $t_1,t_2$ from citizen science, if $s^{t_1}=\text{humming bird}$, $s^{t_2}=owl$, skill variety $skill-variety(t_1,t_2)=1$. 

%In some application, it is reasonable to assume that an {\em intrinsic diversity ordering}~\cite{vee2008efficient} between the task attributes are available which can potentially be used to quantify skill-variety. For example, two sub-tasks associated with the same location that require to observe two different species may require more skill variety than two other sub-tasks that require observing the same species but from two different locations. We formalize this as, $\text{skill-domain} \prec \text{location}$. If such a diversity ordering is available from the domain experts, we model skill-variety considering diversity ordering~\cite{vee2008efficient}.

%(2) Order based modeling: An interesting alternative of the set based skill-variety computation  is to design an {\em ordering based}~\cite{DBLP:conf/icde/RoyDAY11,cao2012keyword} solution, where a {\em chain} of sub-tasks will be designed for a worker. Using Example~\ref{ex1}, this would mean creating an ordering $\mathcal{R}$ of the tasks for each worker where the skill-variety is ideally very high. Unlike set based approaches, here, skill-variety of a given set of sub-tasks will be computed considering each $k$ adjacent sub-tasks in the designed chain ($k=2$ means pair-wise adjacent tasks). As a simple example, if $3$ sub-tasks $t_2 \rightarrow t_1 \rightarrow t_3$ are designed for a worker, for $k=2$,  $skill-variety(t_2 \rightarrow t_1 \rightarrow t_3)= Jaccard(s^{t_2},s^{t_1})+ Jaccard(s^{t_1},s^{t_3})$.  %This modeling could also be extended when diversity based ordering between the attributes are available.


%The directions described above give rise a number of interesting computation problems.

%MPS formula requires us to quantify {\em skill variety} for a given set of tasks. We note that modeling skill variety leads to interesting research problems. Additionally, how to solve the objective function described above efficiently remains to be a challenging problem.
\vspace{-0.1in}
\subsection{Open Problems}
\vspace{-0.1in}
 {\bf  Solving the optimization problem:}
How to design an effective solution to maximize worker motivation based on the aforementioned objective function formulation is challenging. %The problem becomes even more complex when skill-variety is modeled as a chain. Even when set based modeling is considered, 
We observe that the proposed optimization problem is NP-hard~\cite{garey1979computers}, using a reduction from the assignment problems~\cite{roy2015task}. In a recent work, we have modeled motivation using {\em only skill-variety} and we have proved that the problem is NP-hard using a reduction from the Maximum Quadratic Assignment Problem~\cite{arkin2001approximating}. For our problem, we note that an integer programming based solution is simply not scalable. We will explore greedy heuristic strategies that are effective and efficient. For example, we will assign tasks to the workers greedily based on the marginal gain~\cite{roy2015task}. 
%if the set-based formulation admits {\em sub-modularity}~\cite{lovasz1983submodular}, as {\em skill-variety} exhibits {\em diminishing return properties}, and other motivational elements simply increase as more tasks are added. If this property is indeed satisfied, we will be able to design greedy approximation algorithms~\cite{nemhauser1978analysis} with theoretical guarantees. For the ordering based model, we will explore orienteering like solutions~\cite{chao1996fast} to design efficient greedy heuristic algorithms.

\noindent {\bf  Complex modeling for estimating intrinsic motivation \& task assignment:} In our preliminary direction, we have assumed that variables associated with intrinsic motivations are independent and could be combined as suggested by Hackman and Oldham~\cite{hackman1976motivation}, or intrinsic and extrinsic motivation could be combined linearly. In reality, that may not be the case. In this open problem, we will study the feasibility of a probabilistic model~\cite{zhao2012bayesian}, namely a {\em hierarchical Bayesian framework}~\cite{liu2014framework} for this problem. If the worker is completely new in the platform, we will bootstrap to collect a small set of evidence. We will consider each of the variables associated with worker motivation as a random variable and present a model using hierarchical Bayesian Networks~\cite{jensen1996introduction} by encoding a joint distribution of these variables over a multi-dimensional space.  This model will first establish the relationship among the intrinsic motivational variables themselves and then between intrinsic and extrinsic motivation to capture a workers' ``preference'' to a given task. We will apply Constraint Based, Score-Based, and Hybrid methods to learn the structure of the network~\cite{tsamardinos2006max}. We will leverage {\em Bayesian Parameter Estimation as well as Maximum Likelihood Estimation techniques} to learn the parameters of the constructed network. For efficient parameter estimation considering this complex joint distribution, we will use Gibbs sampling~\cite{carter1994gibbs}. 

%As outlined in the aforementioned open problem, if we are successful to model the motivational variables using a graphical model through probabilistic modeling, we shall extend that model to a hierarchical Bayesian Network with constraints that 

%In our preliminary work, we have assumed that variables associated with intrinsic motivation are combined as suggested by Hackman and Oldham. Moreover, we have aggregated intrinsic and extrinsic motivation linearly, as described in Equation~\ref{eqn:eq3}. In this open problem, we will explore alternative modeling. 






\vspace{-0.2in}
\subsection{Optimized Single-Label Acquisition Involving Crowd}\label{label}
\vspace{-0.1in}
We now investigate our proposed optimization framework for single-label acquisition. This problem is examined by augmenting active learning principles with worker-centric optimization (refer to Section~\ref{hf}).

{\bf Objectives:} We are assuming a setting where single-label acquisition is difficult, expensive, and time consuming (such as, Example~\ref{ex1}). We adapt a set of popular as well as well-known active learning principles\cite{al1,qbc1,qbc2,error-reduction} that are proposed to optimize system-centric cirteria, such as, {\em minimizing uncertainty or maximizing expected error-reduction} that are known to be effective in supervised (classification) algorithms~\cite{al-svm,al-svm2,al-dtree,korner2006multi}. We augment these active learning principles with worker-centric optimization. Given a pool of unlabeled instances  (of sub-tasks) and an available set of workers, the objective is to select sub-tasks for further labeling and assign workers for annotations, such that, the assignment optimizes both system and workers. The same sub-task may be annotated  by multiple workers.

{\bf Challenges:} An oracle, who knows the ground truth, no longer exists in crowdsourcing; instead, multiple workers, with varying expertise (skill), are available. Under this settings, how to realign traditional active learning goals that are system-centric (i.e., optimizes underlying computational task) requires further investigations. How to systematically design {\em optimization function}, i.e., one that combines worker-centric optimization in traditional active learning settings~\cite{active-learning-cs1,active-learning-cs2} is the second important challenge. An equally arduous challenge is the efficiency issue which is mostly overlooked in the existing research. Finally, when to terminate further label acquisition also needs to be examined.

\vspace{-0.1in}
\subsection{ Proposed Directions}
\vspace{-0.1in}
Our overall approach is iterative, where, in each round a set of sub-tasks are selected for annotation and a set of workers are chosen. Once annotations are received, the underlying classification model is retrained. After that, either the process terminates or we repeat. It has three primary directions: (1) {\em in a given round, which sub-tasks are to be selected for annotation and assigned to which workers?} (2) {\em  how to aggregate multiple annotations to obtain the ``true'' label?} (3) {\em when to stop?}  

{\em Which sub-tasks are to be selected and assigned to which workers?} We take a set of well-known active learning techniques, such as, {\em uncertainty sampling \cite{al1}, query-by-committee \cite{qbc1,qbc2}, or expected-error reduction~\cite{error-reduction}, used in popular classification algorithms, such as, Naive Bayes~\cite{entropy}, SVM~\cite{al-svm,al-svm2}, Decision Trees~\cite{al-dtree}, or ensemble classification\cite{korner2006multi}} and study them in crowdsourcing.

When a single classifier with a binary classification task is involved and the classifier is probabilistic (such as Naive Bayes), we consider existing uncertainty sampling~\cite{al1} techniques. We use entropy~\cite{entropy} to model uncertainty to choose that sub-task for labeling whose posterior probability of being positive is closest to $0.5$. For non-probabilistic classifiers (such as SVM or Decision Tree), we explore {\em heterogeneous approach}~\cite{lewis1994heterogeneous}, in which a probabilistic classifier selects sub-tasks for training the non-probabilistic classifier. We also study existing expected-error reduction~\cite{error-reduction} techniques that select the sub-tasks to minimize the expected future error of the supervised algorithm, considering {\em log-loss or $0/1$-loss}. We study the query-by-committee\cite{qbc1,qbc2} technique, we choose that sub-task for further labeling which has the {\em highest disagreement}. 

Active learning principles  mentioned above are too {\em ideal} to be useful in a crowdsourcing platform. A simple alternative is to design a {\em staged solution}, where we first select the tasks and then the workers~\cite{active-learning-cs1}. For us, we can take the task-selection solution from~\cite{active-learning-cs1} and then plug in our worker-centric optimization (Section~\ref{hf}) to compose tasks for the workers. We, however, argue that such a staged solution is {\em sub-optimal}, simply because, tasks selected by {\em active learning} techniques may end up having a very low worker-centric optimization, resulting in poor outcome overall. We therefore propose a global optimization that combines (1) worker-centric goals (recall Equation~\ref{eqn:eq3}). (2) active learning principles considering workers with varying expertise. 

Recall Section~\ref{dm} and note that $q^t$ represents sub-task $t$'s contribution towards a given active learning goal (for example, how much $t$ reduces uncertainty or expected-error) at a given iteration. Let $S^{t_u}$ represent the sub-tasks assigned to $u$ with value 
$V(S^{t_u})$ (recall Equation~\ref{eqn:eq3}). Considering worker's skill $s^{u_t}$ as a probability, $u$'s {\em expected contribution} to $t$ is  $s^{u_t} * q^t$~\cite{clemen2007aggregating}. One possible way to combine them is as a multi-objective global optimization function where the objective is to select sub-tasks and workers that maximize a weighted linear aggregation of worker and task-centric optimization (Equation~\ref{eqn:eq2}, where $W_1,W_2$ are specific weights). While linear aggregation is not the only way, it is more likely to admit efficient solutions, where the weights are tunable by domain experts (by default, $W_1=W_2=0.5$). 


%In our initial direction, we combine them in a weighted linear fashion  considering weights by combining both worker and task-centric criteria, where the latter is modeled as a linear weighted aggregation by multiplying $q^t$ with $s^{u_t}$ ($u$'s skill/accuracy in task $t$).

\begin{equation}\label{eqn:eq2}
 \text{ Maximize } \mathcal{V} =  \sum_{u \in \mathcal{U}} [W_1 * V(S^{t_u}) +  W_2 * \sum_{t \in S^{t_u}} (s^{u_t}*q^t)]
\end{equation}

Additionally, if a task has a cost budget associated that could be assigned either as a constraint to this optimization problem, or we could use cost as another objective as part of the optimization function, akin to one of our recent works~\cite{roy2015task}. Nevertheless, we acknowledge that designing the ``ideal'' optimization model that suffices the need of every application is practically impossible. We address this in the open problems.

{\em  Aggregating multiple annotations:} Another challenge is how to combine annotations from multiple workers with varying expertise to obtain the ``true'' label. We apply weighted majority voting types of approach~\cite{ho2013adaptive}, where the weights are chosen according to the skills of the workers. We also consider iterative algorithm for this purpose. Examples of iterative techniques include EM or Expectation Maximization\cite{hung2013evaluation}. The main idea behind EM is to compute in the $E$ step the probabilities of possible answers to each task by weighting the answers of workers according to their current expertise, and then to compute in the $M$ step re-estimates of the expertise of workers based on the current probability of each answer. The two steps are iterated until convergence.  We explore Bayesian solution~\cite{clemen2007aggregating} to probabilistically obtain the true label, i.e., given workers' annotations and skill, compute $Pr(t=0)$ and $Pr(t=1)$ and choose the one which has the higher probability.

%There exists a number of complex technical open problems.
\vspace{-0.1in}
\subsection{Open Problems}\label{opp}
\vspace{-0.1in}
{\bf  Solving the optimization problem} Solving the optimization problem described above is challenging. In a very recent work, we have formalized  task assignment as a linear combination of task relevance (based on a Boolean match between worker expertise and the skill requirements of a task) and skill-diversity~\cite{edbt172} and proved the problem to be NP-Complete~\cite{feo1990class,feo1992one}. We use Maximum Quadratic Assignment Problem (MAXQAP in short)~\cite{arkin2001approximating} to design an efficient algorithm with approximation factor $1/4$. For our problem, we will examine if it is at all possible to design an objective function (perhaps as a special case) to exploit its nice structural properties, such as, {\em sub-modularity or cancavity}. Such an effort is made for active learning problems recently~\cite{hoi2006batch} without considering human workers. We will also study the possibility of staged algorithms and heuristic solutions, as described above. To make the algorithm computationally efficient, we will examine how to design incremental active learning strategies~\cite{qi2009two}, such as finding the new classification model that is most similar to the previous one, under a set of constraints. 


{\bf Complex function design and stopping condition} We note that the formulation described in Equation~\ref{eqn:eq2} is rather {\em simple} - a linear function may not be adequate to combine worker and task-centric optimization. We will explore non-linear multiplicative functions. Another possible way is to formalize this as a bi-criteria optimization problem and design pareto-optimal solution that does not require us to assign any specific weight to the individual functions~\cite{bilo2004pareto,anagnostopoulos2012online,asudeh2014crowdsourcing}.  Finally, we  will examine {\em when to terminate this iterative process}. For the overall classification task $\mathcal{T}$, when quality threshold is not reached or budget is not exhausted (these are two hard stopping conditions), we will design stopping condition by measuring the  confidence~\cite{vlachos2008stopping} of the classification model, or availability of suitable workers.

{\bf Develop a number of optimization models that are likely to cover a variety of scenarios}  We realize that what constitutes the ``ideal'' optimization model is an extremely difficult problem and highly application dependent (e.g., Which factors are important? Should we add or multiply different human factors? In the case of linear weighting, what should be the weighting coefficients?). Even a domain expert who is very knowledgeable about the specific application may not be able to shed enough light on this. We hope to develop a rich set of different models that will cover the various types of applications. This idea of developing a set of optimization models draw parallels from Web Search and Information Retrieval - where a set of alternative criteria, such as relevance, diversity, and coverage, are considered~\cite{baeza1999modern}. In our case, this is analogous to developing models that only consider workers skills/expertise, or cost, or motivation, or includes a subset of human factors that we are interested to study in this project. 
\vspace{-0.2in}
\section{Optimized Multi-Labels Acquisition Involving Crowd}\label{unlab}
\vspace{-0.1in}
We now investigate the multi-labels acquisition scenario. We are unaware of any related work that performs multi-label acquisition in an active learning settings involving crowd. Although one can transform a multi-label task to several single-label tasks, this simple approach can generate many tasks, incurring a
high cost and latency. Akin to the previous section, our effort is to design solutions that adapt a few recent active learning works~\cite{multi0,multi1,multi2,multi3} for multi-label acquisition and combine that with worker-centric optimization, described in Section~\ref{hf}.

{\bf Objectives:} We will adapt a few known active learning algorithms for multi-label classifications using Support Vector Machine (SVM), Naive Bayes, or Ensemble classifiers~\cite{multi0,multi1,multi2,multi3}. We will combine and augment them with {\em worker-centric optimization through human factors modeling}. Using Example~\ref{ex1}, this is akin to selecting the most appropriate unidentified image of the species and select the most appropriate workers to provide multiple labels. Since a task could be labeled by multiple workers, we will study how to aggregate multiple responses and infer the correct labels (truth inference problem) of a task. We will also explore the use of correlations among different labels to improve the inference quality. Finally, we will investigate the stopping condition or {\em convergence criteria}. 



{\bf Challenges:} Workers may exhibit different characteristics in multi-label tasks: a conservative worker would only select labels that the worker is certain of, while a casual worker may select more labels. To determine the multi-label tasks’ results, the key is to
devise the so-called ``worker model'' to accurately express the behavior of the worker in answering multi-labels. Furthermore, different from single-label tasks, correlations among labels inherently exist in multi-label tasks. For Example~\ref{ex1}, consider one pairwise label dependency: if the insect in the image is labeled as Papilionidae (Family name) , then it is highly probable that it also has label
Swallowtail (Sub-family name). Therefore, how to understand and leverage label correlation is another challenge. Finally, how to systematically design {\em optimization function}, i.e., one that combines worker-centric optimization in active learning settings~\cite{multi0,multi1,multi2,multi3} is the final important challenge.


%An oracle, who knows the ground truth, no longer exists in crowdsourcing; instead, multiple workers, with varying expertise (skill), are available. Under this settings, how to realign traditional active learning goals that are system-centric (i.e., optimizes underlying computational task) requires further investigations. How to systematically design {\em optimization function}, i.e., one that combines worker-centric optimization in traditional active learning settings~\cite{active-learning-cs1,active-learning-cs2} is the second important challenge. An equally arduous challenge is the efficiency issue which is mostly overlooked in the existing research. Finally, when to terminate further label acquisition also needs to be examined.

\vspace{-0.1in}
\subsection{Proposed Directions}
\vspace{-0.1in}
Our overall approach is iterative here as well, where, in each round a set of sub-tasks are selected to be annotated with multi-labels and a set of workers are chosen. Once multiple labels are acquired, the underlying classification model is retrained. After that, either the process terminates or we repeat. It has three primary directions: (1) {\em Task assignment} (2) {\em  Truth Inference, i.e.,  aggregate multiple annotations to obtain the ``true'' labels.} (3) {\em Label Correlation}.

{\em Task Assignment:} In our preliminary investigation, we have studied the active learning problem for the multi-label scenario considering the widely popular SVM classifier using the {\em Maximum-Margin Uncertainty Sampling}. Uncertainty sampling \cite{al1} is one of the simplest and most effective active learning strategies used for single-label classification. The central idea of this strategy is that the active learner should query the instance which the current classifier is most uncertain about. For binary SVM classifiers, the most uncertain instance can be interpreted as the one closest to the classification boundary by selecting the sample with the smallest classification margin. Multi-label active learning methods simply extend this binary uncertainty concept into the multi-label learning scenarios by integrating the binary
uncertainty measures associated with each individual class in independent manners, such as taking the minimum over all classes, and taking the average over all classes.

In our initial direction, given the active learning principle, we combine that with worker-centric optimization and design an objective function akin to Equation~\ref{eqn:eq2}, as described in Section~\ref{label}. Obviously, exploring alternative optimization models, or how to design a set of optimization functions that can handle a variety of scenarios, or when to stop the iterative process are additional challenges. Once we understand these challenges for the single-label acquisition problem in Section~\ref{label}, we believe they will extend for the multi-label scenarios.

{\em  Truth Inference Problem:}
The truth inference problem, i.e, how to aggregate the annotations provided by multiple workers and generate the actual set of labels requires deeper attention for the multi-label scenario. As the correct set of labels associated with each sub-task is unknown (ground-truth is unknown), the accuracy or expertise of a worker can only be estimated based on the collected answer. To model worker expertise, we compute the following two measures, {\em True Positive (TP)} and {\em False Positive (FP)}. TP is the number of labels that a worker selected correctly and FP is the number of labels she selected incorrectly. Unlike a prior work~\cite{zhao2012bayesian}, False Negative and True Negative are not relevant, if the workers annotate the labels. In the case where workers validate the given labels, these latter two measures are also relevant. Once these measures are computed, we design a worker's contingency table and calculate her expertise. After that, we design an iterative approach, which can jointly infer the correct labels associated with the tasks and the expertise of the workers.  Our iterative solution is motivated by the Expectation Maximization (EM) algorithms and comprises of the following two steps: (step 1), we assume that the worker expertise is known and constant, and infer the probabilistic truth of each object and label pair. (step 2), based on the computed probabilistic truth of each object and label pair, we re-estimate workers expertise.

{\em  Label correlation:}
Since the annotated labels of an object are not independent (Recall Example~\ref{ex1} and note that Papilionidae (Family name) and Swallowtail (Sub-family name) are highly correlated), we study how label correlations can be inferred and facilitate truth inference. In our initial direction, we leverage the existing label correlation techniques~\cite{lc1,lc2} to generate the {\em pairwise label correlations} and regard them as prior input to our problem. %Pairwise correlation is computed on a pair of labels, whereas higher order label correlations is among a set of labels. 
For example, the conditional dependency of two labels defines the probability that one label is correct for an object under the condition that the other label is correct. Capturing the higher order label correlations requires computing the joint probability which could be computationally expensive. Once label correlation is computed, we shall explore how to use that information for improved truth inference. 
 \vspace{-0.1in}
\subsection{Open Problems}
\vspace{-0.1in}

\noindent {\bf Alternative Active Learning Strategy Design}
In our initial direction, we have discussed how to adapt uncertainty sampling to design active learning strategies for SVM classifier for multi-label scenario. The average number of correct labels assigned to each instance in a multi-label data set is called its label cardinality. Thus the number of predicted  labels of an unlabeled instance is expected to be consistent with the label cardinality computed on the labeled data. For an unlabeled instance, this inconsistency measure could be defined as the distance between the number of correctly predicted labels so far and the label cardinality of the labeled data. We will study this {\bf label cardinality inconsistency}~\cite{tsoumakas2006multi} to select that sub-task where the label inconsistency is highest.
Additionally, we will also study the active learning strategies known for other classifiers, such as Naive Bayes and Ensemble methods could be adapted to our problem~\cite{multi0,multi1,multi2,multi3}. Alternatively, task selection can be guided by a version space analysis such that it will give rise to maximum reduction in the version space of the classifier~\cite{versionspace}.

\noindent {\bf Truth Inference with Label Correlation} We will study how to use the information obtained from label correlation to improve the truth inference. Intuitively, our truth inference problem could benefit from label correlation in the following way: using Example~\ref{ex1}, if label correlation infers high correlation among two labels, let's say, Papilionidae and Swallowtail (family and sub-family of butterflies), it is likely that  Papilionidae and Mimic Sulfurs (which is a sub-family of butterflies, but Mimic Sulfurs belong to a different family (Pieridae) will have a very low correlation. Therefore, the probabilistic truth of the labels which have Mimic Sulfurs should be downgraded to reflect that fact. It has been shown in Information Retrieval that the more frequent two words occur together in text corpus, the more similar their vectors are~\cite{baeza1999modern}. We will regard each label as a word and compute the similarity (e.g., cosine similarity) between the vectors of two labels. We will explore widely popular Sigmoid function~\cite{ito1992approximation} to map a probability value to a real value, re-scale the value based on label correlation, and then revert the re-scaled correlation back to a probability score using the Sigmoid function again. 


%\noindent {\bf P3.3.3.3 : Relevant Label Sparsity}
%Traditional active learning techniques for multi-label learning do not take into account that the relevant labels are usually sparse~\cite{yang2009effective}. However, in the real world scenario, the average number of relevant labels per data point is small leading to relevant label sparsity. In this open problem, we intend to study how to design active learning strategies considering label sparsity. A related work~\cite{multi3} has developed an alternate inference technique  for  the  sparse  Bayesian  multi-label  graphical  model to carry out efficient mutual information based active learning. The authors have developed an approximation to the mutual information which is tightly coupled to their inference algorithm. This approximation is computed much more efficiently than the exact mutual information. We shall study the applicability of such techniques for our Bayesian model, as well as other classifiers, such as SVM and Ensemble Methods.


\vspace{-0.1in}
\section{Conclusion}
\vspace{-0.1in}
The goal of this article is to propose an {\em an optimized human-machine intelligence
framework for single and multi-label tasks through active learning}.  We conceptualize an iterative framework that judiciously employs human workers to collect single or multiple labels associated with such tasks, which, in turn are used by the supervised machine algorithms to make intelligent prediction. Our basic approach is adapt a few existing {\em active learning techniques for single and multi-label classification, but study them in the context of crowdsourcing, especially considering worker-centric optimization, i,e., human factors}. Our innovation lies in systematically characterizing variables to model human factors, designing optimization models that appropriately combine system and worker-centric goals, and designing effective solutions. 
\vspace{-0.1in}
\section{Acknowledgment}
\vspace{-0.1in}
The work of Senjuti Basu Roy is supported by the National Science
Foundation under Grant No.: 1814595 and Office of Naval Research
under Grant No.: N000141812838. 



\vspace{-0.2in}
\bibliographystyle{abbrv}
%\bibliography{BIB/mixed}
\bibliography{refs-latest,paperbib}



\end{document}


