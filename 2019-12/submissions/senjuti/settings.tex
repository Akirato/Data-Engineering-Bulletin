\vspace{-0.2in}
\section{Worker-Centric Optimization through Human Factors Modeling}\label{hf}
\vspace{-0.1in}
Recall Section~\ref{intro} and note that worker-centric optimization is a common theme across single and multi-labels tasks, which we first examine here.

{\bf Objectives:}
Our objective here is to explore mathematical models for worker-centric optimization in crowdsourcing platforms. Specifically, given an available pool of tasks and workers where workers perform repetitive tasks, we first obtain human factors of the workers by analyzing their past tasks and then study the problem of task assignment to enable worker-centric optimization. A recent work performs an ethnomethodological study at {\em Turker Nation}\footnote{\small http://www.turkernation.com/} and argues~\cite{martin2014being} that it is critical to enable worker-centric optimization.
Our effort here is to make a formal step towards that goal, independent of any specific system-centric optimization (i.e., the active learning principles). Therefore, such a study has a much broader applicability that goes beyond active learning. Of course, our framework will ultimately combine both system and worker-centric criteria. 

 {\bf Challenges:} While the significance of human factors is well-acknowledged in crowdsourcing, the challenge is to be able to estimate them effectively and propose appropriate models that could capture them during task assignment. Added to the fact is the dependence of the underlying crowdsourcing domain, which makes some of these factors more important than the rest (e.g., unlike AMT, there is no monetary pay-offs in citizen science activity, but skill variety is acknowledged to be critical to avoid monotony). 

%{\bf Our Prior Work:} In our recent investigation, we have studied some human-factors, such as, worker skill (accuracy), wage (pay-off), acceptance ratio of tasks, social affinity~\cite{hf1,roy2015task,DBLP:journals/corr/RahmanRTAD15} and proposed mathematical functions to include them explicitly in task assignment. We have designed medium-scale user studies~\cite{roy2015task,DBLP:journals/corr/RahmanRTAD15} in Amazon Mechanical Turk (AMT), involving hundreds of workers to learn the relationship between these factors and have observed that skill, wage, acceptance ratio - all follow normal distributions. We also have observed that a strong positive correlation exists between worker accuracy and expected wage. These studies have demonstrated empirically that task assignment considering human factors yields superior outcomes compared to the one without. In another recent work, we present how to combine task relevance (Boolean match of worker's expertise with the skill requirements of the tasks) and motivation in the task assignment process~\cite{edbt172}, albeit in a non-active learning context. Moreover, this work only considers {\em skill-variety} in formalizing motivation. In that sense, this latter work is still preliminary and does not solve the problem we propose. A recent survey paper summarizes different aspects of worker centricity in crowdsourcing platforms~\cite{amer2016toward}. 


\vspace{-0.1in}
\subsection{Proposed Directions}
\vspace{-0.1in}
First, we propose how to model and estimate human factors~\cite{hf1,motiv00,motiv0,motiv1,motiv2,motiv3,motiv4} that are pertinent to capture motivation using the variables that are described in Section~\ref{dm}. Then, we describe mathematical models that leverages these estimated human factors to explicitly assign tasks to workers. 

{\bf Estimating human factors:} We leverage the past task completion history of the workers as well as the new tasks to compute a Boolean task completion matrix $T$, where the rows are the workers and the columns are the (sub)-tasks. If a worker $u$ has completed a (sub)-task $t$ successfully in the past, the corresponding entry gets a $1$, it gets a $0$ otherwise. We assume that the factors that capture intrinsic motivation, i.e., skill variety, task identity, task significance, autonomy, feedback are independent yet latent variables. The second matrix we consider is the task factor matrix $\mathcal{F}$, where the rows are the tasks and the columns are the motivation related latent variables.  The final matrix is the user factor matrix $\mathcal{U}$ where rows are the factors and columns are the users. This matrix could be fully latent or observed.  In case it is latent, we minimize the error function, as defined below:

\begin{equation}
\sum_{i,j}(t_{ij}-\mathcal{U}_iF_j)^2+\lambda(\|\mathcal{U}\|^2+ \|\mathcal{F}\|^2)
\vspace{-0.1in}
\end{equation}

Here, $\lambda$ is the regularization parameter. The goal is to find ${\mathcal U}$ and ${\mathcal F}$ such that it minimizes the error. For any new worker and new task, the predicted task completion score is calculated by multiplying $U_i$ with $F_j$. Here, the important thing is to notice that the optimization function only minimizes the error for which ratings are present.  We apply the alternating least square approach~\cite{stigler1981gauss} to solve this problem. This is an iterative approach, where at each iteration, we fix the tasks' latent factor matrix $\mathcal{F}$ in order to solve for $\mathcal{U}$ and vice versa. We have designed a similar solution for predicting tasks to workers considering implicit workers' feedback\cite{DBLP:journals/corr/RahmanJR16}.

 
{\bf Worker-Centric Task Assignment:} The solution above only estimates the intrinsic motivational factors, but does not describe how to aggregate them together or combine with extrinsic motivation to perform worker-centric task assignment. 

Psychologists Hackman and Oldham~\cite{hackman1976motivation} have combined factors associated to intrinsic motivations defined {\em motivating potential score (MPS)} :

\begin{equation}\label{eqn}
\begin{aligned}
MPS= \frac{\text{skill-variety} + \text{task-identity} + \text{task-significance}}{3}  
 *\text{ autonomy } * \text{ feedback}
\end{aligned}
\end{equation}

Considering this aforementioned formulation, we study the worker-centric task assignment as a global optimization problem to maximize the {\em aggregated intrinsic and extrinsic motivation}. For a given set of tasks $S^{t_u}$,  $V(S^{t_u})$ represents the overall motivation for worker $u$, by combining  her extrinsic motivation  (EXTM) (recall Section~\ref{dm} that EXTM could be modeled using wage $w^t$) and intrinsic motivation, i.e, {\em motivating potential score (MPS)}(refer to Equation~\ref{eqn})~\cite{hackman1976motivation}. In our initial effort, we combine them linearly, as that allows us to design efficient algorithms. Assigning a set of tasks per worker is reasonable as well as desirable from worker's perspective, because workers in a typical crowdsourcing platform intend to undertake multiple tasks as opposed to a single task. Workers may also have constraints, such as,  not spend more than $X^u$ hours, 
or the aggregated wage must at least be $b^u$ dollars. 
 
Technically, we want to assign tasks to the workers {\em to maximize the aggregated motivation, such that the assignment satisfies each worker-specific constraints}. One such optimization function is described in Equation~\ref{eqn:eq3} (Recall Section~\ref{dm} where $time^t$ and $w^t$ are the duration and wage of sub-task $t$, respectively).

\begin{equation}\label{eqn:eq3}
 \text{ Maximize }  \sum_{u \in \mathcal{U}} [V(S^{t_u}) =  EXTM(S^{t_u}) + MPS(S^{t_u})]
\end{equation}
\vspace{-0.2in}
\begin{align*}
 V(S^{t_u}) =
\begin{cases} 
 \\ \text{ if } \sum_{t \in S^{t_u}} time^t \leq X^u \text{ and } \sum_{t \in S^{t_u}} w^t \geq b^u \\
0 \text{\qquad otherwise} 
\end{cases}
\vspace{-0.1in}
\end{align*} 


%Recall that a task $t$ is associated with a set $A$ of attributes, whose values, we assume are provided by the domain experts and are available at our disposal. Formally, for each task, we know the required skill domain $s^t$, duration $time^t$, significance $sig^t$, identity $iden^t$, autonomy $auto^t$, feedback $fb^t$, and wage $w^t$.

As a simple example, given two tasks $i$ and $j$, we can add the individual significance $sig^i+sig^j$, identity $iden^i+iden^j$, autonomy $auto^i+auto^j$, or feedback $fb^i+fb^j$. Similarly, the  wage of two tasks could also be added and normalized to compute EXTM. 
Alternative problem formulation is explored below.
%However, in order to compute MPS,  we still need to model and quantify the {\em skill variety} of a given set of tasks. 

%{\em Modeling Skill-variety:} We intend to study two different ways to capture skill variety. 

%(1) Set based modeling: Simple set based measures, such as, {\em Jaccard Distance}~\cite{baeza1999modern} can  be used to quantify {\em skill variety} between a given set of tasks. As an example, given $2$ tasks $t_1,t_2$ from citizen science, if $s^{t_1}=\text{humming bird}$, $s^{t_2}=owl$, skill variety $skill-variety(t_1,t_2)=1$. 

%In some application, it is reasonable to assume that an {\em intrinsic diversity ordering}~\cite{vee2008efficient} between the task attributes are available which can potentially be used to quantify skill-variety. For example, two sub-tasks associated with the same location that require to observe two different species may require more skill variety than two other sub-tasks that require observing the same species but from two different locations. We formalize this as, $\text{skill-domain} \prec \text{location}$. If such a diversity ordering is available from the domain experts, we model skill-variety considering diversity ordering~\cite{vee2008efficient}.

%(2) Order based modeling: An interesting alternative of the set based skill-variety computation  is to design an {\em ordering based}~\cite{DBLP:conf/icde/RoyDAY11,cao2012keyword} solution, where a {\em chain} of sub-tasks will be designed for a worker. Using Example~\ref{ex1}, this would mean creating an ordering $\mathcal{R}$ of the tasks for each worker where the skill-variety is ideally very high. Unlike set based approaches, here, skill-variety of a given set of sub-tasks will be computed considering each $k$ adjacent sub-tasks in the designed chain ($k=2$ means pair-wise adjacent tasks). As a simple example, if $3$ sub-tasks $t_2 \rightarrow t_1 \rightarrow t_3$ are designed for a worker, for $k=2$,  $skill-variety(t_2 \rightarrow t_1 \rightarrow t_3)= Jaccard(s^{t_2},s^{t_1})+ Jaccard(s^{t_1},s^{t_3})$.  %This modeling could also be extended when diversity based ordering between the attributes are available.


%The directions described above give rise a number of interesting computation problems.

%MPS formula requires us to quantify {\em skill variety} for a given set of tasks. We note that modeling skill variety leads to interesting research problems. Additionally, how to solve the objective function described above efficiently remains to be a challenging problem.
\vspace{-0.1in}
\subsection{Open Problems}
\vspace{-0.1in}
 {\bf  Solving the optimization problem:}
How to design an effective solution to maximize worker motivation based on the aforementioned objective function formulation is challenging. %The problem becomes even more complex when skill-variety is modeled as a chain. Even when set based modeling is considered, 
We observe that the proposed optimization problem is NP-hard~\cite{garey1979computers}, using a reduction from the assignment problems~\cite{roy2015task}. In a recent work, we have modeled motivation using {\em only skill-variety} and we have proved that the problem is NP-hard using a reduction from the Maximum Quadratic Assignment Problem~\cite{arkin2001approximating}. For our problem, we note that an integer programming based solution is simply not scalable. We will explore greedy heuristic strategies that are effective and efficient. For example, we will assign tasks to the workers greedily based on the marginal gain~\cite{roy2015task}. 
%if the set-based formulation admits {\em sub-modularity}~\cite{lovasz1983submodular}, as {\em skill-variety} exhibits {\em diminishing return properties}, and other motivational elements simply increase as more tasks are added. If this property is indeed satisfied, we will be able to design greedy approximation algorithms~\cite{nemhauser1978analysis} with theoretical guarantees. For the ordering based model, we will explore orienteering like solutions~\cite{chao1996fast} to design efficient greedy heuristic algorithms.

\noindent {\bf  Complex modeling for estimating intrinsic motivation \& task assignment:} In our preliminary direction, we have assumed that variables associated with intrinsic motivations are independent and could be combined as suggested by Hackman and Oldham~\cite{hackman1976motivation}, or intrinsic and extrinsic motivation could be combined linearly. In reality, that may not be the case. In this open problem, we will study the feasibility of a probabilistic model~\cite{zhao2012bayesian}, namely a {\em hierarchical Bayesian framework}~\cite{liu2014framework} for this problem. If the worker is completely new in the platform, we will bootstrap to collect a small set of evidence. We will consider each of the variables associated with worker motivation as a random variable and present a model using hierarchical Bayesian Networks~\cite{jensen1996introduction} by encoding a joint distribution of these variables over a multi-dimensional space.  This model will first establish the relationship among the intrinsic motivational variables themselves and then between intrinsic and extrinsic motivation to capture a workers' ``preference'' to a given task. We will apply Constraint Based, Score-Based, and Hybrid methods to learn the structure of the network~\cite{tsamardinos2006max}. We will leverage {\em Bayesian Parameter Estimation as well as Maximum Likelihood Estimation techniques} to learn the parameters of the constructed network. For efficient parameter estimation considering this complex joint distribution, we will use Gibbs sampling~\cite{carter1994gibbs}. 

%As outlined in the aforementioned open problem, if we are successful to model the motivational variables using a graphical model through probabilistic modeling, we shall extend that model to a hierarchical Bayesian Network with constraints that 

%In our preliminary work, we have assumed that variables associated with intrinsic motivation are combined as suggested by Hackman and Oldham. Moreover, we have aggregated intrinsic and extrinsic motivation linearly, as described in Equation~\ref{eqn:eq3}. In this open problem, we will explore alternative modeling. 





