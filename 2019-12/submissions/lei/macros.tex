\usepackage{color}
%\usepackage{fullpage} %!!!!! WARNING!!!! USE OF FULLPAGE PACKAGE LEADS TO COMPILATION ERROR WITH SIG-ALTERNATE
%\usepackage{mathptmx}
\usepackage{times}
\usepackage{subfig}
\usepackage{latexsym}
\usepackage{multirow}
\usepackage{mathtools}
\usepackage{epstopdf}

%\usepackage{amsmath, amssymb, amsfonts} %amsthm, }
\usepackage{graphics, graphicx, color}
\usepackage[table]{xcolor}
%\usepackage{algorithm, algorithmic} %, ifthen}

%\usepackage{booktabs}
%\usepackage{caption}
%\renewcommand{\thesection}{\arabic{section}}

\usepackage{epsfig}
%\usepackage{amssymb}
%, amsthm}
%\usepackage{amsmath}
%\usepackage{amsfonts}

\usepackage[ruled,linesnumbered]{algorithm2e}
\usepackage{hyperref}
\renewcommand{\algorithmcfname}{ALGORITHM}
\SetAlFnt{\small}
\SetAlCapFnt{\small}
\SetAlCapNameFnt{\small}
\SetAlCapHSkip{0pt}
\IncMargin{-\parindent}

%%
%\newtheorem{observation}{Observation}[section]
%\newtheorem{definition}{Definition}[section]
%\newtheorem{theorem}{Theorem}
%\newtheorem{lemma}{Lemma}[section]
%\newtheorem{claim}{Claim}[section]
%\newtheorem{example}{Example}[section]
%%\newtheorem{fact}{Fact}
%%\newtheorem{invariant}{Invariant}
%%\newtheorem{property}{Property}
%\newtheorem{corollary}{Corollary}
%\newtheorem{proposition}{Proposition}[section]

%\providecommand{\algorithmautorefname}{Algorithm}

\newcommand{\child}{{\tt\textsc{Child}}}
\newcommand{\In}{{\tt\textsc{In}}}
\newcommand{\Out}{{\tt\textsc{Out}}}

\newcommand{\suc}{{\tt\textsc{succ}}}
\newcommand{\pred}{{\tt\textsc{pred}}}

\newcommand{\attr}{{\tt{Att}}}

\newcommand{\oracle}{{\cal O}}



%
%\newcommand{\reva}[1]{#1}%{{\color{blue}{#1}{\typeout{#1}}}}
%\newcommand{\revb}[1]{#1}%{{\color{red}{#1}{\typeout{#1}}}}
%\newcommand{\revc}[1]{#1}%{{\color{magenta}{#1}{\typeout{#1}}}}
%
%%THE FOLLOWING IS TO WRITE THE REVIEWRS'S COMMENT, TO BE REPLACED BY {} LATER
%
%\newcommand{\revacomm}[1]{#1}%{{\color{blue}{\bf(#1)}{\typeout{#1}}}} %{}
%\newcommand{\revbcomm}[1]{#1}%{{\color{red}{\bf(#1)}{\typeout{#1}}}} % {}
%\newcommand{\revccomm}[1]{#1}%{{\color{magenta}{\bf(#1)}{\typeout{#1}}}} %{}

\newcommand{\reva}[1]{{\color{red} {#1}}}
\newcommand{\revb}[1]{{\color{blue} {#1}}}
\newcommand{\revc}[1]{{\color{magenta} {#1}}}

\newcommand{\scream}[1]{} %{{\color{purple}{\texttt{\textbf{ *#1*}}}{\typeout{#1}}}}
\newcommand{\blue}[1]{{\color{blue}{\bf * #1 *}{\typeout{#1}}}}
\newcommand{\red}[1]{{\color{red}{\texttt{\textbf{*#1*}}}{\typeout{#1}}}}
\newcommand{\ans}[1]{{\it ? #1 ?}{\typeout{#1}}}
\newcommand{\eat}[1]{}
\newcommand{\cut}[1]{}
\newcommand{\neweat}[1]{}
\newcommand{\eg}{\emph{e.g.}}
\newcommand{\ie}{\emph{i.e.}}
\newcommand{\figref}[1]{Fig.~\ref{#1}}
\newcommand{\angb}[1]{\langle #1 \rangle}
\newcommand{\comple}[1]{\widebar{#1}}


\newcommand{\photodb}{{\tt PhotoDB}} %{{\tt IndividualPhotoDB}}
\newcommand{\myphotodb}{{\tt MyPhotoDB}}
\newcommand{\cost}{{\tt cost}}
\newcommand{\pvt}{{\tt c}}

\newtheorem{observation}{Observation}
%\newtheorem{definition}{Definition}
%\newtheorem{theorem}{Theorem}
%\newtheorem{lemma}{Lemma}
%\newtheorem{claim}{Claim}
%\newtheorem{example}{Example}
%\newtheorem{fact}{Fact}
%\newtheorem{invariant}{Invariant}
%\newtheorem{property}{Property}
%\newtheorem{corollary}{Corollary}
%
%\newcommand{\ComputeSeparate}{\textsc{ComputeSeparate}}
\newcommand{\var}{\textsc{Var}}
\newcommand{\be}{\begin{enumerate}}
\newcommand{\ee}{\end{enumerate}}
\newcommand{\Algo}{\textsc{Algo}}
\newcommand{\No}{\textsc{No}}
\newcommand{\Yes}{\textsc{Yes}}
\newcommand{\Dom}{\texttt{Dom}}
\newcommand{\Codomain}{\texttt{CoDom}}
%\newcommand{\tup}[1]{\mathbf{#1}}
\newcommand{\tup}[1]{\vec{#1}}
\newcommand{\proj}[2]{\Pi_{{#1}}({#2})}

\newcommand{\true}{\textsc{True}}
\newcommand{\false}{\textsc{False}}

\newcommand{\Rows}{\texttt{Rows}}
\newcommand{\inset}{\texttt{P}}
\newcommand{\outset}{\texttt{Q}}

% For questions and answers.

\newcommand{\Tova}[1]{{\blue *(Tova): #1 *}}
\newcommand{\Benoit}[1]{{\purple *(Benoit): #1 *}}
\newcommand{\sudeepa}[1]{\red{*(Sudeepa): #1 *}}
%
%\newcommand{\scream}[1]{{\bf * #1 *}{\typeout{#1}}}
%\newcommand{\ans}[1]{{\it ? #1 ?}{\typeout{#1}}}
%\newcommand{\angb}[1]{{\langle #1 \rangle}{\typeout{#1}}}
%\newcommand{\eat}[1]{}
%\newcommand{\eg}{\emph{e.g.}}
%\newcommand{\ie}{\emph{i.e.}}


\newcommand{\wrt}{{with respect to~}}

\newcommand{\val}{{\texttt{val}}}
\newcommand{\type}{{\texttt{type}}}
\newcommand{\link}{{\texttt{link}}}
\newcommand{\for}{{\texttt{for}}}
\newcommand{\while}{{\texttt{while}}}
\newcommand{\repeatloop}{{\texttt{repeat\_loop}}}
\newcommand{\ismatch}{{\texttt{is\_match}}}


\newcommand{\select}{{\texttt{SELECT~}}}
\newcommand{\where}{{\texttt{WHERE~}}}
\newcommand{\from}{{\texttt{FROM~}}}
\newcommand{\groupby}{{\texttt{GROUP~BY~}}}
\newcommand{\orderby}{{\texttt{ORDER~BY~}}}
\newcommand{\desc}{{\texttt{DESC~}}}

\newcounter{countposid}
\setcounter{countposid}{0} 
\newcommand{\posid}{\addtocounter{countposid}{1}%
	\color{black!60}\small$\!_{\mathbf{\alph{countposid}}}$}