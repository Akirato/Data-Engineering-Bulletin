\documentclass[11pt]{article} 

\usepackage{deauthor,times,graphicx}
%\usepackage{url}
\usepackage{hyperref}



\begin{document}

The big data revolution and advancements in machine learning
technologies have revolutionized decision making, advertising,
medicine, and even election campaigns. Data-driven software now
permeates virtually every aspect of human activity and has the
ability to shape human behavior: it affects the products we view
and purchase, the news articles we read, the social interactions we
engage in, and, ultimately, the opinions we form. Yet, data is an
imperfect medium, tainted by errors, omissions, and biases. As a
result, discrimination shows up in many data-driven applications,
such as advertisements, hotel bookings, image search, and vendor
services. In this issue, we bring together an exciting collection
of recent and ongoing work that focuses on the problems of
fairness, diversity, and transparency in data-driven systems. This
collection highlights the central role that the data management
research community can play in detecting, informing, and mitigating
the effects of bias, skew, and misuse of data, and aims to create
bridges with work in related communities.

We start with ``Nutritional Labels for Data and Models'', by
Stoyanovich and Howe. This paper argues for informational and
warning labels for data, akeen to nutritional labels, that specify
characteristics of data and how it should be consumed. These
nutritional labels help humans determine the fitness of models and
data, aiding the interpretability and transparency of
decision-making processes.

The second paper, ``Data Management for Causal Algorithmic
Fairness'', by Salimi, Howe, and Suciu, provides a brief overview
of fairness definitions in the literature, and argues for the use
of causal reasoning in defining and reasoning about fairness. The
paper exposes a vision of the opportunities of applying data
management techniques, such as integrity constraints, query
rewriting, and database repair to enforcing fairness, detecting
discrimination, and explaining bias.

In the third paper, ``A Declarative Approach to Fairness in
Relational Domains'', Farnadi, Babaki, and Getoor focus on notions
of fairness that capture the relational structure of a domain, and
propose a general framework for relational fairness. Fairness-aware
probabilistic soft logic includes a language for specifying
discrimination patterns, and an algorithm for performing inference
under fairness constraints.

The next paper, ``Fairness in Practice: A Survey on Equity in Urban
Mobility'', by Yan and Howe, places its focus on practical societal
implications of fairness in the domain of transportation. The paper
presents the findings of equity studies in mobility systems, such
as bike-sharing and ride-hailing systems, and reviews experimental
methods and metrics.

Again motivated by the societal implications of fairness and
diversity, Benabbou, Chakraborty, and Zick put their sights on the
allocation of public resources. ``Fairness and Diversity in Public
Resource Allocation Problems'' focuses on two real-world cases, the
allocation of public housing in Singapore and public school
admissions in Chicago, models them as constrained optimization
problems, and analyzes the welfare loss in enforcing diversity.

We conclude with ``Towards Responsible Data-driven Decision Making
in Score-Based Systems'', by Asudeh, Jagadish, and Stoyanovich. The
paper focuses on designing fair and stable rankings, and discusses
how these technologies can assess and enhance the coverage of
training sets in machine learning tasks.

Thank you to all the authors for their insightful contributions,
which bring into focus new and exciting challenges, and identify
opportunities for data management research to contribute tools and
solutions towards critical societal issues. Thank you also to
Haixun Wang for his valuable assistance in putting together the
issue. I hope you enjoy this collection.

\end{document}


