\begin{abstract}
    
An essential ingredient of successful machine-assisted decision-making, particularly in high-stakes decisions, is interpretability --– allowing humans to understand, trust and, if necessary, contest, the computational process and its outcomes.   These decision-making processes are typically complex:  carried out in multiple steps, employing models with many hidden assumptions, and relying on datasets that are often used outside of the original context for which they were intended.   In response, humans need to be able to determine the ``fitness for use'' of a given model or dataset, and to assess the methodology that was used to produce it.  
    
To address this need, we propose to develop interpretability and transparency tools based on the concept of a nutritional label, drawing an analogy to the food industry, where simple, standard labels convey information about the ingredients and production processes. Nutritional labels are derived automatically or semi-automatically as part of the complex process that gave rise to the data or model they describe, embodying the paradigm of interpretability-by-design. In this paper we further motivate nutritional labels, describe our instantiation of this paradigm for algorithmic rankers, and give a vision for developing nutritional labels that are appropriate for different contexts and stakeholders.
    
\end{abstract}