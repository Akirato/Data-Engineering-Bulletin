\documentclass[11pt]{article} 

\usepackage{deauthor,times,graphicx}
%\usepackage{url}
\usepackage{hyperref}

\begin{document}

\section*{Icing on the Cake}

I have had the honor and pleasure of serving for 25+ years and over 100 issues as the Editor-in-Chief (EIC) of the Data Engineering Bulletin, the very publication in which this letter is being published.  I never dreamed, while pondering the Bulletin EIC offer from Rakesh Agrawal, then the TCDE chair in 1992, that I would make the Bulletin so significant a part of my career.  To now get rewarded with the TCDE Service Award is truly ``icing on the cake''.  I am thankful to the TCDE both for the opportunity to serve as Bulletin EIC and now for being honored for this service with this award.

The Bulletin has been such a large part of my technical career and my primary service activity until just recently, when I have become 
involved with Computer Society governance.  And the beauty of how this all worked out is that the Bulletin has truly been a ``labor of love''.  Where else can database professionals learn what is happening in a subarea of our field, brought together in a single issue, with contributions from research and industrial leaders.  

In the database area, which changes so fast, the ability of the Bulletin to provide a special issue on a new topic is both unique and invaluable.  The ability of Bulletin editors to bring leading technologists together to write articles for an issue is the ``magic sauce'' that makes the entire enterprise a success.  Over the years, it has been my pleasure to work with so many of the gifted editors whose work you see in every issue published.  I like to think that I also contributed to the success of the Bulletin-- but my success was one level indirect.  It was my success over the years of convincing distinguished members of the database community to serve as Bulletin editors.  As one mark of this success, the editors I have appointed include seven Codd Award winners, all but one prior to their receiving the award.  And I have no doubt there will be more winners in the future.

The Bulletin would not exist without articles written by so many distingushed members of our database community.  Their willingness to contribute articles is a direct result of you, our readers, who so eagerly consume Bulletin articles.  The result of this is a virtuous cycle: distinguished editors attract distinguished authors, who write articles that are read and cited by many members of our database community.  So you, dear reader, have played an essential role in making this system work. 

Over the years, the Bulletin has transformed from solely paper publication to a mixed paper-electronic publication to finally an entirely electronic publication.  Over that time, my job at Digital Equipment Corp. (DEC) transformed into a job at Microsoft.  My thanks to both employers, who so generously permitted me to spend time on the Bulletin for so many years, and who provided the initial web infrastructure that made the Bulletin available electronically.

Haixun Wang, my successor and current Bulletin EIC, now has three issues "under his belt".  So the future of the Bulletin looks very promising.  He has recently introduced an "opinion" section, and asked me to contribute an opinion piece in the first issue with the new section.  This was my first non-letter Bulletin publication since 1987 (before I became EIC).  I am hoping it is not the last as, like so many others in our community, I value the Bulletin as a channel for publishing my technical contributions.
 
And now, finally, I too have the pleasure of reading Bulletin articles-- focusing on their technical content, rather than being concerned (and consumed) by formatting and editorial issues.  I have already begun enjoying this post-EIC role, and look forward to this continuing.  Thank you all for contributing to the success of the Bulletin and for making my involvement so personally gratifying.

\end{document}


