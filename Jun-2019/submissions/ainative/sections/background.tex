
%!TEX root = ../main.tex
\vspace{-.5em}
\section{Background}
\label{sec:background}
\vspace{-.25em}
Reviewing the development history of database in the past fifty years~\cite{Ramakrishnan:2002:DMS:560733}, it has undergone three generations.
%ref: History of Database Management Systems 

\subsection{First Generation: Standalone Database}
The first generation, stand-alone database, comes to solve the problems of data storage, data management and query processing~\cite{DBLP:books/daglib/0006734}. In traditional file system, 1) data is directly stored in a set of raw files, which have no relations with each other (flat file); 2) for each search operation, a different search application has to be written; 3) and the query efficiency is low. First generation databases centralize data management under the supervision of the data experts and provide uniform interface to data. 
Firstly, database stores data in related form (e.g., hierarchical type, relational model and graph). And data is indexed to further enhance query efficiency. Secondly, with the storage engine, database can manage how data is stored in memory or on disk. Thirdly, database provides inbuilt searching operations (e.g., seq scan and index scan). User can access data via unified query statements (e.g., QUEL and SQL). The first generation database is mainly for single-machine applications and examples include Postgre95 and the early IBM relational database systems. 

\subsection{Second Generation: Cluster Database}
The second generation, cluster database, comes to ensure high availability and reliability in fault tolerant environments. The standalone database fails to solve those problems for two reasons. Firstly, once the computer blows up, the database is not available. Secondly, the resource of one computer (e.g., CPU, RAM and disk space) is limited. If there is a huge spike in traffic, it is not going to get all the hits. Database clustering technique is to use multiple machines together to store the same data (data redundancy). Firstly, it can achieve high availability. Even if one node crashes, we still can access all the data from the other nodes. Secondly, it can achieve high reliability. With proper synchronization mechanism, it ensures data consistency on each node while avoiding data errors or loss. Typical database clustering techniques include Oracle RAC, MongoDB Replication, Master Slave Replication in MySQL and etc.

\subsection{Third Generation: Distributed Cloud Database}
The third generation of database is distributed cloud database. Firstly, with distributed processing, it can further distribute the workload among working nodes, compared with cluster database~\cite{DBLP:journals/jidm/FigueiredoBM10}. Because it divides the database into fragments. Each working node (except nodes for replication) stores different fragments and can process different query  operations parallely. And with distributed architecture, it can easily realize elastic expansion. Secondly, non-cloud-based databases require enterprises to provide all the infrastructure and resources needed to build and manage the database instances. While, with Database as a Service (DBaaS), cloud database allows those companies to free up personnel and focus on important tasks.
%databases in cloud computing: a literature review
Besides, cloud database also makes it easy to scale up or down their databases, since it can be implemented by cloud device vendors with virtual technology. 
So cloud platforms naturally support distributed databases. 
% Cloud Databases: Future of Distributed Databases
And the distributed cloud database has come to better realize elastic expansion and provide fast and stable services. Products include GCP, AWS RDS, Microsoft Azure and etc.

