\newcommand{\parai}[1]{\noindent\textit{#1}}

\section{Background}
\label{sec:background}

In this section, we recall the \SecAgg protocol first, then the compression methods that we wish to adapt to \SecAgg, namely, scalar quantization, pruning, and product quantization.

\subsection{Secure Aggregation}
\label{subsec:secagg}

\SecAgg refers to a class of protocols that allow the server to aggregate client updates without accessing them individually. While \SecAgg alone does not entirely prevent client data leakage, it is a powerful and widely-used component of current at-scale cross-device FL implementations~\cite{kairouz2019advances}. Two main approaches exist in practice: software-based protocols relying on Multiparty Computation (MPC)~\cite{bonavitz2019federated,bell2020secure,LightSecAgg}, and those that leverage hardware implementations of Trusted Execution Environments (TEEs)~\cite{huba2021papaya}.
% While these approaches have substantial differences, they impose similar constraints on compatible update compression techniques: operating over finite fields and assuming that aggregation commutes with decompression.


\SecAgg relies on additive masking, where clients protect their model updates $g_i$ by adding a uniform random mask $m_i$ to it, guaranteeing that each client’s masked update is statistically indistinguishable from any other value.
% Masks are generated so that when all the masked client updates are aggregated by the server (typically through addition), the server obtains the exact aggregate (e.g., the sum of updates).
At aggregation time, the protocol ensures that all the masks are canceled out. For instance, in an MPC-based \SecAgg, the pairwise masks cancel out within the aggregation itself, since for every pair of users $i$ and $j$, after they agree on a matched pair of input perturbations, the masks $m_{i,j}$ and $m_{j,i}$ are constructed so that $m_{i,j}=-m_{j,i}$.
Similarly and as illustrated in Fig.~\ref{fig:secagg_summary}, in a TEE-based \SecAgg, the server receives $h_i = g_i + m_i$ from each client as well as the sum of the masks $\sum_i m_i$ from the TEE and recovers the sum of the updates as
% by unmasking the aggregated updates as
%\begin{equation*}
$
      \sum_i g_i = \sum_i h_i - \sum_i m_i.
$
%\end{equation*}
We defer the discussion of DP noise addition by \SecAgg protocols to Section~\ref{sec:discussion}.

\para{Finite Group.}
\SecAgg requires that the plaintexts---client model updates---be elements of a finite group, while the inputs are real-valued vectors represented with floating-point types.
This requirement is usually addressed by converting client updates to fixed-point integers and operating in a finite domain (modulo~$2^p$) where $p$ is typically set in prior literature to 32 bits. The choice of \SecAgg bit-width~$p$ must balance communication costs with the accuracy loss due to rounding and overflows.

% The other constraint is due to the fact that SecAgg is designed so that the server sees only the aggregate (the sum or the weighted sum) of individual gradients, plus some noise injected by a differentially private mechanism. A drop-in decompression operator $D$ must commute with SecAgg, or be close to being one:
% \[
% D\left(\sum_i g_i+\mathrm{noise}\right) \approx \sum_i D(g_i)+\mathrm{noise}.
% \]
\para{Minimal Complexity.}
\looseness=-1
TEE-based protocols offer greater flexibility in how individual client updates can be processed; however, the code executed inside TEE is part of the trusted computing base (TCB) for all clients. In particular, it means that this code must be stable, auditable, defects- and side-channel-free, which severely limits its complexity. Hence, in practice, we prefer compression techniques that are either oblivious to \SecAgg's implementation or require minimal changes to the TCB.

% In the remainder of the paper, we focus on the TEE-based approach for its simplicity, scalability and compatibility with asynchronous FL. relies on random masking to encrypt the local model updates. The server then incrementally aggregates the encrypted updates and unmasks the aggregate using the sum of the random masks transmitted by the Trusted Secure Aggregator (TSA) that sits within the TEE. More formally, let us denote $??$ a local client update, represented in floating-point precision, generally \texttt{fp32}. First, the client converts $??$ to a fixed point representation. Then, the client generates a random mask $m \in \Z^d$ and computes the sum modulo a number. \ashkan{TODO}

% SecAgg protocols prevent the server from accessing individual client updates by aggregating them and transmitting only the aggregate to the server. While such mechanisms alone do not entirely prevent data leakage, they constitute a vital component of cross-device FL implementations~\cite{kairouz2019advances}. SecAgg relies either on Secure Multiparty Computation \cite{bonavitz2019federated,so2021secure} or on a Trusted Executed Environment or TEE~\cite{huba2021papaya}. In the remainder of the paper, we focus on the TEE-based approach for its simplicity, scalability and compatibility with asynchronous FL.

% TEE-based SecAgg relies on random masking to encrypt the local model updates. The server then incrementally aggregates the encrypted updates and unmasks the aggregate using the sum of the random masks transmitted by the Trusted Secure Aggregator (TSA) that sits within the TEE. More formally, let us denote $??$ a local client update, represented in floating-point precision, generally \texttt{fp32}. First, the client converts $??$ to a fixed point representation. Then, the client generates a random mask $m \in \Z^d$ and computes the sum modulo a number.

% Since the server, by design, never observes

\subsection{Compression Methods}
\label{subsec:comp_methods}
In this subsection, we consider a matrix $W \in \mathbb{R}^{\cin\times \cout}$ representing the weights of a linear layer to discuss three major compression methods with distinct compression/accuracy tradeoffs and identify the challenges \SecAgg faces to be readily amenable to these popular quantization algorithms.

\subsubsection{Scalar Quantization}
\label{subsec:sq}

\looseness=-1 Uniform scalar quantization maps floating-point weight $w$ to $2^b$ evenly spaced bins, where $b$ is the number of bits. Given a floating-point scale $s > 0$ and an integer shift parameter $z$ called the zero-point, we map any floating-point parameter $w$ to its nearest bin indexed by $\{0,\dots, 2^b-1\}$:

\centerline{$w \mapsto \clamp(\round(w /s) + z, [0, 2^b - 1] ).$}

%
The tuple $(s, z)$ is often referred to as the quantization parameters (\texttt{qparams}).
With $b=8$, we recover the popular \texttt{int8} quantization scheme \cite{jacob2017quantization}, while setting $b = 1$ yields the extreme case of binarization \cite{courbariaux2015binaryconnect}.
The quantization parameters $s$ and $z$ are usually calibrated after training a model with floating-point weights using the minimum and maximum values of each layer.
% The accuracy drop due to this post-training quantization can be mitigated by pre-conditioning the network during training with techniques such as Quantization-Aware Training or QAT~\cite{krishnamoorthi2018quantizing}.
% The quantization parameters can also be defined per-channel instead of per-layer to diminish the quantization error at the cost of a small memory overhead.
The compressed representation of weights $W$ consists of the \texttt{qparams} and the integer representation matrix $W_q$ where each entry is stored in~$b$~bits.
Decompressing any integer entry $w_q$ of~$W_q$ back to floating point is performed by applying  the (linear) operator $w_q \mapsto s\times(w_q - z)$.

\para{Challenge.}
The discrete domain of quantized values and the finite group required by \SecAgg are not natively compatible because of the overflows that may occur at aggregation time. For instance, consider the extreme case of binary quantization, where each value is replaced by a bit.
We can represent these bits in \SecAgg with $p=1$, but the aggregation will inevitably result in overflows.

\subsubsection{Pruning}
\label{subsec:rp}

Pruning is a class of methods that remove parts of a model such as connections or neurons according to some pruning criterion, such as weight magnitude~(\cite{lecun1990optimal,hassabi1992second}; see \cite{Blalock20} for a survey). \cite{konen2016federated} demonstrate client update compression with random sparsity for federated learning. Motivated by previous work and the fact that random masks do not leak information about the data on client devices, we will leverage random pruning of client updates in the remainder of this paper.
% as it is easiest to combine with SecAgg.
% Let $\texttt{rand}$ be a function generating random entries in interval $[0, 1)$.
% For a sparsity level $0\leq\rho\leq 1$, where $\rho=1$ yields a zero matrix, a client prunes entries~$w$ of~$W$ as:
% % \karthik{TODO: decide on notation for rand}
% \[w \mapsto \begin{cases}
% 0 & \text{if } \texttt{rand()} < \rho \\
% w & \text{otherwise}.
% \end{cases}
% \]
% \sayan{should we number the equations ?}
A standard method to store a sparse matrix is the coordinate list (COO) format\footnote{See the  {torch.sparse documentation}, \url{https://pytorch.org/docs/stable/sparse.html}.}, where only the non-zero entries are stored (in floating point or lower precision), along with their integer coordinates in the matrix.
This format is compact, but only for a large enough compression ratio, as we store additional values for each non-zero entry.
Decompression is performed by re-instantiating the uncompressed matrix with both sparse and non-sparse entries.

\para{Challenge.}
\modif{Pruning model updates on the client side is an effective compression approach} as investigated in previous work. However, the underlying assumption is that clients have different masks, either due to their seeds or dependency on client update parameters (\eg weight magnitudes). This is a challenge for \SecAgg as aggregation assumes a dense compressed tensor, which is not possible to construct when the coordinates of non-zero entries are not the same for all clients.

\subsubsection{Product Quantization}
\label{subsec:pq}


Product quantization (PQ) is a compression technique developed for nearest-neighbor search \cite{jegou2011product} that can be applied for model compression \cite{stock2019bit}.
Here, we show how we can re-formulate PQ to represent model updates.
We focus on linear layers and refer the reader to~\cite{stock2019bit} for adaptation to convolutions.
Let the \emph{block size} be $d$ (say, 8), the number of \emph{codewords} be $k$ (say, 256) and assume that the number of input channels, $\cin$, is a multiple of $d$.
To compress $W$ with PQ, we evenly split its columns into subvectors or blocks of size $d \times 1$ and learn a \emph{codebook} via $k$-means to select the $k$ codewords used to represent the $\cin\times\cout/d$ blocks of $W$. PQ with block size $d=1$ amounts to non-uniform scalar quantization with $\log_2 k$ bits per weight.
% More formally, we first reshape $ W$ into a matrix of size $d \times \cin \cout / d$ and with a slight abuse of notation, we will also use  $W$ to denote the reshaped matrix and work only in the reshaped space.
% Note that the reshaping approach applies to convolutional weights as well: e.g., for a 2D convolution with a kernel of size of $k_s$, we need to change the reshaping part to get matrix of size $d \times \cin\cout k_s^2/d$.

The PQ-compressed matrix $W$ is represented with the tuple $(C, A)$, where $C$ is the codebook of size $k \times d$ and $A$ gives the assignments of size $\cin \times\cout / d$.
% \begin{align*}
% C & \text{    the codebook of size } k \times d,  \\
% A & \text{    the assignments of size }\cin \cdot\cout / d.
% \end{align*}
Assignments are integers in $[0, k-1]$ and denote which codebook a subvector was assigned~to.
To decompress the matrix (up to reshaping), we index the codebook with the assignments, written in PyTorch-like notation as
%\begin{equation*}
$
    \widehat {W} = C[A].
$
%\end{equation*}
% (appropriating a PyTorch-like notation) and perform a reshaping operation. PQ is naturally extensible to convolutional layers~\cite{stock2019bit}.

\para{Challenge.}
There are several obstacles to making PQ compatible with \SecAgg.
First, each client may have a different codebook, and direct access to these codebooks is needed to decode each client's message.
Even if all clients share a (public) codebook, the operation to take assignments to produce an (aggregated) update is not linear, and so cannot be directly wrapped inside \SecAgg.
%In theory PQ is a linear operation, since we can encode each client's choice of codeword for a block with a 1-hot vector of length $k$, and com

