\section{Related Work}



Truth discovery is a traditional research direction as we may often receive diverse and even conflicting information about one event \citep{li2016survey}. In the pioneering research \citep{yin2008truth}, authors discuss the truth discovery problem when there are many conflicting facts about one subject on different websites.
Besides information from websites, truth discovery is also important in many other areas such as social sensing \citep{wang2012truth} and crowdsourcing \citep{li2014the,Whitehill2009WhoseVS}.

Mobile crowdsensing \citep{zhang20144w1h}, as a particular type of crowdsourcing that needs workers to do location-based sensing tasks, would also face the truth discovery problem \citep{Wang2014SurrogateMS}. Meanwhile, privacy protection is also an important issue to consider in crowdsensing, especially for location privacy \citep{han2021hidden,wang2016differential,Wang2017LocationPT,Wang2019PersonalizedPT,Wang2020SparseMC,Wang2019MobileCT}. Most prior research focuses on protecting crowdsensing participants' location privacy in task allocation \citep{Wang2017LocationPT,Wang2019PersonalizedPT,Wang2019MobileCT} or for particular crowdsensing tasks such as missing data inference \citep{wang2016differential,Wang2020SparseMC}. 

Recently, some studies investigate the privacy-preserving truth discovery in crowdsensing \citep{Miao2015CloudEnabledPT,Miao2017ALP,Miao2019PrivacyPreservingTD,Zhang2021ReliableAP,ZHANG2020101848,Xu2019EfficientAP}. One research direction is applying data perturbation methods such as differential privacy to participants' sensed data \citep{Li2018AnET,Li2020TowardsDP}, but these methods degrade the truth finding accuracy. Another research direction follows the federated learning \citep{yang2019federated} paradigm that participants' raw data will not be directly sent to the server with certain encryption techniques, while the aggregation results (i.e., detected truths) can be accurately learned. However, the existing privacy-preserving truth discovery methods usually suffer from certain assumptions which may not stand in reality, e.g., online/non-concluding participants \citep{Miao2015CloudEnabledPT,Miao2017ALP,Miao2019PrivacyPreservingTD}, or third-party non-concluding servers \citep{Zhang2021ReliableAP,ZHANG2020101848}. 
Moreover, no prior work considers hiding participants' completed tasks or tracking participants' trustworthiness in a privacy-preserving manner, which has been addressed by our work.

\begin{table*}[t]
	\footnotesize
	\centering
	\caption{Comparison of our work and representative related work. (NTP: No Third Party, AT: Assess Trustworthiness,  CA: Collusion Attacks)}
	\label{tab:related_work}
	\begin{tabular}{@{}lccccccc@{}}
		\toprule
		\textbf{} &
		\multicolumn{2}{c}{\textbf{Privacy Protection}}
		&
		\textbf{NTP} &
		\textbf{AT} &
		\multicolumn{2}{c}{\textbf{Connection Loss}} & \textbf{CA} \\
		&
		\textit{Sensed Data} &
		\textit{Completed Tasks}
		&
		&
		& \textit{Fault Tolerance} & \textit{Bias Avoidance}
		& \\
		\midrule
		\citet{Miao2017ALP} & $\surd$  & $\times$ & $\times$ & $\times$  & $\times$ & $\times$ & $\times$ \\
		\citet{Zheng2018LearningTT} & $\surd$  & $\times$ & $\times$ & $\times$  & $\surd$ & $\times$ & $\times$ \\
		\citet{Miao2019PrivacyPreservingTD}  & $\surd$  & $\times$ & $\surd$ & $\times$  & $\surd$ & $\times$  & $\times$ \\
		\citet{Xu2019EfficientAP}  & $\surd$ & $\times$ & $\surd$ & $\times$ & $\surd$ & $\times$ & $\surd$ \\
		\citet{Zheng2020PrivacyAwareAE} & $\surd$ & $\times$ & $\surd$ & $\times$ & $\surd$ & $\times$ & $\surd$ \\
		\citet{Zhang2021ReliableAP}  & $\surd$ & $\times$ & $\times$ & $\times$ & $
		\surd$ & $\times$ & $\times$\\
		Our Work & $\surd$ & $\surd$ & $\surd$ & $\surd$ & $\surd$ & $\surd$ & $\surd$ \\ \bottomrule
	\end{tabular}
\end{table*}


Table \ref{tab:related_work} summarizes the characteristics of our work and representative related work published in top venues recently. In particular, our work is the \textbf{first} privacy-preserving crowdsensing truth discovery research that considers (i) \textit{providing a feasible solution to participant trustworthiness assessment} when the trustworthiness scores are not revealed, and (ii) \textit{hiding participants' completed tasks} to provide stronger privacy protection. Moreover, when dealing with the connection loss during the iterative crowdsensing truth discovery process, our work (i) proposes an adaptive event confidence updating function to reserve the data contributions of drop-out participants to avoid the truth bias toward alive participants, and (ii) designs an SSS-based scheme to defend against participants' collusion attacks while ensuring the high communication efficiency.

