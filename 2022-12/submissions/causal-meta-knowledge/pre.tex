\subsection{Definitions and Notations}

In this paper, we follow the definition of knowledge graph as in~\cite{ji2021survey}:
\begin{definitionnew}
A \textbf{Knowledge Graph (KG)} is defined as $\mathcal{G}=(\mathcal{E},\mathcal{R},\mathcal{F})$, where
$\mathcal{E}$, $\mathcal{R}$ and $\mathcal{F}$ are sets of entities, relations and facts, respectively.
Every fact is a triple $(e_h,R,e_t) \in \mathcal{F}$, where $e_h, e_t\in \mathcal{E}$ and $R \in \mathcal{R}$ are head entity, tail entity and the relation between entities, respectively. Without loss of generality, we simultaneously represent a fact as $R(e_h,e_t)$.
\end{definitionnew}

%Here we introduce a similar data structure with KG: heterogeneous information graph, following the definition in ~\cite{sun2011pathsim}:

%\begin{definition}
%A \textbf{heterogeneous information network (HIN)} \cite{sun2011pathsim}is a directed graph $G=(V, E, \Phi, \Psi)$, where $V$ is the set of nodes (or entities) of the graph; $E \subseteq V \times V$ is the set of edges connecting the nodes in $V$; and $\Phi$ and $\Psi$ are functions for labeling nodes and edges. We have $\Phi: V \rightarrow \mathcal{A}$, where $\mathcal{A}$ is the set of node classes, and $\Psi: E \rightarrow \mathcal{B}$, where $\mathcal{B}$ is the set of edge types.
%\end{definition}

%Given a complex heterogeneous information network, it is necessary to provide its meta level (i.e., schema-level), which is called network schema.

%\begin{definition}
%A \textbf{network schema}\cite{sun2011pathsim} is a meta template for a HIN $G=(V, E)$ with the object type mapping $\phi: V \rightarrow \mathcal{A}$ and the link mapping $\psi: E \rightarrow \mathcal{R}$, which is a directed graph defined over object types $\mathcal{A}$, with edges as relations from $\mathcal{R}$, denoted as $T_G=(\mathcal{A}, \mathcal{R})$.
%\end{definition}

%Generally, both KG and HIN are graph-based models for representing relational data.
%The HIN provides this constraint via labeling function $\Phi$, requiring each node to correspond to a single node type, whereas the description of node types is not necessary for the general KG.
%The relationship between KG and HIN is a subject of debate in academia.
%A recently published article by


%\begin{definition}
%A \textbf{meta-path}\cite{sun2011pathsim} $\mathcal{P}$ is a path defined on the graph of network schema $T_G=(\mathcal{A}, \mathcal{R})$, and is denoted in the form of $A_1 \stackrel{R_1}{\longrightarrow} A_2 \stackrel{R_2}{\longrightarrow} \ldots \stackrel{R_l}{\longrightarrow} A_{l+1}$, which defines a composite relation $R=R_1 \circ R_2 \circ \ldots \circ R_l$ between type $A_1$ and $A_{l+1}$, where - denotes the composition operator on relations.
%\end{definition}




%A Knowledge graph(KG) $\mathcal{G}=(\mathcal{E},\mathcal{R},\mathcal{T})$ can be regarded as a kind of heterogeneous information network\lsx{?}, where
%$\mathcal{E}$
% $\mathcal{E} = \{e_1,e_2,\cdots,e_{|\mathcal{E}|}\}$
%is the \textit{entity} set,
% of $\mathcal{G}$
% and $|\mathcal{E}|$ is the number of the entities in $\mathcal{G}$;
%$\mathcal{R}$
% $\mathcal{R} = \{R_1,R_2,\cdots,R_{|\mathcal{R}|}\}$
%is the \textit{relation} set,
% in $\mathcal{G}$, which includes $|\mathcal{R}|$ different types of relations;
%and $\mathcal{T} \in \mathcal{E} \times \mathcal{R} \times \mathcal{E}$ is the \textit{fact} set, where the facts are represented as triples.
%A triple can be denoted as $<e_h,R,e_t>$ or $R(e_h,e_t)$,
%where $e_h$ is the head entity, $e_t$ is the tail entity, and these two entities are connected by the relation $R$ to form a \textit{fact} in $\mathcal{G}$.
% Under the closed world assumption (CWA)
% ~\cite{hogan2021knowledge}, the fact set includes all the existing relationships between any entity pair from $\mathcal{R} \times \mathcal{R}$.
First-order logic (FOL) offers a pivotal way to represent real-world knowledge for reasoning. Horn rules, as a special and typical case of FOL rules, propose to represent a target relation by a body of conjunctive relations.

\begin{definitionnew}
\label{def:horn_rule}
A \textbf{Horn Rule}, generally chain-like, is given as,
\begin{equation*}
    R_h(x,y) \leftarrow R_{b_1}(x,z_1) \circ \cdots \circ R_{b_l}(z_{l-1},y)
\end{equation*}
where, $R_h(x,y)$ signifies the rule head (target relation) that we wishes to reason and $R_{b_1}(x,z_1) \circ \cdots \circ R_{b_l}(z_{l-1},y)$ is the rule body (relation path). For simplicity, we denote a Horn rule as $R_h:\mathbf{R_b}$, where $\mathbf{R_b}=[R_{b_1}, \cdots, R_{b_l}]$. To reason $R_h$, the size of the rule space is $|\mathcal{B}_h|$.
Every \textbf{closed path} of such Horn rule is required to: 1) connect $(x,y)$ via the rule body, which is a sequence of relations $\mathbf{R_b}$, and 2) ensure $(x,y)$ are accessible directly via the target relation $R_h$. Closed paths are also known as \textbf{rule instances}.
\end{definitionnew}

%Each KG corresponds to a concept set $\mathcal{C}$, and each concept $C \in \mathcal{C}$ corresponds to a concept-specific entity set  $\mathcal{E}_C \in \mathcal{E}$.
%Therefore the concept can be treated as the label of the entity class.


\subsection{Problem Statement}
The goal of this work is to learn the \textbf{causal rule} that is formalized as the horn rule.
Specifically, the objective of traditional logical rule learning is to assign a plausibility score $\mathbf{S}(R_h|\mathbf{R_b})$ to each rule in the discovered rule space, which can be subsequently aggregated to answer queries about the KG.
Currently, plausibility scores are defined over closed paths (e.g., the PCA confidence for AMIE~\cite{galarraga_amie_2013}), which are correlational observations.
We have demonstrated that these scores are prone to spurious correlations and therefore result in inaccurate predictions under OoD settings, in Sec. \ref{section:introduction}.
Therefore the other aim is to give a plausibility score based on causal strength.


In this paper, the causal rules are mined for link prediction in KG.
We follow the commonly accepted problem definition of link prediction in KG~\cite{rossi2021knowledge,tiwari2021revisiting}:
given an observed KG $\mathcal{G}$ with missing facts, our goal is to predict the correct entity for an given query $(e_h,R_h,?)$ (or $(?,R_h,e_t)$).
%For simplicity, we refer to the known entity in the prediction as source entity and the entity to be predicted as target entity.
% Specifically, the objective of logical rule learning is to assign a plausibility score $\mathbf{S}(R_h|\mathbf{R_b})$ to each rule in the discovered rule space, which are subsequently aggregated to rank all possible answers. Currently, plausibility score are defined over closed paths (e.g., the PCA confidence for AMIE~\cite{galarraga_amie_2013}), which are associational observations. We elaborate these scores are prone to spurious correlations and therefore result in inaccurate predictions under OoD settings, in Sec. \ref{section:model}. 