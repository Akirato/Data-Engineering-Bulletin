In recent years, the explosive increase of information facilitates the massive knowledge graphs, which in turn increase burden of people to understand and leverage the regularity behind these superficial facts.
Therefore, the metaknowledge, defined as the knowledge about knowledge, is proposed to identify complex processes of knowledge production and consumption.
Unfortunately, even though the current correlation-based rule mining methods in knowledge graph distill the rule-formed metaknowledge, they can not explain the processes of knowledge production.
In this paper, we focus on capturing the metaknowledge with causality which is generally regarded as one of the most promising techniques to reveal the interactions between components in the complex system.
To the best of our knowledge, this is the first attempt to interpret the knowledge graph from the causal perspective.

% Learning to predict missing links is a critical task for many knowledge graph-based applications. 
% The existing representation-based learning methods, which achieves promising results thanks to the success of deep learning technologies, are difficult to be applied in the sensitive areas, such as drug discovery, due to their uninterpretability.
% In contrast, the rule-based methods tend to be more interpretable.
% However, both representation-based and rule-based approaches focus on the \textit{association} between connection information of entity pairs and predicted links, and are susceptible to the spurious correlation. This makes the performance of these algorithms significantly degrade when the data distribution changes.

For this purpose, we propose a causal metaknowledge method for link prediction, which achieves entity-level link prediction by discovering concept-level topological causality. 
Specifically, we first formalize causal metaknowledge as causal rule, following the form of logical rule.
Then, we transform the relational data into propositional data to learn the causal relationships between topological structures.
And an efficient algorithm for discovering \emph{local} causal relationships is proposed using the $d$-seperation criterion.
Eventually, the causal rules generated based on the mined relationships are used for link prediction.
Both simulation-based and real data-based experiments demonstrate the effectiveness of the proposed approach, especially under the Out-of-Distribution(OoD) settings.