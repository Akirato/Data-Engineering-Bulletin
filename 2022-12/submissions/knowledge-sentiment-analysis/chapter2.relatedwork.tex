\section{Related Work}


Research related to implicit sentiment analysis is not fully developed compared to the field of explicit sentiment analysis\cite{erik2017sentiment,liu2020sentiment}.
For implicit sentiment, it can be divided into factual implicit sentiment and rhetorical implicit sentiment.
Rhetorical implicit sentiment can be subdivided into metaphorical or simile, rhetorical question, and ironic \cite{liao2019identification}.
Early research on implicit sentiment analysis focused on the task of identifying metaphors \cite{LiaoJian2018}, and as early as last century, Wilks et al. summarized metaphor, a common linguistic phenomenon, into a series of linguistic rules from the traditional cognitive and linguistic perspectives, through which metaphorical expressions and contexts, with the connections between them, are expressed.
Building on this theory, Mason et al. \cite{mason2004cormet} proposed a CorMet mapping of ``source domain-target domain" that can be identified by a specific domain, based on the linguistic feature that words can express different meanings in different domains. The CorMet model, which identifies between specific domains, was proposed.
Later, Lakoff \cite {J2008LakoffGeorge} studied the process of metaphorical expression in depth and defined it as a conceptual mapping from the source domain to the target domain in terms of cognition, common sense, etc.
In an earlier period, sentiment lexicon based methods played an important role in the field of sentiment analysis, but for implicit sentiment, the method could not be used directly for simple matching due to the lack of obvious sentiment words in the text variety.
Shutova et al.\cite{shutova2013statistical} addressed this problem to some extent by using clustering to obtain the abstract metaphors embedded in words; based on this work, the researchers went further to study the distribution of metaphorical concepts by combining weakly supervised and unsupervised approaches \cite{shutova2017multilingual}, and after experimentally testing several hierarchical clustering methods under specific constraints, a metaphorical pattern recognition algorithm was proposed, which achieved good results on a multilingual test set.
Zhang et al.\cite{zhang2011identifying} used sufficient collation and induction of sentiment texts to find that some phrases and special collocations, although without obvious sentiment cues, can implicitly express some implicit sentiment tendencies when they are expressed in sentences through metaphor or irony, and proposed a method to identify nouns with implicit sentiment domain features.
Balahur et al.\cite{balahur2011detecting}, on the other hand, focused on identifying the emotional tendency of the target with the help of contexts that do not contain emotion words, proposed an implicit sentiment analysis method based on common sense knowledge, and constructed an EmotiNet common sense knowledge base with relevant sentiment concepts; in subsequent work \cite{ balahur2012detecting}, the authors continued to improve their modeling method based on this common sense knowledge base and found that the sentiment of the target sentence could be discriminated from the context where all sentiments were implicitly expressed, and the experimental results showed that the method could effectively improve the effect of sentiment analysis.
Zhao et al.\cite{zhao2012collocation} took an alternative approach by introducing pseudo-contextual information into implicit sentiment analysis, and obtained the best results at that time in comparison experiments.
Tong et al. \cite{tong2013can} developed a theory-based discriminant system to analyze the relevance of ambiguous expressions to influence the euphemism of implicit affective utterances.
As deep learning methods are widely used in the field of sentiment analysis, many scholars have also started to use deep learning for implicit sentiment analysis tasks. Pulkit et al.\cite{mehndiratta2017detection} investigated recognition algorithms oriented to sarcastic expressions by using deep learning models. Qian et al.\cite{qian2018hierarchical} applied the idea of hierarchical classification to the task of identifying hate speech, and the method can effectively classify thirteen classes of hate speech categories.


In China, more and more scholars have started to focus on research targeting Chinese implicit sentiment analysis tasks in recent years.
Liao et al.\cite{liao2017freerl} first deconstructed the target text using syntactic analysis methods and proposed a representation learning framework based on tree convolution; immediately afterwards, the researchers in their paper \cite{LiaoJian2018,liao2019identification} provided a review of the work on Chinese implicit sentiment analysis, focusing on the characteristics of factual implicit sentiment; and based on word-level sentiment goals, sentence-level implicit sentiment expressions, and chapter-level contextual explicit sentiment tendencies, a multi-level semantic fusion representation learning approach was proposed for language modeling of factual implicit sentiment on two manually labeled factual implicit sentiment datasets. Experiments were conducted to verify that contextual information plays an important role in implicit sentiment recognition.
Wei et al. \cite{wei2020BiLSTM} proposed a multi-polar orthogonal attention model MPOA by comparing the differences between affective tendency features in social media texts, which also introduced a BERT pre-trained language model to represent the texts, and the experimental results showed that the model can effectively bridge the differences between words and affective tendency polarity corresponding to attention.
Zhao et al.\cite {ZhaoRongMei2020} proposed a hybrid neural network model based on CNN-BiLSTM-Attention, which extracts text features by CNN, obtains contextual information by Bi-LSTM. In addition, the model enables the extraction of semantic and structural features from the word and sentence levels respectively,  and highlights emotional information features that contribute more to emotional expressions by using the attention mechanism, and this hybrid model has improved the effectiveness of Chinese emotional utterance analysis.
Zuo et al. \cite{zuo2020context} proposed a context-specific heterogeneous graph convolution model (CsHGCN) , which introduces the idea of heterogeneous graphs into implicit sentiment analysis tasks and builds a contextual representation framework through graph convolutional neural networks, with improved results compared to network approaches such as tree graph convolution and tree long and short-term memory networks.
Yang et al. \cite {YangShanLiang2021} proposed an implicit sentiment analysis model based on graph attention neural network, by constructing a heterogeneous graph of text and word correspondences, aggregating semantic information using graph convolutional networks, and calculating the contribution of words with to the sentiment expression of the text using an attention mechanism, thus allowing the model to focus on words that are more important for sentiment polarity.
Chen et al.\cite {ChenQiuChang2022} proposed a parallel hybrid model of bidirectional long-short time neural network and context-aware tree recurrent neural network (CA-TRNN) for the problem of inaccurate feature information extraction in Chinese implicit sentiment analysis and gradient explosion or gradient disappearance for chapter-level text information extraction by existing serialization models, which effectively improved the accuracy of classification results with small time cost and better application capability.
In addition to digging deeper into the semantic information in Chinese implicit sentiment texts, the recognition of implicit sentiment usually requires the introduction of external sentiment features and sentiment knowledge.
Wang et al. \cite{wang2020chinese} fused three levels of semantic information in sentiment expressions, and proposed a method based on hierarchical knowledge enhancement to alleviate the problem of ``weak features" in Chinese implicit sentiment expressions, while introducing a multi-pooling approach to model the ``multiple confusion weak features" problem. The accuracy of the model has been effectively improved, and the F1 score is 5.9\% higher than the previous best model.

From the above review of existing research, it is clear that, in contrast to the task of explicit sentiment analysis, research on implicit sentiment analysis is still immature, remains in the exploratory stage and faces a large number of problems and challenges.
