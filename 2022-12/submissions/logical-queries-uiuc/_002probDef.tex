In this section, we formally give the definition of this paper. Table ~\ref{table1} gives the notations used in our paper.
A knowledge graph can be treated as set of triples $<Subject, Predicate, Object>$ ($<s, p, o>$ for short) where "Subject" and "Object" are the entities in the knowledge graph and "Predicate" are the relationship between two entities. If we treat the knowledge graph as an attributed graph, the entities will correspond to the nodes in the attributed graph, and predicates correspond to the attributed edges. So, We can also use $G=(V, E, L)$ to denote the knowledge graph where $V$ is the set of entities or nodes, $E$ is the set of predicates or edges and $L$ is the label set of nodes and edges. We will use both of these two definitions in our paper.


\begin{table}[]
	\centering
	\caption{Notations and Definition}
	\vspace{-0.5\baselineskip}
	\begin{tabular}{|c|l|}
		\hline
		Symbols       & Definition                \\ \hline
		$\mathcal{Q}$=\{$V_Q$, $E_Q$, $L_Q$\} & an attributed query graph \\ \hline
		$\mathcal{G}$=\{$V_G$, $E_G$, $L_G$\} & an attributed data graph  \\ \hline
		$\mathcal{H}_{i}$         & a partial matching with $i$ nodes \\ \hline
		$\mathcal{Q}_{i}$         & a partial query graph with $i$ nodes \\ \hline
		$u$, $\hat{u}$      & nodes in query graph      \\ \hline
		$v$, $\hat{v}$      & vertices in data graph       \\ \hline
		m($u_i$) = $v_i$ & matching function       \\ \hline
		N($u$)          & neighborhood of query node $u$  \\ \hline
		C($u$)          & candidate set of query node $u$  \\ \hline
		f($\mathcal{Q}$, $\mathcal{H}$)          & the loss function     \\ \hline
	\end{tabular}
\label{table1}
\vspace{-0.8\baselineskip}
\end{table}