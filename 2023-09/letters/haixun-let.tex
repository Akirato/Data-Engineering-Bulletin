\documentclass[11pt]{article} 

\usepackage{deauthor,times,graphicx}
%\usepackage{url}
\usepackage{hyperref}

\begin{document}

This special issue of the IEEE Data Engineering Bulletin is dedicated to a timely subject: high-dimensional similarity searches. 

High-dimensional similarity searches are indispensable in a variety of fields. For instance, they are employed for time series analysis and forecasting, which have numerous applications in science, medicine, and business. Recently, the subject of vector databases has garnered significant attention, primarily driven by the emergence of Large Language Models (LLMs) and Retrieval Augmented Generation (RAG), where  vector databases have assumed a crucial role in facilitating various applications in the domain of generative AI ranging from  chatbots to AI agents. On the other hand, the continuous progress and widespread adoption of vector databases have been occurring over the past few decades. Specifically, it brought about a revolution in the domain of information retrieval by enhancing the traditional approach of term-based retrieval with embedding-based retrieval. This novel technique, which involves utilizing vector-based semantic search, significantly enhances the ability to retrieve relevant information. 

This special issue, curated by Associate Editor Themis Palpanas,  gathers insights from various branches of this field. The articles contained within this special issue provide an array of theoretical and practical solutions for similarity searches in high-dimensional spaces. Our authors investigate algorithms to improve similarity searches, explore the evolution of graph-and tree-based indexes, and chart the course for improving data management systems. The articles delve into aspects of optimal design strategies to enhance computational efficiency in the era of AI, examine the deployment of techniques like locality-sensitive hashing and product quantization, and delve into the improvements dynamic space partitions can offer for faster and more accurate searches.

We believe the topic of vector databases is significant for the database community, for its role as a bridge connecting data management and artificial intelligence.  While considerable progress has been made in the field of vector-based similarity search, it is crucial to recognize the existence of ongoing obstacles that require attention and resolution. For example, there exist numerous relationships among the vast amount of data represented by the vector database. Given a question, how do we ensure all pertinent information is retrieved through vector search?  Additionally, what strategies can be employed to narrow the disparity between pre-training an LLM on the vast amount of data and in-context learning via RAG, which disregards the majority of the data that could potentially have relevance to the given question? Thus, our shared objective is to enhance the advancement of this discipline, and our overarching aspiration encompasses two expansive domains: database and artificial intelligence. 

We would like to express our profound gratitude to all the authors who contributed to this issue, to Themis Palpanas for bringing these insightful articles to the forefront, and to Nurendra Choudhary for his assistance in the publication process.


\end{document}

