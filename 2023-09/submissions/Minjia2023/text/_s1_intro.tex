\section{Introduction}\label{minjia_sec:intro}

\subsection{Approximate Nearest Neighbor Search}

Finding the top-k nearest neighbors among database vectors for a query has long been a key building block to solve problems such as large-scale information retrieval and image search~\cite{lv2004image, philbin2007object, kulis2009kernelized}, recommendation~\cite{das2007google}, entity resolution~\cite{hoffart2012kore}, and sequence matching~\cite{berlin2015assembling}. As database size and vector dimensionality increase, exact nearest neighbor search becomes expensive and impractical due to latency and memory constraints~\cite{weber1998quantitative, beyer1999nearest, bohm2001searching}. Therefore, to reduce the search cost, various approximate nearest neighbor search (ANNS) algorithms have been proposed to improve efficiency substantially while mildly relaxing accuracy constraints, leading to the so-called accuracy-vs-efficiency tradeoffs. 

Over the years, a variety of algorithms for Approximate Nearest Neighbor Search (ANNS) have been developed with the goal of enhancing computational and memory efficiency. To improve the compute efficiency, well-designed indexes have been introduced, including tree structure-based~\cite{kd-tree,r-star-tree,flann}, hashing-based~\cite{lsh}, and proximity graph-based approaches~\cite{hnsw, nsg}. To improve memory efficiency, various compression algorithms have also been applied to ANNS, such as product quantization-based methods~\cite{product-quantization, opq, cartesian-kmeans, inverted-multi-index, lopq}. These methods can also be combined to improve both compute and memory efficiency simultaneously. For a more detailed understanding and comparison of ANNS algorithms, we recommend several literature that provide excellent surveys~\cite{ann-survey,li2020approximate,wang2021comprehensive}.  

\subsection{Modern Applications and Requirements}

ANNS holds significant relevance in contemporary applications, particularly in conjunction with deep learning models, as it facilitates innovative search scenarios. Traditional entity retrieval relies on keyword matching and user behavior signals. However, with the progression of deep learning, it is now possible to construct models that yield vectors with close distances for entity inputs sharing similar “views”. With these models, one then can encode unstructured data into embedding vectors in a high dimensional space $\mathds{R}^d$~\cite{dssm,multi-field-neural-ranking}. These vectors capture the similarities between various entities within the latent space. As a result, the nearest embeddings for a specific query often symbolize entities with similar semantics in the latent space. ANNS then emerges as a natural choice for managing these vectors while ensuring both speed and accuracy in retrieval.

Vector-based search has already been integrated into many modern applications. For instance, Web-scale search engines like Google~\cite{rankbrain} and Bing~\cite{sptag,diskann} utilize embeddings for documents (e.g., word2vec~\cite{word2vec} and doc2vec~\cite{doc2vec}) and images (e.g., VGG~\cite{vgg}) to retrieve semantically related entities in response to user queries. Major e-commerce players like Amazon~\cite{amazon-search} have developed recommendation systems that embed both the product catalog and the search query, recommending products whose embeddings are closest to the embedded search query. YouTube has built search engine that embeds videos to vectors for video recommendation~\cite{youtube-embed}. More recently, vector search has been employed in retrieval augmented generation in large language models (LLMs), where vector search can be used to expand LLMs knowledge by incorporating external data sources~\cite{retrieval-augmented-generation-deepmind}. Vector search also presents a fertile ground for exploring future applications. For instance, recent advancements in deep learning have enabled models to capture multimodal relationships, such as through the use of multimodal foundation models~\cite{multimodal}. Consequently, the underlying vector search systems can also leverage ANNS to handle multi-modality entities. However, how to effectively handle different modalities and capture the full range of interconnections and relationships among them via ANNS remains an open question. This includes whether various modalities benefit from using the same or different vector search methods, which is an exciting area for future exploration.

As vector search goes to a larger scale, where the dimension scales from $\sim$100 to $\sim$1000 and the number of vectors scales from millions to billions, the challenge of serving latency becomes more prominent even with novel ANNS algorithms. For instance, online interactive services (e.g., web search engine) often require responses to be returned within a few or tens of milliseconds, as delayed responses could degrade user satisfaction and affect revenue~\cite{reduce-web-latency}. However, as the number of entities (such as images and documents) grows rapidly and deep learning embeddings expand to higher dimensions (from embedding sentences to full documents), it becomes increasingly difficult to find highly accurate results in large datasets while adhering to latency constraints. Many vector search services, such as text and image search, require intensive computation and may not be feasible due to latency violations. Therefore, how to transform these applications from impossible to ship due to latency violation to well-fitting SLA is crucial for the practical adoption of vector search. Another big requirement for large-scale vector search is cost reduction. Large-scale services deal with a vast volume of requests and could necessitate thousands of machines for a single application. Therefore, decreasing the number of machines while maintaining the same search quality and latency is crucial for reducing the total cost of ownership for the application.

\subsection{New Hardware Architectures and Opportunities}

Existing ANN algorithms have mostly exploited the uni-core CPU infrastructure and standard memory hierarchy. This infrastructure uses processors whose performance increased with Moore's Law, thus limiting the need for high levels of concurrent execution on a single machine. However, processors are no longer providing ever higher uni-core performance. Meanwhile, the prior infrastructure used DRAM for main memory. However, the main memory capacity is often quite limited to hold a large volume of data. As multi-core processors become ubiquitous and new memory architecture such as heterogeneous memory becomes available, new opportunities for large-scale vector search exist:

\begin{itemize}
    \item \textbf{Design for multi-core}: Modern CPUs are often equipped with high-performance multi-core. Since uni-core speed has pretty much saturated, we need to get better at exploiting a large number of cores by addressing at least two important aspects:
    \begin{enumerate}
        \item Multi-core CPUs provide high concurrency, but as the level of concurrency increases, synchronization among different cores are more likely to block and limit scalability.
        \item The performance of multi-core processes also depend on the shared memory bandwidth utilization. According to the roofline model~\cite{roofline},  the performance of an application is not only bounded by the compute capability but also the bandwidth performance. So how to make the best utilization of memory bandwidth needs great care.
    \end{enumerate} 
    
    \item \textbf{Design for modern memory devices}:  Vector search at large scale is very memory consuming and easily runs out of memory with a few hundred millions of vectors. When the dataset becomes too large to fit on a single machine, one approach is to use the compressed representations of the database points, such as Hamming codes~\cite{hamming-distance} and product quantization~\cite{link-and-code,opq,product-quantization,lopq,cartesian-kmeans}. However, the performance of these methods deteriorates rapidly at higher recall targets, because they calculate approximate distance based on compressed vectors instead of on the original data vectors. Another approach is to exploit storage. In DiskANN~\cite{diskann}, the authors explore slow storage to achieve billion-scale ANNS in a single machine. However, disk latency is a major problem. While persistent media such as SSD offers lower latency and much higher I/O ops per second than traditional disks, they are still several orders of magnitude slower than DRAM. Based on this assumption, data access to the persistent media during search should be minimized. As a result, DiskANN maintains a copy of compressed data in memory with product quantization~\cite{diskann}, which results in loss of in-memory search quality. It then performs a re-ranking using full-precision coordinates stored on SSD, using block-level data accesses but with expensive SSD accessing time. While methods such as DiskANN show promising results, the emergence of Heterogeneous Memory (HM) brings opportunities to significantly improve ANNS. HM combines cheap, slow but extremely large memory with expensive, fast but small memory (e.g., traditional DRAM) to achieve a good balance between production cost, memory performance and capacity. Because of the large memory capacity, HM can use full-precision vectors with accurate distance computation. Since memory access latency/bandwidth of slow memory components in HM is much faster than slow storage such as SSD, it is possible to occasionally access data in slow memory during search without paying the expensive cost of data accesses. That being said, realizing the full performance potential of HM for ANNS is still quite challenging. Although slow memory such as PMM performs $\sim$80X times faster than SSD, it is still $\sim$3X slower than DRAM in terms of random access latency~\cite{pmm-perf}. Therefore, a naive data placement strategy can hurt the search efficiency badly. Therefore, one may still wonder if we can leverage HM for ANNS to achieve both high search accuracy and low search latency, especially when the dataset cannot fit in DRAM (fast memory)?
\end{itemize}

In this work, we revisit the similarity search problem in light of the recent advances in the field. Two new system optimization methods are introduced, dedicated to improving the efficiency and scaling of vector search while simultaneously delivering high accuracy. They are particularly appropriate for the new hardware architectures discussed above. Specially:
\begin{itemize}
    \item iQAN~\cite{iqan} is a parallel search algorithm that exploits intra-query parallelism in graph-based vector search to obtain significant latency reduction in vector search on multi-core architectures with high accuracy. This approach includes a set of optimizations that boost convergence, avoid redundant computations, and mitigate synchronization overhead.
    \item HM-ANN~\cite{hm-ann} is a heterogeneous memory-based technique that shatters the memory barrier of deploying large-scale vector search via NVMe memory, enabling billion-scale vector search with low deployment cost. It carefully constructs vector search indices via a memory hierarchy-aware algorithm, hence leading to substantially better search performance as the vectors grow to be larger than the DRAM capacity. It also employs parallel search algorithms to boost the in-memory search efficiency, leading to faster search speed.    
\end{itemize} 

In drawing broader lessons from this work, we believe that effectively leveraging multi-core and exploiting the memory hierarchy are the keys to high-performance vector search on modern processors. Further, based on the above methods, we discuss open research problems, including exploring hierarchical parallelism to meet both latency and throughput targets, highly concurrent vector search with addition and deletion, automating the index construction for vector search, and the interactions with modern applications in Section~\ref{minjia_sec:future}. 






