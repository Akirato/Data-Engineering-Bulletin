
\subsection{Challenges}
there are several challenges and potential problems that could emerge in the future use and development of vector databases. 

\subsubsection{Scalability}
As we continue to generate and collect massive amounts of data in various domains, the scalability of vector databases becomes a critical concern \cite{DBLP:journals/tbd/JohnsonDJ21,DBLP:journals/jcisd/TingleTCGKDMI23}. 
Consider the case of a global e-commerce platform, such as Taobao. Every day, Taobao collects vast amounts of data on user behavior, product details, transactions, and much more.
%The challenge of scalability is twofold: accommodating the explosively growing volume of high-dimensional data, and maintaining efficient, rapid query responses.
%
%\textbf{Handling High Volumes of Data}
%The digital universe is expanding at an unprecedented rate. As more industries and sectors embrace the use of machine learning and AI technologies, high-dimensional vector data is being produced and collected at a scale never seen before. In this context, vector databases must be designed to not only store but manage and manipulate these large volumes of data efficiently.
The traditional approach to handling large data volumes is to distribute the data across multiple nodes or servers \cite{DBLP:conf/cidr/CherniackBBCCXZ03}. However, distributing high-dimensional vector data is not straightforward. Unlike traditional relational data, where each record is independent and can be easily partitioned, high-dimensional vectors often need to be compared with one another during the similarity searches \cite{DBLP:journals/access/YangZCY19}, which complicates the distribution process. 
%
%
%\textbf{Maintaining Efficient Query Responses}
%Scalability is not just about storing more data; it's equally about delivering efficient query responses on this large-scale data. As the volume of data increases, ensuring that queries are processed and results are returned in a timely manner becomes increasingly challenging. 
Moreover, ANNS algorithms are commonly used in vector databases to speed up the search process. 
Existing algorithms use in-memory indexes for low latency and high throughput. However, as the data scales, these algorithms may still suffer from increased latency since it becomes expensive to load all the date into the memory and disk-based indexes must be considered. 
Furthermore, the infrastructure supporting the vector database will play a significant role in query performance. Techniques such as data sharding, load balancing, and efficient use of hardware will be crucial in ensuring that the system can handle high-concurrency, low-latency queries. 
DiskANN algorithms \cite{NEURIPS2019_09853c7f, DBLP:journals/corr/abs-2211-12850} index 5-10 times more points/machine using inexpensive SSDs with less than 10ms latency, which is close to the query latency in the in-memory indexes.
As we move into a future characterized by increasingly large volumes of high-dimensional data, research and development in this area will be crucial. Both data distribution strategies and query optimization techniques will need to evolve in tandem with the growing demands placed on these systems.


\subsubsection{Data Privacy and Security}
In the era of big data and machine learning, the issue of data privacy and security has taken center stage \cite{DBLP:conf/trustcom/GheidC16,DBLP:conf/ccs/WangLWTZ09}. 
%
%This is no less true for vector databases, where the high-dimensional data stored can often represent sensitive information. Ensuring the privacy and security of this data is a paramount concern, involving not just protection from unauthorized access, but also careful consideration of the privacy implications intrinsic to vector representations.
%
%\textbf{Protecting Vector Databases from Unauthorized Access.}
Like any database system, a vector database is potentially vulnerable to unauthorized access and data breaches. As vector databases become more widely used and the data they store becomes increasingly valuable, they will inevitably become attractive targets for cyber-attacks \cite{8355379}. This necessitates robust security measures to prevent unauthorized access.
Security protocols should include strong access controls, secure data transmission, and encryption strategies to protect the stored data. Moreover, constant monitoring and audit trails can help detect any suspicious activities. Implementing these measures requires a deep understanding of both database security and the specific vulnerabilities that may be unique to vector databases.
%
%\textbf{Privacy Implications of Vector Representations.}
Beyond the conventional concerns of data security, vector databases present another dimension of privacy challenge: the potential sensitivity of the vector representations themselves. High-dimensional vectors in these databases often capture meaningful patterns and structures in the data they represent. For instance, a vector could represent a user's behavior patterns, a patient's medical record, or a document's semantic content. This raises significant privacy concerns, as individuals might be identifiable from their corresponding vectors, even when the original data is anonymized.
A related concern is the potential for 'information leakage' from the vector representations . In some scenarios, it might be possible to reverse-engineer sensitive information from the vectors, especially if the method used to generate the vectors is known \cite{DBLP:journals/corr/abs-2205-10364}. This requires careful consideration of how vectors are generated and how they can be sufficiently anonymized or perturbed to prevent such leakage.
%These challenges will necessitate the development of new privacy-preserving techniques tailored to the context of vector databases. This might include methods for differential privacy, a mathematical technique that ensures the outputs of database queries do not reveal sensitive information about any individual record, or cryptographic techniques like homomorphic encryption that allows computations on encrypted data.
%%In conclusion, as the use of vector databases expands, the challenges of data privacy and security become more pressing. Future research and development in this area need to address both the traditional aspects of data security and the unique privacy concerns raised by high-dimensional vector representations. This will be critical in ensuring that vector databases can be used responsibly and ethically in a wide range of applications.

\subsubsection{Hardware Efficiency}
The growing dependence on vector databases for high-dimensional data processing has highlighted the importance of hardware efficiency. These databases often utilize approximate nearest neighbor (ANN) search algorithms to find similar vectors, resulting in a significant computational load. This challenge is further amplified in distributed systems that rely on GPUs, DRAM, and other hardware components, where fully utilizing hardware capabilities is crucial for performance but often difficult to achieve.
Recently, GPU-based ANN indexes such as SONG \cite{DBLP:conf/icde/ZhaoTL20} and GGNN \cite{DBLP:journals/tbd/GrohRWL23} have been proposed, achieving a two orders of magnitude speedup compared to CPU-based methods for ANN searches. However, there are two factors to consider when delegating distance computation to GPUs. First, GPUs require interaction with the host's software and/or hardware layers, resulting in data transfer overhead for computation. Second, ANNS distance computations can be performed using a few simple, lightweight vector processing units, making GPUs a less cost-efficient choice for these tasks.
In addition, a software-hardware collaborative approach can combine software and hardware components to achieve highly scalable approximate nearest neighbor search services. A study called CXL-ANNS \cite{DBLP:conf/usenix/JangCBLKJ23} disaggregates DRAM from the host via Compute Express Link (CXL) and places all essential datasets into its memory pool. Another study, FANNS \cite{DBLP:journals/corr/abs-2306-11182}, automatically co-designs hardware and algorithms on FPGAs when given a user-provided recall requirement on a dataset and a hardware resource budget. Compared to purely CPU- or GPU-based methods, these software-hardware collaborative approaches achieve superior performance.

%
%\textbf{GPU-based methods.} Recently, GPU-based ANN indexes such as SONG \cite{DBLP:conf/icde/ZhaoTL20} and GGNN \cite{DBLP:journals/tbd/GrohRWL23} have been proposed, achieving a two orders of magnitude speedup compared to CPU-based methods for ANN searches. However, there are two factors to consider when delegating distance computation to GPUs. First, GPUs require interaction with the host's software and/or hardware layers, resulting in data transfer overhead for computation. Second, ANNS distance computations can be performed using a few simple, lightweight vector processing units, making GPUs a less cost-efficient choice for these tasks.
%
%\textbf{Software-hardware collaborative methods.} A software-hardware collaborative approach combines software and hardware components to achieve highly scalable approximate nearest neighbor search services. A study called CXL-ANNS \cite{DBLP:conf/usenix/JangCBLKJ23} disaggregates DRAM from the host via Compute Express Link (CXL) and places all essential datasets into its memory pool. Another study, FANNS \cite{DBLP:journals/corr/abs-2306-11182}, automatically co-designs hardware and algorithms on FPGAs when given a user-provided recall requirement on a dataset and a hardware resource budget. Compared to purely CPU- or GPU-based methods, these software-hardware collaborative approaches achieve superior performance.







